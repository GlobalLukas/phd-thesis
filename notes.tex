\documentclass[11pt]{report}
\usepackage[utf8]{inputenc}
\usepackage[english]{babel}
\usepackage{hyphenat}
\usepackage{amsmath}
\usepackage{amsfonts}
\usepackage{amssymb}
\usepackage{amsthm}
\usepackage{imakeidx}
\usepackage{hyperref}
\usepackage{enumitem}
\setlist[enumerate]{itemsep=-1mm}
\usepackage[boxsize=6pt]{ytableau}
\usepackage{tikz-cd}
\usepackage{tikz}
\usepackage{stmaryrd}
\usetikzlibrary{decorations.markings,arrows,arrows.meta}
\setlist[enumerate]{itemsep=0mm}
\usepackage[a4paper,left=25mm,right=25mm,top=25mm,bottom=35mm]{geometry}
\usepackage{accents}
\newcommand*{\dt}[1]{%
  \accentset{\mbox{\large\bfseries .}}{#1}}
\newcommand*{\ddt}[1]{%
  \accentset{\mbox{\large\bfseries .\hspace{-0.25ex}.}}{#1}}

\linespread{1.275}

\counterwithout{figure}{chapter}

\tikzset{every path/.style={thick}}

\newtheorem{theorem}{Theorem}[section]
\newtheorem{lemma}[theorem]{Lemma}
\newtheorem{prop}[theorem]{Proposition}
\newtheorem{corollary}[theorem]{Corollary}
\newtheorem{conjecture}[theorem]{Conjecture}

\theoremstyle{definition}
\newtheorem{definition}[theorem]{Definition}

\theoremstyle{remark}
\newtheorem*{remark}{Remark}

\theoremstyle{remark}
\newtheorem*{example}{Example}

\newenvironment{claim}[1]{\par\noindent\textit{Claim.}\space#1}{}
\newenvironment{claimproof}[1]{\par\noindent\textit{Proof of claim.}\space#1}{\hfill $\Diamond$}

\newcommand{\Diff}{\operatorname{Diff}}
\newcommand{\Hom}{\operatorname{Hom}}
\newcommand{\End}{\operatorname{End}}
\newcommand{\Der}{\operatorname{Der}}
\newcommand{\coim}{\operatorname{coim}}
\newcommand{\im}{\operatorname{im}}
\newcommand{\coker}{\operatorname{coker}}
\newcommand{\id}{\textnormal{id}}
\newcommand{\N}{\mathbb{N}}
\newcommand{\Z}{\mathbb{Z}}
\newcommand{\Q}{\mathbb{Q}}
\newcommand{\R}{\mathbb{R}}
\newcommand{\C}{\mathbb{C}}
\newcommand{\A}{\mathbb{A}}
\newcommand{\F}{\mathbb{F}}
\newcommand{\I}{\mathrm{i}}
\newcommand{\llbraces}{\{\!\!\{}
\newcommand{\rrbraces}{\}\!\!\}}
\renewcommand{\P}{\mathbb{P}}

\makeindex[intoc]

\title{
\huge \textsc{PhD}
}
\author{
-------- \\~\\~\\
%Doktorgrad
\\~\\~\\~\\~\\~\\~\\~\\~\\
}
\date{
\begin{tabular}{ll}
Autor: & Lukas Johannsen \\
Erstgutachter: & Prof. Dr. Gleb Arutyunov \\
Zweitgutachter: & ? \\
Ort und Datum: & Hamburg im ? 2027
\end{tabular}
}

\begin{document}
\maketitle

~

\thispagestyle{empty}
\setcounter{page}{0}

\pagebreak

\chapter*{Abstract}

\section*{Acknowledgments}

%\footnotetext{This thesis has benefited from large language models.}

\tableofcontents

\chapter{Notes}

\section{Basics of elliptic structures}

\subsection{Elliptic functions and theta functions}

\begin{definition}
An elliptic curve $E$ (over $\C$) is a smooth projective curve or Riemann surface of genus 1. These are of the form $E \cong \C/\Lambda$ for $\Lambda = \Z \oplus \tau \Z$ for $\tau \in \mathbb{H} = \{ z \in \C \mid \Im z > 0 \}$. This does not faithfully parametrize elliptic curves, but $M_{1,1} := \mathbb{H}/SL_2(\Z)$ does, where $SL_2(\Z)$ acts by Möbius transformations. Algebraically, every elliptic curve may be brought into the form
\begin{equation*}
Y^2 Z = 4 X(X-Z)(X-\lambda Z),
\end{equation*}
where $\lambda$ is the $\lambda$-invariant of $E$, which is also not faithful up to an action of $SL_2(\Z)/\Gamma(2) \cong S_3$. The invariant combination
\begin{equation*}
j = 2^8 \frac{(\lambda^2-\lambda+1)^3}{\lambda^2(\lambda-1)^2}
\end{equation*}
is the $j$-invariant, yielding a bijection $j: M_{1,1} \to \C$. Stacky points are at $j=0$ and $j=12^3=1728$, where $\Lambda$ becomes the lattice of the Eisenstein and Gauss integers, respectively, with automorphism groups $\Z_2$ and $\Z_3$ (plus the usual involution $Y \mapsto -Y$, giving $\Z_4$ and $\Z_6$). There is an isomorphism
\begin{equation*}
\C/\Lambda \to E(\C), \quad z+\Lambda \mapsto
\begin{cases}
[\wp(z|\tau):\wp'(z|\tau):1], & z \notin \Lambda \\
[0:1:0], & z \in \Lambda.
\end{cases}
\end{equation*}

An elliptic function is a meromorphic function $f: \C \to \C$ that is $\Lambda$-periodic such that it descends to a meromorphic function on $E$. Theta functions are sections of certain line bundles over $E$, which may be represented as entire functions $\vartheta: \C \to \C$ satisfying
\begin{equation*}
\vartheta(z+1|\tau) = \vartheta(z|\tau), \quad \vartheta(z+\tau|\tau) = \exp(- \pi \I \tau - 2 \pi \I z) \vartheta(z|\tau)
\end{equation*}
This line bundle has in fact only one section up to a prefactor, and this is the Jacobi theta function.
\end{definition}

\subsection{Belavin's elliptic $R$-matrix}

\cite{article:etingof:1998}

\begin{definition}
Let $\xi = \exp(2\pi\I/\ell)$. Define a projectively flat rank $\ell$ vector bundle on $E = \C/\Lambda$ by the two monodromies
\begin{equation*}
A =
\begin{pmatrix}
1 & 0 & \cdots & 0 \\
0 & \xi & \cdots & 0 \\
\vdots & \vdots & \ddots & \vdots \\
0 & 0 & \cdots & \xi^{\ell-1} \\
\end{pmatrix}
,\quad
B = \exp(-\pi\I \tfrac{\ell-1}{\ell})
\begin{pmatrix}
0 & 1 & \cdots & 0 \\
\vdots & \vdots & \ddots & \vdots \\
0 & 0 & \cdots & 1 \\
1 & 0 & \cdots & 0 \\
\end{pmatrix},
\end{equation*}
satisfying $A^\ell,B^\ell = 1$ and $BA = \xi AB$ giving a flat/holomorphic $PGL_\ell$-bundle $P \to E$. Belavin's elliptic $R$-matrix is the unique $R$-matrix satisfying the QYBE and unitarity as well as
\begin{enumerate}[label=(\roman*)]
\item $R^B(z)$ has simple poles only at $\eta + \Lambda$,
\item $R^B(0) = P$,
\item $R^B(z+1) = A_1 R^B(z) A_1^{-1} = A_2^{-1} R^B(z) A_2$, $R^B(z+\tau) = B_1 R^B(z) B_1^{-1} = B_2^{-1} R^B(z) B_2$.
\end{enumerate}
In particular, the $R$-matrix lives on the $\ell$-fold cover $\bar E = \C/\ell \Lambda$, where $P$ is trivialized. We may view the $R$-matrix as an element of
\begin{equation*}
\operatorname{End} \C^\ell \otimes \Gamma(E,P \ltimes \operatorname{End} \C^\ell),
\end{equation*}
where $\Gamma(E,P \ltimes \operatorname{End} \C^\ell)$ are meromorphic sections of the associated bundle $P \ltimes \operatorname{End} \C^\ell$.
\end{definition}

\subsection{RTT representations}

\cite{article:etingof:1998}

\begin{definition}
Define the category $\mathsf{C}_B$ whose objects are vector spaces $V$ equipped with an invertible element
\begin{equation*}
L(z) \in \operatorname{Mat}_\ell(\operatorname{End} V \otimes \mathcal{M}(\bar E)),
\end{equation*}
thus having values in $\operatorname{End} \C^\ell \otimes \operatorname{End} V$, satisfying
\begin{equation*}
R_{12}^B(z-w) L_1(z) L_2(w) = L_2(w) L_1(z) R_{12}^B(z-w).
\end{equation*}
Morphisms are linear maps $f: V \to V'$ satisfying $\varphi L(z) = L'(z) \varphi$. There is a tensor structure via
\begin{equation*}
(V,L(z)) \otimes (V',L'(z)) := (V \otimes V', L_{12}(z) L_{13}'(z))
\end{equation*}
and finite-dimensional objects $(V,L(z))$ have duals $(V^*,L^*(z))$ with $L^*(z) = (L(z)^{-1})^{t_2}$. Belavin's $R$-matrix ensures the existence of a vector representation $(\C^\ell,R^B(z))$.
\end{definition}

\section{Elliptic Drinfeld functor}

In order to define an elliptic Drinfeld functor, we first need an analog of the Yangian representation on $\C^\ell[y]$. For fixed $z$, we can view the coefficients of $R^B(z-y)$ as meromorphic $\operatorname{End} \C^\ell$-valued functions $f_{ij}(y)$ with at most a simple pole at $z-\eta + \Lambda$. They satisfy $f_{ij}(y+1) = \operatorname{Ad}_A^{-1} f_{ij}(y)$ and $f_{ij}(y+\tau) = \operatorname{Ad}_B^{-1} f_{ij}(y)$ and are thus meromorphic sections of the adjoint bundle $\operatorname{Ad} P$. These naturally act on sections of the bundle associated to $\C^\ell$. Let us abbreviate the space of such sections as
\begin{equation*}
\Theta := \{ f: \C \to \C^\ell \text{ meromorphic} \mid f(y+1) = A^{-1} f(y), f(y+\tau) B^{-1} f(y) \}.
\end{equation*}
We obtain an RTT representation on $\Theta$. More generally, we may define
\begin{equation*}
L_N(z) := R_{01}^B(z-y_1) \cdots R_{0N}^B(z-y_N)
\end{equation*}
whose coefficients act on the space
\begin{equation*}
\Theta_N := \{ f: \C^N \to (\C^\ell)^{\otimes N} \text{ meromorphic} \mid f(y_i+1) = A_i^{-1} f(y_i), f(y_i+\tau) B_i^{-1} f(y_i) \}.
\end{equation*}
These are the sections of a vector bundle $V_N \to E^N$, which can be pulled back to the ($\eta$-deformed) configuration space of points of $E$ or even on $M_{1,1+N}$ such that the $R$-matrix allows us to put an RTT representation on its sections.

There is a commuting action of $S_N$ from the right on $\Theta_N$ via $R$-matrices: $(i \ j) \mapsto R_{ij}^B(y_i-y_j)$, as well as sections of the structure sheaf of the configuration space of points on $E$. Together, these form a generalization of the degenerate affine Hecke algebra. If these sections of the structure sheaf are replaced by the theta functions of \cite{article:hasegawa:1995}, we may let the elliptic difference operators act. This space is
\begin{equation*}
Th^l = \{ f: \C^N \to \C \text{ holomorphic} \mid f(y+e_i) = f(y), f(y+\tau e_i) = \exp(-\pi\I l\tau-2\pi\I l y_i) f(y) \}.
\end{equation*}
Then the $S_N$-invariant subspace is spanned by the ${N+l \choose l}$ $\hat{\mathfrak{gl}}_N$-characters of level $l$. This defines a line bundle $L$ on $E^N$.

All in all, we may twist the vector bundle $V_N$ by $L$, obtaining $V_N \otimes L \to E^N$ and we have an action of $S_N$ giving the descent data for a vector bundle $W$ on the quotient stack $E^N \sslash S_N$. Its space of meromorphic sections gives an object in $\mathsf{C}_B$ via $L_N(z)$. This defines a functor from quasi-coherent modules on $E^N \sslash S_N$ to $\mathsf{C}_B$.

Define the space
\begin{equation*}
\Theta_{\ell,N}^l = \{ f: \C^N \overset{\text{mer}.}\longrightarrow (\C^\ell)^{\otimes N} \mid f(y_i+1) = A_i^{-1} f(y_i), f(y_i+\tau) = \exp(-\pi\I l\tau-2\pi\I l y_i) B_i^{-1} f(y_i) \}.
\end{equation*}
These are sections of a vector bundle on $E^N$ and we have actions
\begin{equation*}
E(\mathfrak{gl}_\ell) \curvearrowright \Theta_{\ell,N}^\bullet \curvearrowleft S_N \ltimes Th^\bullet
\end{equation*}
via $L_N(z)$, permutations acting via $R$-matrices, and theta functions acting by scalar multiplication. This is graded by the level $l$. We now want to compute $\Theta_{\ell,N}^l \otimes_{S_N \ltimes Th^l} Th^l$. This is done by projecting out the action of $S_N$ on $\Theta_{\ell,N}^\bullet$. This is the Hilbert space of the elliptic spin RS model, you might call it the space of non-abelian characters of $A_{N-1}^{(1)}$ graded by level. Then we let Ruijsenaars difference operators act after Hasegawa. The reason this also acts on the sections of the vector bundle is the existence of a connection.

\subsection{Generalized Schur-Weyl duality}

In general, we would like to construct a Schur-Weyl duality for any bundle of conformal blocks for any genus. Braid/Hecke generators are obtained as monodromies along the configuration space, which become $R$-matrices, the coordinates become one set of generators and tangent vectors give a second set of generators. For genus zero, this gives the Schur-Weyl duality between the loop Yangian and the degenerate double affine Hecke algebra, while for genus one, this gives a Schur-Weyl duality between a degenerate elliptic double affine Hecke algebra and a loop elliptic quantum group $LE(\mathfrak{gl}_\ell)$.

\section{Loop Yangian}

\subsection{As quantization of rational spin RS Poisson algebra}

\cite{article:arutyunov:1998}

The Hamiltonian reduction in the classical case is done as follows: Start with $(A,g,S) \in \mathfrak{gl}_N^* \times GL_N \times \mathfrak{gl}_N^*$ with the Poisson structure
\begin{align*}
\{ A_1,A_2 \} &= \tfrac{1}{2} [P,A_1-A_2], \quad
\{ S_1,S_2 \} = -\tfrac{1}{2} [P,S_1-S_2], \\
\{ A_1,g_2 \} &= g_2 P, \quad \{ g_1,g_2 \} = \{ S_1,g_2 \} = \{ S_1,A_2 \} = 0.
\end{align*}
Now factorize $S_{ij} = \sum_\alpha a_i^\alpha b_j^\alpha$ with $a_i^\alpha,b_i^\beta$ canonically conjugate, defining $S_{ij}^{\alpha\beta} := a_i^\alpha b_j^\beta$, where $\alpha,\beta$ can range in $1,...,\ell$. Then
\begin{equation*}
\{ S_1^{\alpha\beta},S_2^{\mu\nu} \} = P_{12} (\delta^{\beta\mu} S_2^{\alpha\nu} - \delta^{\alpha\nu} S_1^{\mu\beta})
\end{equation*}
and
\begin{equation*}
T^{\alpha\beta}(z) = \delta^{\alpha\beta} + \operatorname{tr} \frac{S^{\alpha\beta}}{z-A} = \delta^{\alpha\beta}+\sum_{n\geq 0} T_n^{\alpha\beta} z^{-n-1}, \quad T_n^{\alpha\beta} = \operatorname{tr} A^n S^{\alpha \beta}
\end{equation*}
generates the classical Yangian. Letting $J_n^{\alpha\beta} := \operatorname{tr} g^n S^{\alpha \beta}$, we define
\begin{equation*}
J^{\alpha \beta}(z) = \sum_{n=-\infty}^\infty J_n^{\alpha \beta} z^{-n-1},
\end{equation*}
which generates the classical loop algebra. On the reduced phase space, they are also given by
\begin{equation*}
J_n^{\alpha\beta} = \sum_{ij} (\mathbf{L}^{n-1})_{ij} \mathbf{a}_j^\alpha \mathbf{c}_i^\beta,
\end{equation*}
where $\mathbf{L},\mathbf{a},\mathbf{c}$ are the invariant versions of $L = TgT^{-1},a$, and $c$ ($T$ begin the diagonalizer for $A$). This makes it clear how the Lax matrix corresponds to the monodromy around a loop. Thus, it is known that
\begin{equation*}
J_1^{\alpha\beta} = \sum_i S_i^{\beta\alpha}, \quad S_i^{\alpha\beta} = \mathbf{c}_i^\alpha \mathbf{a}_i^\beta, \quad \mathbf{c}_i^\alpha = \sum_\beta S_i^{\alpha\beta}, \quad \mathbf{a}_i^\alpha = \frac{S_i^{\beta\alpha}}{\sum_\gamma S_i^{\beta\gamma}}
\end{equation*}

Recall that the loop Yangian $LY(\mathfrak{gl}_\ell)$ is Schur-Weyl dual to the degenerate double affine Hecke algebra $\ddt H_N$ and that the center of the Yangian $Y(\mathfrak{gl}_\ell)$ generated by the quantum determinant gives Hamiltonians for the quantum trigonometric spin CM model, while the center of the loop algebra $L(\mathfrak{gl}_\ell)$ gives Hamiltonians for the quantum rational spin RS model. This gives natural quantizations to $T_n^{\alpha\beta}$ and $J_n^{\alpha\beta}$. The formula for $J_1^{\alpha\beta}$ suggests the quantization rule
\begin{equation*}
S_i^{\alpha\beta} \to e_i^{\alpha\beta} \otimes X_i,
\end{equation*}
where $e_i^{\alpha\beta}$ is a matrix unit acting on the $i$th tensorand and $X_i$ is the $i$th Laurent generator of $\ddt H_N$. The following Poisson rules remain to be checked:
\begin{equation*}
\{ S_i^{\alpha\beta},S_j^{\mu\nu} \} = \frac{1}{y_i-y_j} (S_i^{\mu\beta} S_j^{\alpha\nu} + S_i^{\alpha\nu} S_j^{\mu\beta}) - \frac{\delta^{\beta\mu}}{y_i-y_j+\eta} (S_iS_j)^{\alpha\nu} + \frac{\delta^{\alpha\nu}}{y_j-y_i+\eta} (S_jS_i)^{\mu\beta} 
\end{equation*}
and $\{ y_i, S_j^{\alpha\beta} \} = S_j^{\alpha\beta} \delta_{ij}$. This can be rewritten as
\begin{equation*}
\{ S_i \otimes S_j \} = \frac{1}{y_i-y_j} (P S_i S_j + S_i S_j P) - \frac{1}{y_i-y_j+\eta} S_i P S_j - \frac{1}{y_i-y_j-\eta} S_j P S_i,
\end{equation*}
where $S_i = e^{\alpha\beta} \otimes S_i^{\alpha\beta}$ and $P$ acts on auxiliary spaces. The corresponding quantized commutation relation is
\begin{equation*}
S_i R_{12}^\hbar(y_i-y_j+\eta) S_j R_{12}^\hbar(y_j-y_i) = R_{12}^\hbar(y_i-y_j) S_j R_{12}^\hbar(y_j-y_i+\eta) S_i.
\end{equation*}
at least up to first order in $\hbar$. This should be understood as an equation in endomorphisms of $(\C^\ell)^{\otimes 2} \otimes \C[y_i,y_j]$ where $S_i$ acts on the first and $S_j$ on the second auxiliary space.

Let us find the consistency conditions for this algebra:
\begin{equation*}
S_1 A_{12} S_2 B_{12} = C_{12} S_2 D_{12} S_1
\end{equation*}
They are that $B$ and $C$ fulfill the QYBE and $A$ and $D$ are respective reps of RTT form. This is the case with the rational $R$-matrix $R^\hbar(z)$.

Let us rewrite
\begin{equation*}
R_{12}(y_i-y_j) S_j e^{-\eta \partial_i} R_{12}(y_j-y_i) e^{-\eta \partial_j} S_i e^{\eta \partial_j} e^{-\eta \partial_j} = S_i e^{-\eta \partial_j} R_{12}(y_i-y_j) e^{-\eta \partial_i} S_j e^{\eta \partial_i} e^{-\eta \partial_i} R_{12}(y_j-y_i).
\end{equation*}
This has the form of a dynamical quadratic algebra \cite{article:nagy:2004}:
\begin{equation*}
A_{12}(\lambda) S_1(\lambda) B_{12}(\lambda) S_2(\lambda+\eta h_1) = S_2(\lambda) C_{12}(\lambda) S_1(\lambda+\eta h_2) D_{12}(\lambda),
\end{equation*}
which are also important in the quantization of RS models in general, although other $A,B,C,D$ are used.

Let us try to find a solution by deforming the easy solutions at $\eta = 0$. There we have for example $S_i = \tilde S_i$ and $S_j = R_{12}^\hbar(y_i-y_j+\eta)^{-1} \tilde S_j R_{12}^\hbar(y_j-y_i+\eta)^{-1}$, where $\tilde S_i$ and $\tilde S_j$ commute. This gives for the LHS
\begin{equation*}
\tilde S_i \tilde S_j R_{12}^\hbar(y_j-y_i+\eta)^{-1} R_{12}^\hbar(y_j-y_i)
\end{equation*}
and for the RHS
\begin{equation*}
R_{12}^\hbar(y_i-y_j) R_{12}^\hbar(y_i-y_j+\eta)^{-1} \tilde S_j \tilde S_i
\end{equation*}
This does not quite match, but it would give a solution if $\tilde S_i \tilde S_j$ implements a permutation of $y_i$ and $y_j$. For example $\tilde S_i = \tilde S_j = e^{y_j \partial_i - y_i \partial_j}$. This is a solution! However, it does not satisfy canonical commutation relations, namely
\begin{align*}
[y_i,\tilde S_i] = y_j \tilde S_i, \quad [y_j,\tilde S_i] = -y_i \tilde S_i.
\end{align*}

On the other hand, if we have $S_i = \tilde S_i$ and $S_j = R_{12}^\hbar(y_i-y_j)^{-1} \tilde S_j R_{12}^\hbar(y_j-y_i)^{-1}$, then we have the LHS
\begin{equation*}
\tilde S_i R_{12}^\hbar(y_i-y_j+\eta) R_{12}^\hbar(y_i-y_j)^{-1} \tilde S_j
\end{equation*}
and the RHS is
\begin{equation*}
\tilde S_j R_{12}^\hbar(y_j-y_i)^{-1} R_{12}^\hbar(y_j-y_i+\eta) \tilde S_i
\end{equation*}
Again, if the exchange of $\tilde S_i$ and $\tilde S_j$ implement a permutation of $y_i$ and $y_j$, we are in business. This happens when $\tilde S_i = e^{-2y_i \partial_i}$ and $\tilde S_j = e^{-2y_j \partial_j}$. Let us make the reparametrization $y_i \to e^{2 \hbar y_i},\eta \to 2 \hbar \eta, \hbar \to 2 \hbar^2$. Then we have $2 y_i \partial_i \to \frac{1}{\hbar} \partial_i$ and we still have the property
\begin{align*}
\lim_{\hbar \to 0} \frac{1}{\hbar} &(S_1 R_{12}^{\hbar^2}(e^{\hbar y_i}-e^{\hbar y_j}+\hbar \eta) S_2 R_{12}^{\hbar^2}(e^{\hbar y_j}-e^{\hbar y_i})) - (\cdots)) \\
&= \frac{1}{z-w} (P_{12} S_1 S_2 + S_1 S_2 P_{12}) - \frac{1}{z-w+\eta} S_1 P_{12} S_2 + \frac{1}{w-z+\eta} S_2 P_{12} S_1
\end{align*}

This gives trigonometric-type canonical commutations relations. Is this an artifact because we have to use the trigonometric $R$-matrix? Indeed, we can also write
\begin{equation*}
S_i R_{12}^{trig,\hbar^2}(\hbar y_i-\hbar y_j+\hbar\eta) S_j R_{12}^{trig,\hbar^2}(\hbar y_j- \hbar y_i) = \cdots
\end{equation*}
and get the same classical limit and also fulfills the commutation relation with the inverse that we need. We have $2Y_i \frac{\partial}{\partial Y_i} = 2Y_i \frac{\partial Y_i}{\partial y_i}^{-1} \frac{\partial}{\partial y_i} = \frac{\partial}{\partial y_i}$ if $Y_i = e^{2y_i}$.

Let us introduce the trigonometric $R$-matrix
\begin{equation*}
\mathcal{R}^q(u) := \frac{u-q}{u-1} \sum_i e_{ii} \otimes e_{ii} + \frac{1-q}{u-1} \sum_{i \neq j} e_{ij} \otimes e_{ji} + \sum_{i \neq j} e_{ii} \otimes e_{jj}.
\end{equation*}
This fulfills
\begin{equation*}
\mathcal{R}^{t^{\hbar^2}}(t^{\hbar z}) = 1 - \frac{\hbar P}{z} + O(\hbar^2)
\end{equation*}
as desired for the classical limit. Furthermore, we indeed have
\begin{align*}
\lim_{\hbar \to 0} \frac{1}{\hbar} &( S_1 \mathcal{R}_{12}^{t^{\hbar^2}}(t^{\hbar z + \hbar \eta}, t^{\hbar w}) S_2 \mathcal{R}_{12}^{t^{\hbar^2}}(t^{\hbar w}, t^{\hbar z}) - \mathcal{R}_{12}^{t^{\hbar^2}}(t^{\hbar z}, t^{\hbar w}) S_2 \mathcal{R}_{12}^{t^{\hbar^2}}(t^{\hbar w + \hbar \eta}, t^{\hbar z}) S_1) \\
&= \frac{1}{z-w} (P_{12} S_1 S_2 + S_1 S_2 P_{12}) - \frac{1}{z-w+\eta} S_1 P_{12} S_2 + \frac{1}{w-z+\eta} S_2 P_{12} S_1.
\end{align*}
Let us again set $S_1 := \tilde S_1, S_2 := \mathcal{R}_{12}^q(ZW^{-1})^{-1} \tilde S_2 \mathcal{R}_{12}^q(WZ^{-1})^{-1}$, so we obtain the equation
\begin{equation*}
\tilde S_1 \mathcal{R}_{12}^q(ZW^{-1}T) \mathcal{R}_{12}^q(ZW^{-1})^{-1} \tilde S_2 = \tilde S_2 \mathcal{R}_{12}^q(WZ^{-1})^{-1} \mathcal{R}_{12}^q(WTZ^{-1}) \tilde S_1,
\end{equation*}
which is again fulfilled if $\tilde S_1$ and $\tilde S_2$ implements
\begin{equation*}
\tilde S_1 \tilde S_2^{-1} ZW^{-1} = WZ^{-1}.
\end{equation*}
We have $\I \coth(z+\I \pi/2) = (\I \coth(z))^{-1}$. Recall that the classical trigonometric $r$-matrix has a $\coth$-like term:
\begin{equation*}
r(t) := \frac{t+1}{t-1} P + Q, \quad Q = \sum_{\alpha\beta} \operatorname{sgn}(\beta-\alpha) e_{\alpha\beta} \otimes e_{\beta\alpha}
\end{equation*}
and it fulfills
\begin{equation*}
r(t^z) = \frac{2}{(t-1) z} P + Q + \frac{P}{z} + O(t-1).
\end{equation*}

We could also deform the $P_{2j}$ to $R_{2j}^\hbar(\hbar)$. Then the LHS becomes
\begin{equation*}
R_{1i}^\hbar(\hbar) R_{2j}^\hbar(\hbar) R_{12}^\hbar(y_i-y_j-\eta) R_{12}^\hbar(y_j-y_i) e^{\I\hbar \partial_i} e^{\I\hbar \partial_j}
\end{equation*}
while the RHS becomes
\begin{equation*}
R_{12}^\hbar(y_i-y_j) R_{12}^\hbar(y_j-y_i-\eta) R_{2j}^\hbar(\hbar) R_{1i}^\hbar(\hbar) e^{\I\hbar \partial_i} e^{\I\hbar \partial_j}.
\end{equation*}
The difference of these two terms is non-zero, but it is of order $\hbar^2,\hbar\eta$!

Let us look at the full analysis of the SRSR relation. For the $\ell=2$ case, Mathematica gives some equations:
\begin{align*}
S_i^{21} S_j^{21} (1+\frac{\hbar}{y_i-y_j}) = (1-\frac{\hbar}{y_i-y_j}) S_j^{21} S_i^{21}, \quad (1 \leftrightarrow 2)
\end{align*}
The other equations are horrendously complicated. Can we solve this first equation and substitute? The solution seems to also require a permutation of $y_i$ and $y_j$, both for $S_i^{12}$ and $S_i^{11}$.

One thing I have not tried is to have $S_i S_j$ include an antisymmetrizer $1-P_{12}$. We have
\begin{equation*}
(A P_{1i}+B P_{1j}) (C P_{2i}+D P_{2j}) = AC P_{i21} + BC P_{1j} P_{2i} + AD P_{1i} P_{2j} + BD P_{j21}
\end{equation*}

Consider the case $N=2$ and introduce the operators
\begin{align*}
p_1 &:= X_1 (y_1-y_2-\eta) - \eta X_1 s_{12} = X_1 t_{12} s_{12} (y_1-y_2) = X_1 x_{21} (y_1-y_2) \\
p_2 &:= (y_2-y_1-\eta) X_2 + \eta s_{12} X_2 = (y_1-y_2) s_{12} t_{12} X_2 = (y_1-y_2) x_{12} X_2.
\end{align*}
Then
\begin{align*}
[y_1,p_1] &= -\I\hbar p_1, \quad [y_1,p_2] = 0 \\
[y_2,p_1] &= 0, \quad [y_2,p_2] = -\I\hbar p_2, \\
[p_1,p_2] &= \I \hbar \eta^2 \frac{2 y_1 - 2 y_2 + \I\hbar}{(y_1-y_2+\I\hbar)^2} p_1 p_2.
\end{align*}
Let us define $S_i^{\alpha\beta} := e_i^{\alpha\beta} \otimes p_i$. Then
\begin{equation*}
[S_1^{\alpha\beta},S_2^{\mu\nu}] = (e_1^{\alpha\beta} \otimes X_1 x_{21} (y_1-y_2)) (e_2^{\mu\nu} \otimes (y_1-y_2)x_{12} X_2)
\end{equation*}

This comes from the general commutators
\begin{align*}
[y_i,X_j]
&= y_i t_{j-1} \cdots t_1 \pi t_{N-1} \cdots t_j - t_{j-1} \cdots t_1 \pi t_{N-1} \cdots t_j y_i \\
&= y_i t_{j-1} \cdots t_1 \pi t_{N-1} \cdots t_j - t_{j-1} \cdots t_1 \pi t_{N-1} \cdots t_i t_{i-1} y_i t_{i-2} \cdots t_j \\
&= y_i t_{j-1} \cdots t_1 \pi t_{N-1} \cdots t_j - t_{j-1} \cdots t_1 \pi t_{N-1} \cdots t_i y_{i-1} t_{i-1} t_{i-2} \cdots t_j + \eta t_{j-1} \cdots t_1 \pi t_{N-1} \cdots t_i t_{i-2} \cdots t_j \\
&= \eta t_{j-1} \cdots t_1 \pi t_{N-1} \cdots t_i t_{i-2} \cdots t_j \\
&= \eta X_j t_{ij} \\
[y_i,X_i]
&= y_i t_{i-1} \cdots t_1 \pi t_{N-1} \cdots t_i - t_{i-1} \cdots t_1 \pi t_{N-1} \cdots t_i y_i \\
&= y_i t_{i-1} \cdots t_1 \pi t_{N-1} \cdots t_i - t_{i-1} \cdots t_1 \pi t_{N-1} \cdots t_{i+1} y_{i+1} t_i - \eta t_{i-1} \cdots t_1 \pi t_{N-1} \cdots t_{i+1} \\
&= y_i t_{i-1} \cdots t_1 \pi t_{N-1} \cdots t_i - t_{i-1} \cdots t_1 \pi t_{N-1} \cdots t_{i+2} y_{i+2} t_{i+1} t_i - \eta X_i t_{i,i+2} - \eta X_i t_{i,i+1} \\
&= y_i t_{i-1} \cdots t_1 \pi t_{N-1} \cdots t_i - t_{i-1} \cdots t_1 \pi y_N t_{N-1} \cdots t_i - \eta X_i \sum_{j=i+1}^N t_{ij} \\
&= y_i t_{i-1} \cdots t_1 \pi t_{N-1} \cdots t_i - t_{i-1} \cdots t_1 y_1 \pi t_{N-1} \cdots t_i - \I \hbar X_i - \eta X_i \sum_{j=i+1}^N t_{ij} \\
&= - \I \hbar X_i - \eta \sum_{i < j} X_i t_{ij} - \eta \sum_{i > j} X_j t_{ij}
\end{align*}
This comes from the formula for the Dunkl operators:
\begin{align*}
[y_i,f(\underline{X})] = -\I\hbar X_i \partial_i f(\underline{X}) - \eta \sum_{j<i} X_j \Delta_{ij} f(\underline{X}) - \eta \sum_{i<j} X_i \Delta_{ij} f(\underline{X})
\end{align*}

The classical Hamiltonian of the rational RS model is $H = \sum_i \operatorname{Tr} S_i = \operatorname{Tr} J_1$. This formula also holds in the quantum case. The main dynamical variables are $y_i$ and $S_i^{\alpha\beta}$. To find the classical limit means showing that the dynamical variables obey the correct equations of motion, i.e. one has to check the commutation relations of the dynamical variables with the Hamiltonian.

Let us compute the Poisson bracket between $T$ and $J$:
\begin{align*}
\{ T_1(z), J_2(q) \} = \sum_{r \in \N} \sum_{n \in \Z} \operatorname{tr}_{12} \{ A_1^r S_1, g_2^n S_2 \} z^{-r-1} q^{-n-1},
\end{align*}
where
\begin{align*}
\{ A_1^r S_1, g_2^n S_2 \}
&= n g_2^{n-1} \{ A_1^r S_1, g_2 \} S_2 + g_2^n \{ A_1^r S_1, S_2 \} \\
&= n g_2^{n-1} (r A_1^{r-1} \{ A_1, g_2 \} S_1 + A_1^r \{ S_1, g_2 \}) S_2 + g_2^n (r A_1^{r-1} \{ A_1, S_2 \} S_1 + A_1^r \{ S_1, S_2 \}) \\
&= n g_2^{n-1} r A_1^{r-1} g_2 P S_1 S_2 - \tfrac{1}{2} g_2^n A_1^r [P,S_1-S_2] \\
&= P (r A_1^{r-1} S_1) (n g_2^n S_2) - \tfrac{1}{2} g_2^n A_1^r [P,S_1-S_2].
\end{align*}
We thus consider
\begin{align*}
P \sum_{r \in \N} \operatorname{tr}_1(A_1^{r-1} S_1) r z^{-r-1} \sum_{n \in \Z} \operatorname{tr}_2(g_2^n S_2) n q^{-n-1}
&= P \frac{\partial}{\partial z} T_1(z) q \frac{\partial}{\partial q} J_2(q)
\end{align*}
and
\begin{align*}
\sum_{r \in \N} \sum_{n \in \Z} \operatorname{tr}_{12} g_2^n A_1^r (S_1-S_2) z^{-r-1} q^{-n-1}
&= T_1(z) \sum_{n \in \Z} \operatorname{tr}_2 g_2^n q^{-n-1} - J_2(q) \sum_{r \in \N} \operatorname{tr}_{12} A_1^r z^{-r-1} \\
&= \delta(q/g_2) T_1(z) - (z-A_1)^{-1} J_2(q).
\end{align*}
It follows that
\begin{equation*}
\{ T_1(z), J_2(q) \} = P \frac{\partial}{\partial z} T_1(z) q \frac{\partial}{\partial q} J_2(q) - \frac{1}{2} \bigg[ \delta(q/g_2) P, T_1(z) \bigg] - \frac{1}{2} \bigg[ \frac{P}{z-A_1},J_2(q) \bigg].
\end{equation*}
How does this quantize? The derivatives probably come from $X$ and $y$ in the Macdonald/Dunkl operator representation.

\subsection{Fock space representation}

\cite{article:kodera:2016}

We construct the level one Fock space. Let $U = \C^\ell[X^{\pm 1}]$. Note that we have an isomorphism
\begin{equation*}
(\C^\ell)^{\otimes N} \otimes_{S_N} \C[X_1^{\pm 1},...,X_N^{\pm 1}] \cong \bigwedge^N U, \quad X_1^{m_1} \cdots X_N^{m_N} \otimes (e_{j_1} \otimes \cdots \otimes e_{j_N}) \mapsto e_{j_1} X^{m_1} \wedge \cdots \wedge e_{j_N} X^{m_j}.
\end{equation*}
This clearly has an action of the affine Yangian by Schur-Weyl duality. Note that we recover the trigonometric Calogero-Moser system. The Fock space is obtained by the inverse limit over $N$ respecting a certain grading.

We clearly have an $R$-matrix $R(y_1-y_2) \in \End(U^{\otimes 2})$ acting via matrix-differential operators. This should generalize to an $R$-matrix in $\End((\bigwedge^N U)^{\otimes 2})$. If the affine Yangian acts faithfully, we might obtain an RTT presentation this way.

\subsection{Conjectural presentation from 4d CS}

We know that the representations of the Yangian control Wilson lines in 4d CS theory. 't Hooft lines on the other hand modify the underlying principal bundle, for example by introducing a twist $g$ (a conjugacy class of $\mathfrak{gl}_\ell$).

We should have that $S^1 \times *$ gets mapped to the 2-category $\mathcal{B}Y(\mathfrak{gl}_\ell)\mathsf{Mod}$, while $* \times S^1$ gets mapped to the 2-category $\mathcal{B}U(\dt{\mathfrak{gl}}_\ell)\mathsf{Mod}$ and $S^1 \times S^1$ gets mapped to $Y(\dt{\mathfrak{gl}}_\ell)\mathsf{Mod}$. Let $A$ be the annulus. Then $A \times S^1$ should give the $Y(\dt{\mathfrak{gl}}_\ell)$-module $\C^\ell[y]$. On the other hand, $S^1 \times A$ should give the $Y(\dt{\mathfrak{gl}}_\ell)$-module $\C^\ell[X^{\pm 1}]$. From this, we have to be able to derive relations for the affine Yangian in the spirit of \cite{article:costello:2018b}.

Let us encode the lowest level bordisms (unions of $S^1 \times *$ and $* \times S^1$) as pairs $[N+M]$ of natural numbers. 1-morphisms $[N+M] \to [N'+M]$ is combinatorially represented as $(\Gamma,\sigma)$ with a composition $\Gamma: N \to N'$ of $N$ by $N'$ numbers with a permutation $\sigma \in S_M$ (a union of $N'$ orderedly punctured discs times the permutation bordism). This however disregards closed 1-morphisms of the form $S^1 \times S^1$. Then 2-morphisms are mapping classes of surfaces, which includes permutation of punctures and Dehn twists. The field theory gives a map
\begin{equation*}
\End((S^1 \times S^1)^{\sqcup N}) \to S \ddt H_N.
\end{equation*}
This tells us that the elements of the affine Yangian do not come from the plain field theory itself but from operators (natural transformations) on the field theory.

A Wilson line with incoming state $e_\alpha$ and outgoing state $e_\beta$ on $A \times S^1 \times [0,1]$ along $[0,1]$ gives an operator $T_{\alpha\beta}(z) = \sum_{r \in \N} T_{ij}^{(n)} z^{-r}: \C^\ell[y] \to \C^\ell[y]$ determined by the $R$-matrix. Similarly, an 't Hooft line on $[0,1] \times S^1 \times A$ gives an operator $J_{\alpha\beta}(q) = \sum_{n \in \Z} J_{ij}^{(n)} q^{-n-1}: \C^\ell[X^{\pm 1}] \to \C^\ell[X^{\pm 1}]$ determined by
\begin{equation*}
J_{\alpha\beta}^{(n)}: e_\alpha \otimes X^k \mapsto -e_\beta \otimes X^{k+n}.
\end{equation*}
These could be Fourier modes of the flux through the 't Hooft line. This implies
\begin{equation*}
J(q) \mapsto -\sum_{\alpha\beta,n} e_{\alpha\beta} \otimes e_{\beta\alpha} X^n q^{-n-1} = -\delta(q/X) P =: -\dt r(q/X), \quad \delta(q/X) := q^{-1} \sum_{n \in \N} (q/X)^n
\end{equation*}
for the vector representation $\C^\ell[X^{\pm 1}]$. Both are operators on $A \times A$, so both act on $\C^\ell[y]$ and $\C^\ell[X^{\pm 1}]$ and satisfy non-trivial relations. These should give the affine Yangian relations when the number of lines $N \to \infty$. The mapping classes relate this to Dehn twists. The interesting stuff happens when we cross a Wilson line with an 't Hooft line. In the vector representation $\C^\ell[y]$, this should have the effect of a twist $g$ and a shift $e^{\hbar \partial}$.

All of this can be summarized in a graphical calculus on the cylinder $S^1 \times [0,1]$ with two types of lines in the axial direction: Wilson and 't Hooft lines. Wilson lines are labeled by a complex coordinate $y$ and 't Hooft lines are labeled by a mode number $n$ or nome $q$. We can act on the diagrams via mapping classes. This way, we can generate the affine Yangian by taking a triangle of lines and letting one of them take values in the algebra, the others in the vector representation.

The loop algebra relation is
\begin{align*}
[J_1(z),J_2(w)] = [\dt r_{12}(z/w),J_2(w)] = [\dt r_{12}(w/z),-J_1(z)].
\end{align*}
Indeed, in the vector representation, we have
\begin{align*}
[-\dt r_{13}(p/X),-\dt r_{23}(q/X)] = [\dt r_{12}(p/q), -\dt r_{23}(q/X)] = [\dt r_{12}(q/p), \dt r_{13}(p/X)],
\end{align*}
where we have made use of the identity
\begin{equation*}
\delta(p/X) \delta(q/X) = \sum_{nm} p^{-n-1} q^{-m-1} X^{-n-m} = \sum_{nk} p^{-n-1} q^{k-n} X^{-k-1} = \delta(p/q) \delta(q/X).
\end{equation*}
Also note that
\begin{equation*}
\delta(p/q) = \sum_n p^{-n-1} q^n = \sum_k p^k q^{-k-1} = \delta(q/p).
\end{equation*}
Let
\begin{equation*}
\dt R(q) := 1 - \hbar \dt r(q).
\end{equation*}
The difference between the two sides of the QYBE is then $\hbar^2 \delta(p) \delta(q) (P_{123} - P_{312})$.

Let us now introduce the crossing matrices
\begin{equation*}
Q(X,y) \in \End \C^\ell \otimes \C[X^{\pm 1}] \otimes \End \C^\ell \otimes \C[y] \subset \End(\C^\ell[X^{\pm 1}] \otimes \C^\ell[y])
\end{equation*}
The two types of 3rd Reidemeister moves involving both Wilson and 't Hooft lines tell us that $Q$ gives rise to representations of $Y(\mathfrak{gl}_\ell)$ and $\dt{\mathfrak{gl}}_\ell$ on $\C[X^{\pm 1}]$ and $\C[y]$, respectively:
\begin{align*}
R_{12}(z-w) Q_{13}(X,z) Q_{23}(X,w) &= Q_{23}(X,w) Q_{13}(X,z) R_{12}(z-w), \\
[Q_{31}(p,y),Q_{32}(q,y)] &= [r_{12}(p-q),Q_{32}(q,y)]
\end{align*}
We know how to construct such representations using Schur-Weyl duality:
\begin{equation*}
Q_{12}(X,y) = r_{12}(X-D_1) R_{12}(y-d_2) \in \End(\overset{1}{\C^\ell[y]} \otimes \overset{2}{\C^\ell[X^{\pm 1}]}),
\end{equation*}
but there is still a problem with ordering.
%We consider the following relation in $\End(\C^\ell \llbracket z^{-1} \rrbracket \otimes \C^\ell \llbracket w^{-1} \rrbracket \otimes \C^\ell \llbraces X \rrbraces)$:
%\begin{align*}
%&\sum_{r,s=0}^\infty (z-w-\eta P_{12}) D_1^r X^{-r-1} (1-\frac{\eta P_{13}}{z-d_3}) D_2^s X^{-s-1} (1-\frac{\eta P_ {23}}{w-d_3}) \\
%&= \sum_{r,s=0}^\infty (z-w-\eta P_{12}) D_1^r D_2^s X^{-r-s-2} \\
%&- \eta \sum_{r,s,m=0}^\infty (z-w-\eta P_{12}) (P_{13} D_1^r X^{-r-1} d_3^m D_2^s z^{-m-1} X^{-s-1} + P_{23} D_1^r D_2^s X^{-r-s-2} d_3^m w^{-m-1}) \\
%&+ \eta^2 \sum_{r,s,m,n=0}^\infty (z-w-\eta P_{12}) P_{13} P_{23} D_1^r X^{-r-1} d_3^m D_2^s X^{-s-1} d_3^n z^{-m-1} w^{-n-1}
%\end{align*}

The representation $\C^\ell[y] \otimes \C^\ell[X^{\pm 1}]$ by Schur-Weyl duality comes from the representation $\C[y] \boxtimes \C[X^{\pm 1}]$ defined according to
\begin{equation*}
U \boxtimes V := \ddot{H}_{N+M} \otimes_{\ddot{H}_N \otimes \ddot{H}_M} (U \otimes V),
\end{equation*}
so we have to look at $\ddot{H}_2 \otimes_{\ddot{H}_1 \otimes \ddot{H}_1} (\C[y_1] \otimes \C[X_2^{\pm 1}])$. Let us consider $(\C^\ell)^{\otimes N} \otimes_{S_N} \ddot{H}_N$. Then
\begin{align*}
&P_{12} v \otimes X_2^s y_1^r s_{12}
\equiv v \otimes X_1^s \left( \frac{y_1-y_2-\eta}{y_1-y_2} y_2^r + \frac{\eta}{y_1-y_2} y_1^r \right) \\
\Leftrightarrow &P_{12} v \otimes X_2^s y_1^r s_{12} - v \otimes X_1^s \frac{\eta}{y_1-y_2} y_1^r \equiv v \otimes X_1^s \frac{y_1-y_2-\eta}{y_1-y_2} y_2^r \\
\Leftrightarrow &P_{12} v \otimes \frac{y_1-y_2}{y_1-y_2-\eta} (X_2/X_1)^s y_1^r s_{12} - v \otimes \frac{\eta}{y_1-y_2-\eta} y_1^r \equiv v \otimes y_2^r
\end{align*}
Setting $r=0$ would give the crossing matrices
\begin{equation*}
Q^s := \frac{y_1-y_2}{y_1-y_2-\eta} \left( \frac{X_1}{X_2} \right)^s s_{12} - \frac{\eta}{y_1-y_2-\eta} P_{12}
\end{equation*}
\begin{align*}
\frac{y_1-y_2}{y_1-y_2-\eta} s_{12} f(y_1) g(X_2) s_{12}^{-1} - \frac{\eta}{y_1-y_2-\eta} f(y_1) g(X_1) = f(y_2) g(X_1)
\end{align*}
\begin{align*}
[Q_{31},Q_{32}] = [\frac{P_{12}}{X_1-X_2},Q_{31}+Q_{32}]
\end{align*}
\begin{equation*}
[Q_{31},Q_{32}] = \frac{\eta}{y_3-y_1-\eta} \frac{\eta}{y_3-y_1-\eta} (P_{321}-P_{123}) = ...
\end{equation*}

These equations have the following non-diagonal solutions that depend on both $q$ and $z$, if we only consider the action on $(\C^\ell)^{\otimes 2}$:
\begin{equation*}
Q^\pm(q,z) = (\alpha + \beta q)(\gamma + \delta z) + \sum_{i \lessgtr j} (\gamma+\delta z) e_{ij} \otimes e_{ji}
\end{equation*}
One solution to the first equation is also given by
\begin{equation*}
Q(q,z) := z - \eta P^q, \quad P^q := \sum_{ij} q^{\operatorname{sgn}(i-j)} e_{ij} \otimes e_{ji}.
\end{equation*}
This also solves the two-parameter QYBE
\begin{equation*}
Q_{12}(p/q,z-w) Q_{13}(p,z) Q_{23}(q,w) = Q_{23}(q,w) Q_{13}(p,z) Q_{12}(p/q,z-w),
\end{equation*}
which specializes to
\begin{align*}
R_{12}(z-w) Q_{13}(q,z) Q_{23}(q,w) &= Q_{23}(q,w) Q_{13}(q,z) R_{12}(z-w), \\
P_{12}^{p/q} Q_{13}(p,z) Q_{23}(q,z) &= Q_{23}(q,z) Q_{13}(p,z) P_{12}^{p/q}.
\end{align*}
The second relation expands to
\begin{equation*}
P_{12}^{p/q} (z^2 - z \eta (P_{13}^p + P_{23}^q) + \eta^2 P_{13}^p P_{23}^q) = (z^2 - z \eta (P_{13}^p + P_{23}^q) + \eta^2 P_{23}^q P_{13}^p) P_{12}^{p/q},
\end{equation*}
which gives
\begin{align*}
[P_{12}^{p/q},P_{13}^p + P_{23}^q] = 0, \\
P_{12}^{p/q} P_{13}^p P_{23}^q = P_{23}^q P_{13}^p P_{12}^{p/q}.
\end{align*}
Let us consider the relation
\begin{equation*}
Q_{12}(p/q,0) S_1(p) S_2(q) = S_2(q) S_1(p) Q_{12}(p/q,0),
\end{equation*}
which is equivalent to
\begin{equation*}
(p/q)^{\operatorname{sgn}(i-j)} s_{jk}(p) s_{il}(q) = (p/q)^{\operatorname{sgn}(l-k)} s_{jk}(q) s_{il}(p)
\end{equation*}

This then generates the relations between $T$ and $J$ in a QTJ relation. A QTJ relation should hold in
\begin{equation*}
\End \C^\ell \otimes \C[X^{\pm 1}] \otimes \End \C^\ell \otimes \C[y] \otimes Y(\dt{\mathfrak{gl}}_\ell)\llbracket z \rrbracket \llbraces q \rrbraces.
\end{equation*}
We can build affine transfer matrices via
\begin{equation*}
\tau(z;X,y) := \operatorname{tr}_1 Q_{12}(X,y) T_1(z-y) \in \End \C^\ell \otimes \C[X^{\pm 1}] \otimes \C[y] \otimes Y(\dt{\mathfrak{gl}}_\ell)\llbracket z^{-1} \rrbracket.
\end{equation*}
The coefficient in the $X$-expansion at order $-1$ after setting $y = 0$ gives
\begin{equation*}
\operatorname{tr}_1 P_{12} T_1(z) \in \End \C^\ell \otimes Y(\dt{\mathfrak{gl}}_\ell)\llbracket z^{-1} \rrbracket
\end{equation*}
Then then $g$-twisted trace on the index 2 is nothing but the $g$-twisted transfer matrix.

We have
\begin{align*}
t_{321}-t_{123} = [t_{13},t_{23}]
&= [s_{12}(X_1-X_2)^{-1},t_{13}+t_{23}] \\
&= [s_{12}(X_1-X_2)^{-1},t_{13}]+[s_{12}(X_1-X_2)^{-1},t_{23}] \\
&= t_{13} (s_{32} (X_3-X_2)^{-1} - s_{12}(X_1-X_2)^{-1}) + t_{23} (s_{13}(X_1-X_3)^{-1} - s_{12}(X_1-X_2)^{-1})
\end{align*}

\subsection{Splitting the loop algebra}

Looking at the RTT presentation of the quantum loop algebra, it might be advisable to split the current $J(z)$ into currents $J^+(z) = \sum_{n \geq 0} J_n z^{-n-1}$ and $J^-(z) = \sum_{n \leq 0} J_n z^{-n-1}$. This is needed because the trigonometric $R$-matrix decouples in both limits $z \to \infty$ and $z \to 0$, see the discussion in \cite{article:costello:2018b}. Then we can use the rational $r$-matrix:
\begin{equation*}
[J_1^+(z),J_2^+(w)] = [r_{12}(z-w),J_1^+(z)+J_2^+(w)],
\end{equation*}
which comes from the $\eta^2$ term of
\begin{equation*}
R_{12}(z-w) S_1^+(z) S_2^+(w) = S_2^+(w) S_1^+(z) R_{12}(z-w), \quad S^+(z) := 1 + \eta J^+(z).
\end{equation*}
Generally, we should view this as a degeneration of the trigonometric case, meaning we have a trigonometric $R$-matrix $\dt R(q)$ satisfying the QYBE
\begin{equation*}
\dt R_{12}(u/v) \dt R_{13}(u) \dt R_{23}(v) = \dt R_{23}(v) \dt R_{13}(u) \dt R_{12}(u/v)
\end{equation*}
and giving a presentation
\begin{equation*}
\dt R_{12}(u/v) S_1^\pm(u) S_2^\pm(v) = S_2^\pm(v) S_1^\pm(u) \dt R_{12}(u/v), \quad \dt R_{12}(u/v) S_1^-(u) S_2^+(v) = S_2^+(v) S_1^-(u) \dt R_{12}(u/v)
\end{equation*}
of the quantum loop algebra. The quantum loop algebra reduces to the loop algebra in the limit $q \to 1$ with the generators renormalized as
\begin{align*}
\sigma_{ij}^{(r)\pm} := \frac{s_{ij}^{(r)\pm}}{q-q^{-1}}, \quad \sigma_{ii}^{(0)\pm} = \frac{s_{ii}^{(0)\pm}-1}{q-1}.
\end{align*}
We have used the trigonometric $R$-matrix
\begin{equation*}
\dt R(u,v,q) = \sum_{ij} (u q^{-\delta_{ij}} - v q^{\delta_{ij}}) e_{ii} \otimes e_{jj} - (q-q^{-1}) \bigg( u \sum_{i > j} e_{ij} \otimes e_{ji} - v \sum_{i < j} e_{ij} \otimes e_{ji} \bigg),
\end{equation*}
%and the part that survives the limit $q \to 1$ should be the quadratic part in the expansion, which is
%\begin{equation*}
%u \sum_i e_{ii} \otimes e_{ii} + \bigg( u \sum_{i > j} e_{ij} \otimes e_{ji} - v \sum_{i < j} e_{ij} \otimes e_{ji} \bigg).
%\end{equation*}
How does this give the loop algebra? According to Connor Patrick's master's thesis, the $q \to 1$ limit will be
\begin{align*}
[\sigma_{ij}^{(r \pm 1)\pm},\sigma_{kl}^{(s)\pm}] - [\sigma_{ij}^{(r)\pm},\sigma_{kl}^{(s \pm 1)\pm}] &= 0 \\
[\sigma_{ij}^{(r-1)-},\sigma_{kl}^{(s)+}] - [\sigma_{ij}^{(r)-},\sigma_{kl}^{(s+1)+}] &= 0
\end{align*}
These relations are fulfilled in the loop algebra, but how do they give the defining relations? Recall that $\sigma_{ij}^{(0)\pm}$ are lower and upper triangular with inverse eigenvalues. Hence, take the $-+$ relation and set $r=1, s=0$. Then
\begin{equation*}
[\sigma_{ij}^{(0)-},\sigma_{kl}^{(0)+}] - [\sigma_{ij}^{(1)-},\sigma_{kl}^{(1)+}] = 0
\end{equation*}
where the first term is non-zero only if $i \geq j$ and $k \leq l$ and also zero when $i=j=k=l$.

Let $P^u := \sum_{ij} u^{\operatorname{sgn}(i-j)} e_{ij} \otimes e_{ji}$. Another normalization of the trigonometric $R$-matrix is
\begin{equation*}
\dt R(u,q) := (u-u^{-1}) \sum_{i \neq j} e_{ii} \otimes e_{jj} + (uq^{-1}-u^{-1}q-q^{-1}+q) \sum_i e_{ii} \otimes e_{ii} + (q^{-1}-q) P^u
\end{equation*}
The normalization factor is
\begin{equation*}
h(u,q) = \frac{1}{qu^{-1}-q^{-1}u}.
\end{equation*}
%The presentation then becomes
%\begin{align*}
%&(u-u^{-1}) \sum_{ln,i \neq j} e_{il} \otimes e_{jn} \otimes s_{il}(p) s_{jn}(q)
%+ (uq^{-1}-u^{-1}q-q^{-1}+q) \sum_{iln} e_{il} \otimes e_{in} \otimes s_{il}(p) s_{in}(q) \\
%&+ (q^{-1}-q) \sum_{ijln} u^{\operatorname{sgn}(i-j)} e_{il} \otimes e_{jn} \otimes s_{jl}(p) s_{nn}(q) \\
%&= (u-u^{-1}) \sum_{km,i \neq j} e_{ki} \otimes e_{mj} \otimes s_{mj}(q) s_{ki}(p)
%+ (uq^{-1}-u^{-1}q-q^{-1}+q) \sum_{ikm} e_{ki} \otimes e_{mi} \otimes s_{mi}(q) s_{ki}(p) \\
%&+ (q^{-1}-q) \sum_{ijkm} u^{\operatorname{sgn}(i-j)} e_{kj} \otimes e_{mi} \otimes s_{mj}(q) s_{ki}(p)
%\end{align*}
Let
\begin{equation*}
Q^X = X \sum_{i < j} e_{ij} \otimes e_{ji} + \frac{X+X^{-1}}{2} \sum_i e_{ii} \otimes e_{ii} + X^{-1} \sum_{i > j} e_{ij} \otimes e_{ji}
\end{equation*}
Then
\begin{equation*}
Q(X,y,\eta) := \lim_{t \to 1} \frac{\dt R(t^{y/2}/X,t^{\eta/2})}{t-1} = \tfrac{X+X^{-1}}{2} y - \eta Q^X
\end{equation*}

The loop Yangian should have the presentation
\begin{align*}
R_{12}(z-w) T_1(z) T_2(w) &= T_2(w) T_1(z) R_{12}(z-w) \\
C_{12}(u/v) J_1^\pm(u) J_2^\pm(v) &= J_2^\pm(v) J_1^\pm(u) C_{12}(u/v) \\
C_{12}(u/v) J_1^-(u) J_2^+(v) &= J_2^+(v) J_1^-(u) C_{12}(u/v) \\
Q_{12}(u,z) T_1(z) J_2^\pm(u) &= J_2^\pm(u) T_1(z) Q_{12}(u,z) \\
J_{ii}^+[0] J_{ii}^-[0] &= 1.
\end{align*}
To compare with the Drinfeld presentation, \cite{article:ding:1993} looks at the unique Gauss decompositions
\begin{equation*}
J^\pm(u) = E^\pm(u) H^\pm(u) F^\pm(u)
\end{equation*}
with $E^\pm(u)$ lower unipotent, $H^\pm(u)$ diagonal, and $F^\pm(u)$ upper unipotent. Define
\begin{equation*}
X_i^+(z) := f_{i,i+1}^+(z)-f_{i,i+1}^-(z), \quad X_i^-(z) := e_{i+1,i}^+(z) - e_{i+1,i}^-(z),
\end{equation*}
They prove the isomorphism by starting with $\ell=2$ and $\ell=3$ and proceeding by induction.

\subsection{Affine Yangian relations}

\cite{article:kodera:2015}

The Yangian for $\mathfrak{sl}_\ell$ in the Cartan presentation has generators $H_{i,r}, X_{i,r}^\pm, i \in \{ 1,...,\ell-1 \}, r \geq 0$ subject to the relations
\begin{align*}
[H_{i,r},H_{j,s}] = 0, \quad [H_{i,0},X_{j,s}^\pm] &= \pm c_{ij} X_{j,s}^\pm, \quad [X_{i,r}^+,X_{j,s}^-] = \delta_{ij} H_{i,r+s} \\
[H_{i,r+1},X_{j,s}^\pm] - [H_{i,r},X_{j,s+1}^\pm] &= \pm \alpha c_{ij} \{ H_{i,r}, X_{j,s}^\pm \} \\
[X_{i,r+1}^\pm,X_{j,s}^\pm] - [X_{i,r}^\pm,X_{j,s+1}^\pm] &= \pm \alpha c_{ij} \{ X_{i,r}^\pm, X_{j,s}^\pm \}
\end{align*}
as well as the Serre relations
\begin{equation*}
\sum_{\sigma \in S_m} \operatorname{ad}(X_{i,r_{\sigma(1)}}^\pm) \cdots \operatorname{ad}(X_{i,r_{\sigma(m)}}^\pm) X_{j,s}^\pm = 0, \quad m = 1-c_{ij}.
\end{equation*}
We can rewrite the first relations as coefficients of
\begin{align*}
[H_i(z),H_j(w)] = 0, \quad [H_i(0),X_j^\pm(z)] &= \pm c_{ij} X_j^\pm(z), \quad [X_i^+(z),X_j^-(w)] = \delta_{ij} \delta(z/w) H_j(w) \\
(z-w)[H_i(z),X_j^\pm(w)] &= \pm \alpha c_{ij} \{ H_i(z),X_j^\pm(w) \} \\
(z-w)[X_i^\pm(z),X_j^\pm(w)] &= \pm \alpha c_{ij} \{ X_i(z)^\pm,X_j^\pm(w) \}
\end{align*}
The affine Yangian is obtained by adding $i=0$ and modifying the cases $(i,j) = (1,0), (0,\ell-1)$ to be
\begin{align*}
[H_{j,r+1},X_{i,s}^\pm] - [H_{j,r},X_{i,s+1}^\pm] &= (\beta-\alpha \mp \alpha) X_{i,s}^\pm H_{j,r} + (\alpha \mp \alpha - \beta) H_{j,r} X_{i,s}^\pm \\
[H_{i,r+1},X_{j,s}^\pm] - [H_{i,r},X_{j,s+1}^\pm] &= (\beta-\alpha \mp \alpha) H_{i,r} X_{j,s}^\pm + (\alpha \mp \alpha - \beta) X_{j,s}^\pm H_{i,r} \\
[X_{i,r+1}^\pm,X_{j,s}^\pm] - [X_{i,r}^\pm,X_{j,s+1}^\pm] &= (\beta-\alpha \mp \alpha) X_{i,r}^\pm X_{j,s}^\pm + (\alpha \mp \alpha - \beta) X_{j,s}^\pm X_{i,r}^\pm,
\end{align*}
which in series notation becomes
\begin{align*}
((z-\alpha)-(w-\beta))[H_j(z),X_i^\pm(w)] &= \mp \alpha \{ H_j(z),X_i^\pm(w) \} \\
((z+\alpha)-(w+\beta))[H_i(z),X_j^\pm(w)] &= \mp \alpha \{ H_i(z),X_j^\pm(w) \} \\
((z+\alpha)-(w+\beta))[X_i^\pm(z),X_j^\pm(w)] &= \mp \alpha \{ X_i^\pm(z),X_j^\pm(w) \}.
\end{align*}
or
\begin{align*}
(z-w)[H_j(z-\beta),X_i^\pm(w-\alpha)] &= \mp \alpha \{ H_j(z-\beta),X_i^\pm(w-\alpha) \} \\
(z-w)[H_i(z-\alpha),X_j^\pm(w-\beta)] &= \mp \alpha \{ H_i(z-\alpha),X_j^\pm(w-\beta) \} \\
(z-w)[X_i^\pm(z-\alpha),X_j^\pm(w-\beta)] &= \mp \alpha \{ X_i^\pm(z-\alpha),X_j^\pm(w-\beta) \}.
\end{align*}
These comes from the automorphism $H_i(z),X_i^\pm(z) \mapsto H_{i-1}(z-\alpha),X_{i-1}^\pm(z-\alpha)$ for $i \neq 0,1$ and $H_i(z),X_i^\pm(z) \mapsto H_{i-1}(z-\beta),X_{i-1}^\pm(z-\beta)$ otherwise. This automorphism rotates the Dynkin diagram and can be used to extend Yangian actions to the affine Yangian.

These generators of the Yangian relate to the $T$-matrix in the following way:
\begin{align*}
X_i^+(z) := \eta \sum_{r \geq 0} X_{i,r}^+ z^{-r-1} = C_i(z+\tfrac{\hbar}{2}(i-1)) A_i(z+\tfrac{\hbar}{2}(i-1))^{-1} \\
X_i^-(z) := \eta \sum_{r \geq 0} X_{i,r}^- z^{-r-1} = A_i(z+\tfrac{\hbar}{2}(i-1))^{-1} B_i(z+\tfrac{\hbar}{2}(i-1)) \\
H_i(z) := 1 + \eta \sum_{r \geq 0} H_{i,r} z^{-r-1} = \frac{A_{i+1}(z+\tfrac{\hbar}{2}(i+1)) A_{i-1}(z+\tfrac{\hbar}{2}(i-1))}{A_i(z+\tfrac{\hbar}{2}(i+1)) A_i(z+\tfrac{\hbar}{2}(i-1))},
\end{align*}
where $A_i(z) = T_{1 \cdots i}^{1 \cdots i}(z), B_i(z) = T_{1\cdots i}^{1 \cdots i-1,i+1}(z)$ and $C_i(z) = T^{1\cdots i}_{1 \cdots i-1,i+1}(z)$, where we have used the quantum minors
\begin{equation*}
T_{i_1 \cdots i_k}^{j_1 \cdots j_k}(z) = \sum_{\sigma \in S_k} \operatorname{sgn} \sigma \cdot t_{i_{\sigma(1)},j_1}(z) \cdots t_{i_{\sigma(k)},j_k}(z-\eta(k-1))
\end{equation*}
These are the matrix coefficients of $k! T_{[1^k]}(z)$. This has a diagrammatic interpretation! We also have the Gauss decomposition $T(z) = X^-(z) H(z) X^+(z)$, which is closely related. Also see remark 3.1.8 in Molev. So we should consider the homomorphism
\begin{equation*}
\Phi: H_i(z) \mapsto h_i(z), X_i^+(z) \mapsto f_{i,i+1}(z), X_i^-(z) \mapsto e_{i+1,i}(z),
\end{equation*}
but how are $H_0(z),X_0^\pm(z)$ mapped? We have $H_0(0) = (e_{11}-e_{\ell\ell}) \otimes 1, X_0^-(0) = e_{1\ell} \otimes t^{-1}, X_0^+(0) = e_{\ell 1} \otimes t$

It is clear that the most non-trivial generators are $H_{0,1},X_{0,1}^\pm$. They are computed as follows:
\begin{align*}
X_{0,1}^+ &= -\frac{1}{2} [H_{1,1}+H_{\ell-1,1},X_0^+]-\frac{1}{2}(((\lambda-\beta)H_1+\beta H_{\ell-1})X_0^+ + X_0^+(\beta H_1+(\lambda-\beta)H_{\ell-1})),
\end{align*}
$X_{0,1}^-$ acts on $(\C^\ell)^{\otimes N} \otimes_{S_N} U$ as
\begin{equation*}
X_{0,1}^-(v \otimes u) = \sum_{j=1}^\ell (e_{1\ell})_j v \otimes y_i u.
\end{equation*}
We also have $H_{0,1} = [X_{0,0}^+,X_{0,1}^-]$, where $X_{0,0}^+ \mapsto \sum_{j=1}^\ell (e_{\ell 1})_j \otimes X_j$, so
\begin{align*}
H_{0,1}
&= \sum_{j \neq k} (e_{\ell 1})_k (e_{1 \ell})_j \otimes [y_j,X_k] + \sum_j ((e_{\ell\ell})_j \otimes y_j X_j - (e_{11})_j \otimes X_j y_j) \\
&= -\eta \sum_{j \neq k} (e_{\ell 1})_k (e_{1 \ell})_j \otimes s_{jk} + \sum_j ((e_{\ell\ell})_j \otimes y_j X_j - (e_{11})_j \otimes X_j y_j) \\
&= -\eta \sum_{j \neq k} (e_{\ell\ell})_k (e_{11})_j \otimes 1 + \sum_j ((e_{\ell\ell})_j \otimes y_j X_j - (e_{11})_j \otimes X_j y_j)
\end{align*}
The general pattern seems to be the following: $H_{0,0}$ only has pure Lie algebra terms. $H_{0,1}$ has pure quadratic Lie algebra terms plus terms that are linear in the Lie algebra and have order 1 and 1 in the Laurent and polynomial generators. $H_{0,2}$ has pure cubic Lie algebra terms plus terms that are quadratic in the Lie algebra with order 1 and 1 in the Laurent and polynomial generators plus terms that are linear in the Lie algebra with order 2 and 2 in the Laurent and polynomial generators.

What does $X_0^\pm(z)$ and $H_0(z)$ look like? We have $X_{0,0}^+ = e_{\ell 1} \otimes t$ and $X_{0,0}^- = e_{1 \ell} \otimes t^{-1}$ and $H_{0,0} = (e_{11}-e_{\ell\ell}) \otimes 1$. We then have $X_{0,1}^-$ and $H_{0,1} = [X_{0,0}^+,X_{0,1}^-]$ as well as
\begin{equation*}
X_{0,r+1}^\pm = \pm \frac{1}{2} [H_{0,1},X_{0,r}^\pm] - \frac{1}{2} \{ H_{0,0}, X_{0,r}^\pm \}, \quad H_{0,r+1} = [X_{0,r}^+,X_{0,1}^-].
\end{equation*}
Let us determine $X_{0,1}^+$:
\begin{align*}
2 X_{0,1}^+
&= [H_{0,1},X_{0,0}^+]-\{ H_{0,0},X_{0,0}^+ \} \\
&= \sum_{j \neq k} \sum_{i \neq j,k} (e_{\ell 1})_k (e_{1 \ell})_j (e_{\ell 1})_i \otimes [[y_j,X_k],X_i] \\
&+ \sum_{j \neq k} ((e_{\ell 1})_k (e_{11})_j \otimes [y_j,X_k] X_j - (e_{\ell 1})_k (e_{\ell\ell})_j \otimes X_j [y_j,X_k]) \\
&+ \sum_{i \neq j}((e_{\ell 1})_i (e_{\ell\ell})_j \otimes [y_j X_j,X_i] - (e_{\ell 1})_i (e_{11})_j \otimes [X_j y_j,X_i]) \\
&+ \sum_i (e_{\ell 1})_i \otimes y_i X_i^2 + (e_{\ell 1})_i \otimes X_i^2 y_i \\
&- 2 \sum_{i \neq j} ((e_{\ell\ell})_j (e_{\ell 1})_i \otimes X_i - (e_{11})_j (e_{\ell 1})_i \otimes X_i \\
&= \sum_{i \neq j \neq k} (e_{1 \ell})_i (e_{\ell 1})_j (e_{\ell 1})_k \otimes [[y_i,X_j],X_k] \\
&+ \sum_{i \neq j} (e_{\ell 1})_i (e_{11})_j \otimes ([y_j,X_i] X_j - [X_j y_j,X_i] + 2X_j) \\
&+ \sum_{i \neq j} (e_{\ell 1})_i (e_{\ell\ell})_j \otimes (-X_j[y_j,X_i]+[y_j X_j,X_i] - 2X_j) \\
&+ \sum_i (e_{\ell 1})_i \otimes (y_i X_i^2 + X_i^2 y_i) \\
&- 2 \sum_{i \neq j} ((e_{\ell\ell})_j (e_{\ell 1})_i \otimes X_i - (e_{11})_j (e_{\ell 1})_i \otimes X_i) \\
&= \sum_{i \neq j} (e_{\ell 1})_i (e_{11})_j \otimes (-\eta s_{ij} X_j + \eta s_{ij} X_i + 2X_j) \\
&+ \sum_{i \neq j} (e_{\ell 1})_i (e_{\ell\ell})_j \otimes (\eta s_{ij} X_i - \eta s_{ij} X_j - 2X_j) \\
&+ \sum_i (e_{\ell 1})_i \otimes (y_i X_i^2 + X_i^2 y_i) \\
&= \sum_{i \neq j} (e_{\ell 1})_i (e_{11} - e_{\ell\ell})_j \otimes (\eta X_i - \eta X_j + 2X_j) \\
&+ \sum_i (e_{\ell 1})_i \otimes (y_i X_i^2 + X_i^2 y_i)
\end{align*}
From this, we determine $H_{0,2}$:
\begin{align*}
2H_{0,2}
&= [2X_{0,1}^+,X_{0,1}^-] \\
&= \sum_k \sum_{i \neq j} [(e_{\ell 1})_i (e_{11}-e_{\ell\ell})_j \otimes (\eta X_i + (2-\eta) X_j), (e_{1\ell})_k \otimes y_k] \\
&+ \sum_k \sum_i [(e_{\ell 1})_i \otimes (y_i X_i^2+X_i^2 y_i),(e_{1\ell})_k \otimes y_k] \\
&= \sum_{i \neq j} \sum_{k \neq i,j}(e_{\ell 1})_i (e_{11}-e_{\ell\ell})_j (e_{1\ell})_k \otimes (\eta^2 s_{ik} + (2-\eta) \eta s_{jk}) \\
&+ \sum_{i \neq j} ((e_{\ell\ell})_i (e_{11}-e_{\ell\ell})_j \otimes (\eta X_i + (2-\eta) X_j) y_i - (e_{11})_i (e_{11}-e_{\ell\ell})_j \otimes y_i (\eta X_i + (2-\eta) X_j)) \\
&+ \sum_{i \neq j} (e_{\ell 1})_i (e_{1\ell})_j \otimes ((\eta X_i + (2-\eta) X_j) y_j + y_j (\eta X_i + (2-\eta) X_j)) \\
&+ \sum_{i \neq j} (e_{\ell 1})_i (e_{1\ell})_k \otimes ((\eta s_{ij} (y_i \pm \eta) (X_i+X_j))+(\eta s_{ij} (X_i+X_j)) y_i) \\
&+ \sum_i ((e_{\ell\ell})_i \otimes (y_i X_i^2+X_i^2 y_i) y_i - (e_{11})_i \otimes y_i (y_i X_i^2+X_i^2 y_i) \\
&= \sum_{i \neq j} \sum_{k \neq i,j} (e_{\ell\ell})_i (e_{11}-e_{\ell\ell})_j (e_{11})_k \otimes \eta^2 \\
&+ \sum_{i \neq j} \sum_{k \neq i,j} (e_{\ell 1})_i ((e_{1\ell})_j (e_{11})_k - (e_{\ell\ell})_j (e_{1\ell})_k) \otimes (2-\eta) \eta \\
&+ \sum_{i \neq j} ((e_{\ell\ell})_i (e_{11}-e_{\ell\ell})_j \otimes (\eta X_i + (2-\eta) X_j) y_i - (e_{11})_i (e_{11}-e_{\ell\ell})_j \otimes y_i (\eta X_i + (2-\eta) X_j)) \\
&+ \sum_{i \neq j} (e_{\ell 1})_i (e_{1\ell})_j \otimes ((\eta X_i + (2-\eta) X_j) y_j + y_j (\eta X_i + (2-\eta) X_j)) \\
&+ \sum_{i \neq j} (e_{\ell\ell})_i (e_{11})_j \otimes ((\eta (y_i \pm \eta) (X_i+X_j))+(\eta (X_i+X_j)) y_i) \\
&+ \sum_i ((e_{\ell\ell})_i \otimes (y_i X_i^2+X_i^2 y_i) y_i - (e_{11})_i \otimes y_i (y_i X_i^2+X_i^2 y_i)
\end{align*}

The affine Lie algebra without central charge in the Cartan matrix presentation relates to the loop presentation in the following way:
\begin{align*}
H_i &\to h_i \otimes 1 \\
H_0 &\to q \frac{d}{dq} \\
E_i &\to e_i \otimes 1 \\
E_0 &\to e_0 \otimes q \\
F_i &\to f_i \otimes 1 \\
F_0 &\to f_0 \otimes q^{-1}
\end{align*}
The elements $e_0$ and $f_0$ are the unique non-zero elements for which $[e_0,f_i] = [f_0,e_i] = 0$ for $i=1,...,\ell$ and $[h_0,e_0] = 2e_0, [h_0,f_0] = -2f_0$ with $h_0 := [e_0,f_0]$, respectively. Then
\begin{equation*}
[E_0,E_i] = [e_0,e_i] \otimes q \overset ?= f_{i-1} \otimes q
\end{equation*}
This would mean
\begin{equation*}
J_{E_i}(q) = \sum_{n \in \Z} e_i \otimes q^{-n-1} = \sum_{n \in \Z} \operatorname{ad}_{E_0}^{-n-1} E_{i+n+1}
\end{equation*}
for $i=1,...,\ell$, where we switch $E$ for $F$ appropriately when $-n-1$ is negative.

Let us suppose that the QJT relation holds. Then we can write down higher QJT relations.
\begin{equation*}
Q_{0[1^k]}(z,q) J_0(q) T_{[1^k]}(z) = T_{[1^k]}(z) J_0(q) Q_{0[1^k]}(z,q),
\end{equation*}
where
\begin{equation*}
Q_{0[1^k]}(z,q) := Q_{0k}(z-\eta(k-1),q) \cdots Q_{01}(z,q), \quad
T_{[1^k]}(z) := T_1(z) \cdots T_k(z-\eta(k-1)).
\end{equation*}
Expanding this gives commutation relations between $A_i(z),B_i(z),C_i(z)$ and $H_{0,0}$ and $X_{0,0}^\pm$, from which we can check the Cartan presentation.

\section{Relations from quantized cotangent bundle}

The rational spin RS model comes from $(A, g, a, b) \in \mathfrak{gl}_\ell^* \times GL_\ell \times \Sigma_{N,\ell}$ which quantize to
\begin{align*}
[A_1,A_2] &= \frac{\hbar}{2} [C_{12},A_1-A_2] \\
[A_1,g_2] &=	 \hbar g_2 C_{12} \\
[g_1,g_2] &= 0 \\
[a_i^\alpha,b_j^\beta] &= \hbar \delta_{ij} \delta^{\alpha\beta}
\end{align*}
and such that $A,g$ and $a,b$ commute.

We now see in the classical case:
\begin{equation*}
\sum_{\alpha\beta} e_{\alpha\beta} \otimes S_i^{\alpha\beta} = \sum_{\alpha\beta} e_{\alpha\beta} \otimes \mathbf{a}_i^\alpha \mathbf{c}_i^\beta = (T^{-1} B U P)_{ii}, \quad B_{ij} := \sum_{\alpha\beta} e_{\alpha\beta} \otimes a_i^\alpha b_j^\beta
\end{equation*}
The $B$-operators fulfill
\begin{align*}
[B_{ij}^1,B_{kl}^2]
&= \sum_{\alpha\beta\gamma\delta} e_{\alpha\beta} \otimes e_{\gamma\delta} \otimes [a_i^\alpha b_j^\beta,a_k^\gamma b_l^\delta] \\
&= \hbar \bigg( \delta_{il} \sum_{\alpha\beta\gamma} e_{\alpha\beta} \otimes e_{\gamma\alpha} \otimes a_k^\gamma b_j^\beta
- \delta_{jk} \sum_{\alpha\beta\delta} e_{\alpha\beta} \otimes e_{\beta\delta} \otimes a_i^\alpha b_l^\delta \bigg) \\
&= \hbar \bigg( \delta_{il} \sum_{\alpha\beta\gamma} P^{12} (e_{\gamma\beta} \otimes e_{\alpha\alpha}) \otimes a_k^\gamma b_j^\beta
- \delta_{jk} \sum_{\alpha\beta\delta} P^{12}(e_{\beta\beta} \otimes e_{\alpha\delta}) \otimes a_i^\alpha b_l^\delta \bigg) \\
&= \hbar P^{12} (\delta_{il} B_{kj}^1 - \delta_{jk} B_{il}^2).
\end{align*}
This means
\begin{equation*}
[B_1^1,B_2^2] = \hbar P^{12} \bigg( \sum_{ijk} C_{12} (e_{kj} \otimes e_{ii}) B_{kj}^1 - \sum_{ijl} C_{12} (e_{jj} \otimes e_{il}) B_{il}^2 \bigg) = \hbar P^{12} C_{12} (B_1^1-B_2^2)
\end{equation*}
We can also write
\begin{align*}
[a_1,b_2] = \hbar C_{12}^r, \quad C_{12}^r := \sum_i \sum_\alpha e_{i\alpha} \otimes e_{\alpha i},
\end{align*}
which will give

The other commutation relations are
\begin{align*}
T_1 T_2 &= T_2 T_1 R_{12}, \quad U_1 U_2 = U_2 U_1 R_{12}^{-1} \\
T_1 P_2 &= P_2 T_1 \bar R_{12}, \quad U_1 P_2 = P_2 U_1 \bar R_{12} \\
[Q_1,P_2] &= \hbar P_2 \sum_i e_{ii} \otimes e_{ii}
\end{align*}
and all other relations are trivial commutation relations. In particular, we derive
\begin{equation*}
T_1^{-1} T_2^{-1} = R_{12} T_2^{-1} T_1^{-1}, \quad P_2 \bar R_{12}^{-1} T_1^{-1} = T_1^{-1} P_2, \quad P_1 U_2 = U_2 P_1 \bar R_{21}^{-1}.
\end{equation*}
We also derive
\begin{equation*}
U_1 P_1 U_2 P_2 = U_2 P_2 U_1 P_1 \underline{R}_{12}^{-1}, \quad \underline{R}_{12} := \bar R_{12}^{-1} R_{12} \bar R_{21}
\end{equation*}
using the relation
\begin{equation*}
R_{21} P_1 \bar R_{21}^{-1} P_2 = P_2 \bar R_{12}^{-1} P_1 \bar R_{21}^{-1} R_{21} \bar R_{12}.
\end{equation*}

This gives
\begin{align*}
R_{21} \tilde S_1^1 \bar R_{21}^{-1} \tilde S_2^2
&= \tilde S_2^2 \bar R_{12}^{-1} \tilde S_1^1 \underline{R}_{12}^{-1} + \hbar P_{12}^{12} \tilde S_1^1 \bar R_{21}^{-1} L_2 - \hbar P_{12}^{12} L_1 \bar R_{21}^{-1} \tilde S_2^2
\end{align*}
or
\begin{equation*}
(R_{21} \tilde S_1^1 + \hbar P_{12}^{12} L_1) \bar R_{21}^{-1} \tilde S_2^2 = \tilde S_2^2 \bar R_{12}^{-1} (\tilde S_1^1 \underline{R}_{12}^{-1} + \hbar L_1 P_{12}^{12})
\end{equation*}
We also find that $[Q_1,\tilde S_2] = \hbar \tilde S_2 \sum_i e_{ii} \otimes e_{ii}$.

Similarly
\begin{equation*}
R_{21} L_1 \bar R_{21}^{-1} \tilde S_2^2
= \tilde S_2^2 \bar R_{12}^{-1} L_1 \underline{R}_{21}
\end{equation*}

Let us take the trace $\operatorname{Tr}_{12} (e_{ii} \otimes e_{jj} -)$ of the first order part in $\hbar$, which will give the Poisson bracket $\{ \tilde S_{ii}^1, \tilde S_{jj}^2 \}$:
\begin{align*}
&\sum_{k \neq l} \frac{1}{q_{kl}} \operatorname{Tr}_{12} (e_{ii} \otimes e_{jj}) (f_{lk} \otimes f_{kl}) \tilde S_1^1 \tilde S_2^2 \\
&- \operatorname{Tr}_{12} (e_{ii} \otimes e_{jj})\tilde S_1^1 \sum_{k \neq l} \frac{1}{q_{kl}} (e_{ll} \otimes f_{kl}) \tilde S_2^2 \\
&+ \operatorname{Tr}_{12} (e_{ii} \otimes e_{jj})\tilde S_2^2 \sum_{k \neq l} \frac{1}{q_{kl}} (f_{kl} \otimes e_{ll}) \tilde S_1^1 \\
&- \operatorname{Tr}_{12} (e_{ii} \otimes e_{jj})\tilde S_2^2 \tilde S_1^1 \sum_{k \neq l} \frac{1}{q_{kl}} (e_{kk} \otimes e_{ll}-e_{kl} \otimes e_{lk}) \\
&- \operatorname{Tr}_{12} (e_{ii} \otimes e_{jj})\tilde S_2^2 L_1 P_{12}^{12} + \operatorname{Tr}_{12} (e_{ii} \otimes e_{jj})P_{12}^{12} L_1 \tilde S_2^2 \\
&= \frac{1}{q_{ji}} (\tilde S_{ii}^1-\tilde S_{ji}^1) (\tilde S_{jj}^2-\tilde S_{ij}^2) \\
&- \frac{1}{q_{ji}} \tilde S_{ii}^1 (\tilde S_{jj}^2-\tilde S_{ij}^2) \\
&+ \frac{1}{q_{ij}} \tilde S_{jj}^2 (\tilde S_{ii}^1-\tilde S_{ji}^1) \\
&- \frac{1}{q_{ij}} \tilde S_{jj}^2 \tilde S_{ii}^1 +
\frac{1}{q_{ji}} \tilde S_{ij}^1 \tilde S_{ij}^2 \\
&- \tilde S_{ij}^2 L_{ji} P^{12} + P^{12} L_{ji} \tilde S_{ij}^2 \\
&= -\frac{1}{q_{ij}} \tilde S_{ii}^1 S_{jj}^2 + \frac{1}{q_{ij}} \tilde S_{ji}^1 \tilde S_{jj}^2 + \frac{1}{q_{ij}} \tilde S_{ii}^1 \tilde S_{ij}^2 - \frac{1}{q_{ij}} \tilde S_{ji}^1 \tilde S_{ij}^2 \\
&+ \frac{1}{q_{ij}} \tilde S_{ii}^1 \tilde S_{jj}^2 - \frac{1}{q_{ij}} \tilde S_{ii}^1 \tilde S_{ij}^2 \\
&+ \frac{1}{q_{ij}} \tilde S_{jj}^2 \tilde S_{ii}^1-\frac{1}{q_{ij}} \tilde S_{jj}^2 \tilde S_{ji}^1 \\
&- \frac{1}{q_{ij}} \tilde S_{jj}^2 \tilde S_{ii}^1 -\frac{1}{q_{ij}} \tilde S_{ji}^2 \tilde S_{ij}^1 \\
&- \tilde S_{ij}^2 L_{ji} P^{12} + P^{12} L_{ji} \tilde S_{ij}^2 \\
&= \frac{1}{q_{ij}} \bigg( O(\hbar) - \tilde S_{ji}^1 \tilde S_{ij}^2 - \tilde S_{ji}^2 \tilde S_{ij}^1 \bigg) - P^{12}(\tilde S_{ji}^1 L_{ij} -L_{ji} \tilde S_{ij}^2),
\end{align*}
which is manifestly antisymmetric and invariant, where we have used
\begin{equation*}
R_{12} = 1 + \sum_{k \neq l} \frac{\hbar}{q_{kl}} f_{kl} \otimes f_{lk}, \quad
\bar R_{12}^{-1} = 1 - \sum_{k \neq l} \frac{\hbar}{q_{kl}} f_{kl} \otimes e_{ll}, \quad
\underline{R}_{12}^{-1} = 1 + \sum_{k \neq l} \frac{\hbar}{q_{kl}} (e_{kk} \otimes e_{ll} - e_{kl} \otimes e_{lk}).
\end{equation*}

Let us look at the algebra of $V := T^{-1} B U$ separately. Note that $\tilde S = VP$. We obtain:
\begin{align*}
(T_1^{-1} B_1^1 U_1) (T_2^{-1} B_2^2 U_2) R_{12}
&= T_1^{-1} T_2^{-1} B_1^1 B_2^2 U_1 U_2 R_{12} \\
&= R_{12} T_2^{-1} T_1^{-1} B_2^2 B_1^1 U_2 U_1 \\
&+ \hbar R_{12} T_2^{-1} T_1^{-1} P_{12}^{12}(B_1^1-B_2^2) U_2 U_1 \\
&= R_{12} (T_2^{-1} B_2^2 U_2) (T_1^{-1} B_1^1 U_1) \\
&+ \hbar P_{12}^{12} W_2 (T_1^{-1} B_1^1 U_1) \\
&- \hbar P_{12}^{12} (T_2^{-1} B_2^2 U_2) W_1,
\end{align*}
or in other words
\begin{equation*}
R_{21} V_1^1 V_2^2 + \hbar P_{12}^{12} W_1 V_2^2 = V_2^2 V_1^1 R_{21} + \hbar V_2^2 W_1 P_{12}^{12}.
\end{equation*}

\subsection{Comparison with Poisson bracket}

Let us derive the Poisson bracket of the $\tilde S$ both from the invariant oscillators and from the presentation $\tilde S = T^{-1}BUP$. The latter gives
\begin{align*}
\{ \tilde S_1^1, \tilde S_2^2 \}
&= \{ T_1^{-1} B_1^1 U_1 P_1, T_2^{-1} B_2^2 U_2 P_2 \} \\
&= (T_2^{-1} B_2^2 U_2 \{ T_1^{-1}, P_2 \} + \{ T_1^{-1}, T_2^{-1} \} B_2^2 U_2 P_2) B_1^1 U_1 P_1 \\
&+ T_1^{-1} T_2^{-1} \{ B_1^1, B_2^2 \} U_2 P_2 U_1 P_1 \\
&+ T_1^{-1} B_1^1 (T_2^{-1} B_2^2 U_2 \{ U_1, P_2 \} + T_2^{-1} B_2^2 \{ U_1, U_2 \} P_2) P_1 \\
&+ T_1^{-1} B_1^1 U_1 (T_2^{-1} B_2^2 \{ P_1, U_2 \} P_2 + \{ P_1, T_2^{-1} \} B_2^2 U_2 P_2) \\
&= (-T_2^{-1} B_2^2 U_2 P_2 \bar r_{12} T_1^{-1} + r_{12} T_1^{-1} T_2^{-1} B_2^2 U_2 P_2) B_1^1 U_1 P_1 \\
&+ T_1^{-1} T_2^{-1} P_{12}^{12} (B_1^1 - B_2^2) U_2 P_2 U_1 P_1 \\
&+ T_1^{-1} B_1^1 (T_2^{-1} B_2^2 U_2 U_1 P_2 \bar r_{12} - T_2^{-1} B_2^2 U_1 U_2 r_{12} P_2) P_1 \\
&+ T_1^{-1} B_1^1 U_1 (-T_2^{-1} B_2^2 U_2 P_1 \bar r_{21} P_2 + P_1 \bar r_{21} T_2^{-1} B_2^2 U_2 P_2) \\
&= - \tilde S_2^2 \bar r_{12} \tilde S_1^1 + r_{12} \tilde S_1 \tilde S_2 + P_{12}^{12} (\tilde S_1^1 L_2-L_1 \tilde S_2^2) + \underline{\tilde S_1^1 \tilde S_2^2 \bar r_{12} - \tilde S_1^1 \tilde S_2^2 r_{12} - \tilde S_1^1 \tilde S_2^2 \bar r_{21} + \tilde S_1^1 \bar r_{21} \tilde S_2^2} \\
&= r_{12} \tilde S_1 \tilde S_2 - \tilde S_2^2 \bar r_{12} \tilde S_1^1 + \tilde S_1^1 \bar r_{21} \tilde S_2^2 + \tilde S_1^1 \tilde S_2^2 \hat r_{12} + P_{12}^{12} (\tilde S_1^1 L_2-L_1 \tilde S_2^2)
\end{align*}
where
\begin{equation*}
r_{12} = \sum_{k \neq l} \frac{1}{q_{kl}} f_{kl} \otimes f_{lk}, \quad \bar r_{12} = \sum_{k \neq l} \frac{1}{q_{kl}} f_{kl} \otimes e_{ll}, \quad \hat r_{12} := \bar r_{12} - r_{12} - \bar r_{21} = \sum_{k \neq l} \frac{1}{q_{kl}} (e_{kk} \otimes e_{ll}-e_{kl} \otimes e_{lk})
\end{equation*}
and we have used
\begin{align*}
\{ T_1^{-1},P_2 \} &= -P_2 \bar r_{12} T_1^{-1}, \quad \{ T_1^{-1}, T_2^{-1} \} = r_{12} T_1^{-1} T_2^{-1}, \quad \{ B_1^1,B_2^2 \} = P_{12}^{12} (B_1^1-B_2^2), \\
\{ U_1,P_2 \} &= U_1 P_2 \bar r_{12}, \quad \{ U_1,U_2 \} = -U_1 U_2 r_{12}.
\end{align*}
Now taking the trace $\operatorname{Tr}_{12} e_{ii} \otimes e_{jj}$ gives
\begin{align*}
\{ \tilde S_{ii}^1, \tilde S_{jj}^2 \}
&= \frac{1}{q_{ij}}
(\tilde S_{ii}^1-\tilde S_{ji}^1)(\tilde S_{jj}^2-\tilde S_{ij}^2) \\
&- \frac{1}{q_{ij}}
\tilde S_{jj}^2 (\tilde S_{ii}^1-\tilde S_{ji}^1) \\
&+ \frac{1}{q_{ji}} \tilde S_{ii}^1 (\tilde S_{jj}^2 -\tilde S_{ij}^2) \\
&+\frac{1}{q_{ij}}
\tilde S_{ii}^1 \tilde S_{jj}^2
-\frac{1}{q_{ji}} \tilde S_{ij}^1 \tilde S_{ji}^2) \\
&+ P^{12} (\tilde S_{ji}^1 L_{ij}-L_{ji} \tilde S_{ij}^2) \\
&= \frac{1}{q_{ij}} \bigg( \tilde S_{ji}^1 \tilde S_{ij}^2 + \tilde S_{ji}^2 \tilde S_{ij}^1 \bigg) + P^{12} (\tilde S_{ji}^1 L_{ij}-L_{ji} \tilde S_{ij}^2),
\end{align*}
which is exactly what we get from the quantum case up to an overall sign. On the other hand, if we use the commutation relations among invariant oscillators, then using $\tilde{\mathbf{S}}_{ij}^{\alpha\beta} = \mathbf{a}_i^\alpha \mathbf{c}_j^\beta$, we get
\begin{align*}
\{ \tilde{\mathbf{S}}_{ii}^{\alpha\beta}, \tilde{\mathbf{S}}_{jj}^{\mu\nu} \}
&= \mathbf{a}_j^\mu \{ \mathbf{a}_i^\alpha, \mathbf{c}_j^\nu \} \mathbf{c}_i^\beta + \{ \mathbf{a}_i^\alpha, \mathbf{a}_j^\mu \} \mathbf{c}_j^\nu \mathbf{c}_i^\beta + \mathbf{a}_i^\alpha \mathbf{a}_j^\mu \{ \mathbf{c}_i^\beta, \mathbf{c}_j^\nu \} + \mathbf{a}_i^\alpha \{ \mathbf{c}_i^\beta, \mathbf{a}_j^\mu \} \mathbf{c}_j^\nu \\
&= \mathbf{a}_j^\mu (-\delta^{\alpha\nu} \mathbf{L}_{ij} + \mathbf{a}_i^\alpha \mathbf{L}_{ij} + \frac{1}{q_{ij}} \mathbf{c}_j^\nu (\mathbf{a}_j^\alpha-\mathbf{a}_i^\alpha)) \mathbf{c}_i^\beta \\
&+ \frac{1}{q_{ij}} (\mathbf{a}_i^\alpha \mathbf{a}_j^\mu+\mathbf{a}_j^\alpha \mathbf{a}_i^\mu-\mathbf{a}_i^\alpha \mathbf{a}_i^\mu-\mathbf{a}_j^\alpha \mathbf{a}_j^\mu) \mathbf{c}_j^\nu \mathbf{c}_i^\beta \\
&+ \mathbf{a}_i^\alpha \mathbf{a}_j^\mu (\frac{1}{q_{ij}} (\mathbf{c}_i^\beta \mathbf{c}_j^\nu + \mathbf{c}_j^\beta \mathbf{c}_i^\nu) + \mathbf{c}_j^\nu \mathbf{L}_{ji} - \mathbf{L}_{ij} \mathbf{c}_i^\beta) \\
&+ \mathbf{a}_i^\alpha (\delta^{\beta\mu} \mathbf{L}_{ji} - \mathbf{a}_j^\mu \mathbf{L}_{ji} + \frac{1}{q_{ij}} \mathbf{c}_i^\beta (\mathbf{a}_i^\mu-\mathbf{a}_j^\mu)) \mathbf{c}_j^\nu \\
&= \frac{1}{q_{ij}} \bigg( \tilde{\mathbf{S}}_{ji}^{\alpha\beta} \tilde{\mathbf{S}}_{ij}^{\mu\nu} + \tilde{\mathbf{S}}_{ji}^{\mu\nu} \tilde{\mathbf{S}}_{ij}^{\alpha\beta} \bigg) + \delta^{\beta\mu} \tilde{\mathbf{S}}_{ij}^{\alpha\nu} \mathbf{L}_{ji} - \delta^{\alpha\nu} \tilde{\mathbf{S}}_{ji}^{\mu\beta} \mathbf{L}_{ij} 
\end{align*}
We must have
\begin{equation*}
\tilde S_{ji}^{\alpha\beta} \tilde S_{ij}^{\mu\nu} + \tilde S_{ji}^{\mu\nu} \tilde S_{ij}^{\alpha\beta} = \tilde S_{ii}^{\mu\beta} \tilde S_{jj}^{\alpha\nu} + \tilde S_{ii}^{\alpha\nu} \tilde S_{jj}^{\beta\mu}, \quad \tilde S_{ij}^{\alpha\nu} L_{ji} = \frac{\sum_\gamma \tilde S_{ii}^{\alpha\gamma} \tilde S_{jj}^{\gamma\nu}}{q_{ji}-\gamma}, \quad \tilde S_{ji}^{\mu\beta} L_{ij} = \frac{\sum_\gamma \tilde S_{jj}^{\mu\gamma} \tilde S_{ii}^{\gamma\beta}}{q_{ij}-\gamma}
\end{equation*}
in order to satisfy the relation from the paper. It looks like the LHS is just the RHS multiplied by $C_{12} P^{12}$, according to
\begin{equation*}
C_{12} P^{12} \cdot \tilde S_{ij}^{\alpha\beta} \otimes \tilde S_{kl}^{\mu\nu} = \tilde S_{kj}^{\mu\beta} \otimes \tilde S_{il}^{\alpha\nu}.
\end{equation*}
This looks like the corresponding reduction in the quantum case, where we impose that exchanging two particles results in an $R$-matrix acting on spins.

We compute the missing Poisson bracket
\begin{align*}
\{ \tilde{\mathbf{S}}_{ij}^{\alpha\beta}, \mathbf{L}_{kl} \}
&= \{ \mathbf{a}_i^\alpha, \mathbf{L}_{kl} \} \mathbf{c}_j^\beta + \mathbf{a}_i^\alpha \{ \mathbf{c}_j^\beta, \mathbf{L}_{kl} \} \\
&= (\frac{1}{q_{ik}}(\mathbf{a}_i^\alpha-\mathbf{a}_k^\alpha) \mathbf{L}_{kl} - \frac{1}{q_{ik}}(\mathbf{a}_i^\alpha-\mathbf{a}_k^\alpha) \mathbf{L}_{il} - \frac{1}{q_{il}}(\mathbf{a}_i^\alpha-\mathbf{a}_l^\alpha)\mathbf{L}_{kl}) \mathbf{c}_j^\beta \\
&+ \mathbf{a}_i^\alpha (\frac{1}{q_{jl}}\mathbf{c}_j^\beta \mathbf{L}_{kl}+\frac{1}{q_{jl}}\mathbf{c}_l^\beta \mathbf{L}_{kj}-\frac{1}{q_{jk}} \mathbf{c}_j^\beta \mathbf{L}_{kl} + \frac{1}{q_{jk}} \mathbf{c}_j^\beta \mathbf{L}_{jl} + \mathbf{L}_{lj} \mathbf{L}_{kl}-\mathbf{L}_{kj} \mathbf{L}_{kl}) \\
&= \frac{1}{q_{ik}} (\tilde{\mathbf{S}}_{ij}^{\alpha\beta} - \tilde{\mathbf{S}}_{kj}^{\alpha\beta}) \mathbf{L}_{kl} - \frac{1}{q_{ik}} (\tilde{\mathbf{S}}_{ij}^{\alpha\beta}-\tilde{\mathbf{S}}_{kj}^{\alpha\beta}) \mathbf{L}_{il} - \frac{1}{q_{il}} (\tilde{\mathbf{S}}_{ij}^{\alpha\beta}-\tilde{\mathbf{S}}_{lj}^{\alpha\beta}) \mathbf{L}_{kl} \\
&+ \frac{1}{q_{jl}} \tilde{\mathbf{S}}_{ij}^{\alpha\beta} \mathbf{L}_{kl} + \frac{1}{q_{jl}} \tilde{\mathbf{S}}_{il}^{\alpha\beta} \mathbf{L}_{kj} - \frac{1}{q_{jk}} \tilde{\mathbf{S}}_{ij}^{\alpha\beta} \mathbf{L}_{kl} + \frac{1}{q_{jk}} \tilde{\mathbf{S}}_{ij}^{\alpha\beta} \mathbf{L}_{jl} + \mathbf{a}_i^\alpha \mathbf{L}_{lj} \mathbf{L}_{kl}-\mathbf{a}_i^\alpha \mathbf{L}_{kj} \mathbf{L}_{kl}
\end{align*}
On the other hand
\begin{align*}
\{ \tilde{S}_1^1, L_2 \}
&= \{ T_1^{-1} B_1^1 U_1 P_1, T_2^{-1} U_2 P_2 \} \\
&= (T_1^{-1} B_1^1 U_1 P_1 \bar r_{21} T_2^{-1} + r_{12} T_1^{-1} T_2^{-1} B_1^1 U_1 P_1) U_2 P_2 \\
&+ T_2^{-1} (-T_1^{-1} B_1^1 U_1 U_2 P_1 \bar r_{21} - T_1^{-1} B_1^1 U_1 U_2 r_{12} P_1) P_2 \\
&+ T_2^{-1} U_2 (T_1^{-1} B_1^1 U_1 P_2 \bar r_{12} P_1 - P_2 \bar r_{12} T_1^{-1} B_1^1 U_1 P_1) \\
&= \tilde S_1^1 \bar r_{21} L_2 + r_{12} \tilde S_1^1 L_2 + \tilde S_1^1 L_2 \hat r_{12} - L_2 \bar r_{12} \tilde S_1^1
\end{align*}

%Let us abbreviate $P^{12} C_{12} =: P_{12}^{12}$ and let us consider the $\hbar$-term on its own:
%\begin{equation*}
%\tilde S_2^2 \bar R_{12}^{-1} L_1 P_{12}^{12} = P_{12}^{12} L_1 \bar R_{21}^{-1} \tilde S_2^2
%\end{equation*}
%Taking $\operatorname{Tr}_{12}(C_{12}(e_{ii} \otimes e_{jj}) \cdots)$ of the LHS after plugging in the definitions gives
%\begin{align*}
%&\operatorname{Tr}_{12} \left[ C_{12}(e_{ii} \otimes e_{jj}) \tilde S_2^2 (1-\hbar \sum_{k \neq l} \frac{1}{y_k-y_l} (e_{kk}-e_{kl}) \otimes e_{ll}) L_1 P_{12}^{12} \right] \\
%&= \operatorname{Tr}_{12} \left[ C_{12}(e_{ii} \otimes e_{jj}) \tilde S_2^2 L_1 P_{12}^{12} \right]
%- \hbar \sum_{k \neq l} \frac{1}{y_k-y_l} \operatorname{Tr}_2\left[ C_{12} (e_{ii} \otimes e_{jj}) \tilde S_2^2 ((e_{kk}-e_{kl}) \otimes e_{ll}) L_1 P_{12}^{12} \right] \\
%&= \tilde S_{2,jj}^2 L_{1,ii} P^{12}
%- \hbar \sum_{k \neq l} \frac{1}{y_k-y_l} \operatorname{Tr}_{12} \left[ (e_{ii} \otimes e_{jj}) \tilde S_2^2 ((e_{kk}-e_{kl}) \otimes e_{ll}) L_1 \right] P^{12} \\
%&= \tilde S_{2,jj}^2 P^{12} \left( L_{1,ii}
%- \frac{\hbar}{y_i-y_j} (L_{1,ii} - L_{1,ji}) \right)
%\end{align*}
%while the RHS gives
%\begin{align*}
%P^{12} L_{1,jj} \left( \tilde S_{2,ii}^2
%- \frac{\hbar}{y_i-y_j} (\tilde S_{2,ii}^2-\tilde S_{2,ji}^2) \right)
%\end{align*}
%From this, we can conclude that $\tilde S_{2,jj}^2$ is diagonal and that $(\tilde S_{2,ii}^2)_{21} = -\frac{\hbar}{y_1-y_2} (\tilde S_{2,ji}^2)_{21}$, similarly for $1 \leftrightarrow 2$. Then solving for the diagonal elements gives no solutions, though. This implies that we have to look for solutions of the larger system!!

\subsection{Relation for diagonal elements}

Let us again take the equation
\begin{align*}
R_{21} \tilde S_1^1 \bar R_{21}^{-1} \tilde S_2^2
&= \tilde S_2^2 \bar R_{12}^{-1} \tilde S_1^1 \underline{R}_{12}^{-1} + \hbar (\tilde S_2^2 \bar R_{12}^{-1} L_1 P^{12} C_{12} - P^{12} C_{12} L_1 \bar R_{21}^{-1} \tilde S_2^2)
\end{align*}
and take the trace $\operatorname{Tr}_{12} e_{ii} \otimes e_{jj}$ to all orders of $\hbar$. The result is
\begin{align*}
[S_{ii}^1,S_{jj}^2]
&= \hbar \bigg( \frac{1}{q_{ij}} \bigg( [\tilde S_{ji}^1,\tilde S_{jj}^2] - \tilde S_{ji}^1 \tilde S_{ij}^2 - \tilde S_{ji}^2 \tilde S_{ij}^1 \bigg) - P^{12}(\tilde S_{ji}^1 L_{ij} -L_{ji} \tilde S_{ij}^2) \bigg) \\
&+ \hbar^2 \operatorname{Tr}_{12} e_{ii} \otimes e_{jj} \bigg( -\sum_{k \neq l} \frac{1}{q_{kl}} f_{lk} \otimes f_{kl} \tilde S_1^1 \sum_{k \neq l} \frac{1}{q_{kl}} e_{ll} \otimes f_{kl} \tilde S_2^2 \\
&+ \tilde S_2^2 \sum_{k \neq l} \frac{1}{q_{kl}} f_{kl} \otimes e_{ll} \tilde S_1^1 \sum_{k \neq l} \frac{1}{q_{kl}} (e_{kk} \otimes e_{ll}-e_{kl} \otimes e_{lk}) \\
&+ \tilde S_2^2 \sum_{k \neq l} \frac{1}{q_{kl}}f_{kl} \otimes e_{ll} L_1 P_{12}^{12}
- P_{12}^{12} L_1 \sum_{k \neq l} \frac{1}{q_{kl}} e_{ll} \otimes f_{kl} \tilde S_2^2 \bigg) \\
&= \hbar \bigg( \frac{1}{q_{ij}} \bigg( [\tilde S_{ji}^1,\tilde S_{jj}^2] - \tilde S_{ji}^1 \tilde S_{ij}^2 - \tilde S_{ji}^2 \tilde S_{ij}^1 \bigg) - P^{12}(\tilde S_{ji}^1 L_{ij} -L_{ji} \tilde S_{ij}^2) \bigg) \\
&+ \hbar^2 \bigg( -\sum_{k \neq l} \sum_{k' \neq l'} \frac{1}{q_{k'l'}} \frac{1}{q_{kl}} 
\operatorname{Tr}_1 e_{l'l'} e_{ii} f_{lk} \tilde S_1^1 \operatorname{Tr}_2 e_{jj} f_{kl} f_{k'l'} \tilde S_2^2 \\
&+ \sum_{k \neq l} \sum_{k' \neq l'} \frac{1}{q_{k'l'}} \frac{1}{q_{kl}}
(\operatorname{Tr}_2 e_{ll} e_{l'l'} e_{jj} \tilde S_2^2 \operatorname{Tr}_1 e_{k'k'} e_{ii} f_{kl} \tilde S_1^1 - \operatorname{Tr}_2 e_{ll} e_{l'k'} e_{jj} \tilde S_2^2 \operatorname{Tr}_1 e_{k'l'} e_{ii} f_{kl} \tilde S_1^1 \\
&+ \sum_{k \neq l} \frac{1}{q_{kl}}
\operatorname{Tr}_2 e_{ll} e_{ij} \tilde S_2^2 \operatorname{Tr}_1 e_{ji} f_{kl} L_1 P^{12}
- \sum_{k \neq l} \frac{1}{q_{kl}}
P^{12} \operatorname{Tr}_1 e_{ll} e_{ij} L_1 \operatorname{Tr}_2 e_{ji} f_{kl} \tilde S_2^2 \bigg) \\
&= \hbar \bigg( \frac{1}{q_{ij}} \bigg( [\tilde S_{ji}^1,\tilde S_{jj}^2] - \tilde S_{ji}^1 \tilde S_{ij}^2 - \tilde S_{ji}^2 \tilde S_{ij}^1 \bigg) - P^{12}(\tilde S_{ji}^1 L_{ij} -L_{ji} \tilde S_{ij}^2) \bigg) \\
&+ \hbar^2 \frac{1}{q_{ij}^2} \bigg( -(\tilde S_{ii}^1-\tilde S_{ji}^1) (\tilde S_{jj}^2-\tilde S_{ij}^2) + \tilde S_{jj}^2 (\tilde S_{ii}^1-\tilde S_{ji}^1) \bigg) \\
&= \hbar \bigg( \frac{1}{q_{ij}} \bigg( [\tilde S_{ji}^1,\tilde S_{jj}^2] - \tilde S_{ji}^1 \tilde S_{ij}^2 - \tilde S_{ji}^2 \tilde S_{ij}^1 \bigg) - P^{12}(\tilde S_{ji}^1 L_{ij} -L_{ji} \tilde S_{ij}^2) \bigg) \\
&+ \hbar^2 \frac{1}{q_{ij}^2} \bigg( -[\tilde S_{ii}^1,\tilde S_{jj}^2] + \tilde S_{ii}^1 \tilde S_{ij}^2 + [\tilde S_{ji}^1,\tilde S_{jj}^2] - \tilde S_{ji}^1 \tilde S_{ij}^2 \bigg)
\end{align*}
Mathematica gives the opposite sign, which is correct:
\begin{align*}
[\tilde S_{ii}^1,\tilde S_{jj}^2]
&= \frac{\hbar}{q_{ij}} (\tilde S_{ji}^2 \tilde S_{ij}^1 + \tilde S_{ji}^1 \tilde S_{ij}^2 - [\tilde S_{ji}^1,\tilde S_{jj}^2]) - \hbar P^{12} (L_{ji} \tilde S_{ij}^2 - \tilde S_{ji}^1 L_{ij}) \\
&+ \frac{\hbar^2}{q_{ij}^2} ([\tilde S_{ii}^1,\tilde S_{jj}^2] - [\tilde S_{ji}^1,\tilde S_{jj}^2] - \tilde S_{ii}^1 \tilde S_{ij}^2 + \tilde S_{ji}^1 \tilde S_{ij}^2).
\end{align*}
Rearranging, we get
\begin{align*}
(1 - \frac{\hbar^2}{q_{ij}^2})[S_{ii}^1,S_{jj}^2]
&= \frac{\hbar}{q_{ij}} \tilde S_{ji}^2 \tilde S_{ij}^1 + \hbar P^{12}(\tilde S_{ji}^1 L_{ij}-L_{ji} \tilde S_{ij}^2) \\
&- \frac{\hbar^2}{q_{ij}^2} \tilde S_{ii}^1 \tilde S_{ij}^2 + \frac{\hbar}{q_{ij}} (1 + \frac{\hbar}{q_{ij}}) \tilde S_{ji}^1 \tilde S_{ij}^2 - \frac{\hbar}{q_{ij}} (1 + \frac{\hbar}{q_{ij}}) [\tilde S_{ji}^1,\tilde S_{jj}^2]
\end{align*}
We also have
\begin{align*}
[\tilde S_{ji}^1,\tilde S_{jj}^2]
&= \frac{\hbar}{q_{ij}} (\tilde S_{ji}^2 \tilde S_{jj}^1+\tilde S_{ji}^1 \tilde S_{ij}^2 - [\tilde S_{ji}^1,\tilde S_{jj}^2]) + \hbar P^{12}(\tilde S_{ji}^1 L_{jj} - L_{ji} \tilde S_{jj}^2) \\
&+ \frac{\hbar^2}{q_{ij}^2} (\tilde S_{ji}^2 \tilde S_{jj}^1 - \tilde S_{ji}^2 \tilde S_{ij}^1) + \frac{\hbar^2}{q_{ij}} P^{12}(L_{ji} \tilde S_{ij}^2 - L_{ji} \tilde S_{jj}^2 - \tilde S_{ji}^2 L_{ij} + \tilde S_{ji}^2 L_{jj})
\end{align*}
which rearranged is
\begin{align*}
(1+\frac{\hbar}{q_{ij}}) [\tilde S_{ji}^1,\tilde S_{jj}^2]
&= \frac{\hbar}{q_{ij}} \tilde S_{ji}^1 \tilde S_{ij}^2 + \hbar (1 + \frac{\hbar}{q_{ij}}) P^{12}(\tilde S_{ji}^1 L_{jj} - L_{ji} \tilde S_{jj}^2) \\
&- \frac{\hbar^2}{q_{ij}^2} \tilde S_{ji}^2 \tilde S_{ij}^1
+ \frac{\hbar}{q_{ij}} (1 + \frac{\hbar}{q_{ij}}) \tilde S_{ji}^2 \tilde S_{jj}^1
+ \frac{\hbar^2}{q_{ij}} P^{12}(L_{ji} \tilde S_{ij}^2 - \tilde S_{ji}^1 L_{ij})
\end{align*}
Substituting this gives
\begin{align*}
(1 - \frac{\hbar^2}{q_{ij}^2})[S_{ii}^1,S_{jj}^2]
&= \frac{\hbar}{q_{ij}} (1 + \frac{\hbar^2}{q_{ij}^2}) \tilde S_{ji}^2 \tilde S_{ij}^1 + \hbar (1 + \frac{\hbar^2}{q_{ij}^2}) P^{12}(\tilde S_{ji}^1 L_{ij}-L_{ji} \tilde S_{ij}^2) - \frac{\hbar^2}{q_{ij}^2} (\tilde S_{ii}^1 \tilde S_{ij}^2 + \tilde S_{ji}^1 \tilde S_{ij}^2) \\
&+ \frac{\hbar}{q_{ij}} (1 + \frac{\hbar}{q_{ij}}) \tilde S_{ji}^1 \tilde S_{ij}^2 - \frac{\hbar^2}{q_{ij}} (1 + \frac{\hbar}{q_{ij}}) P^{12}(\tilde S_{ji}^1 L_{jj} - L_{ji} \tilde S_{jj}^2) - \frac{\hbar^2}{q_{ij}^2} (1 + \frac{\hbar}{q_{ij}}) \tilde S_{ji}^2 \tilde S_{jj}^1
\end{align*}

We have to use the oscillator representation to reorder the indices. Clearly we have
\begin{equation*}
B_{ij}^{\alpha\beta} B_{kl}^{\mu\nu}
= a_i^\alpha b_j^\beta a_k^\mu b_l^\nu
= a_i^\alpha a_k^\mu b_j^\beta b_l^\nu + a_i^\alpha [b_j^\beta,a_k^\mu] b_l^\nu
= a_k^\mu a_i^\alpha b_j^\beta b_l^\nu - \hbar \delta_{jk}^{\beta\mu} a_i^\alpha b_l^\nu
= B_{kj}^{\mu\beta} B_{il}^{\alpha\nu} + \hbar \delta_{ij}^{\alpha\beta} B_{kl}^{\mu\nu} - \hbar \delta_{kj}^{\mu\beta} B_{il}^{\alpha\nu}
\end{equation*}
or
\begin{equation*}
B_1^1 B_2^2 = C_{12} P^{12} B_1^1 B_2^2 + \hbar B_2^2 - \hbar C_{12} P^{12} B_2^2.
\end{equation*}
Similarly,
\begin{equation*}
B_{ij}^{\alpha\beta} B_{kl}^{\mu\nu}
= a_i^\alpha b_j^\beta a_k^\mu b_l^\nu
= a_i^\alpha b_l^\nu a_k^\mu b_j^\beta
- \hbar \delta_{jk}^{\beta\mu} a_i^\alpha b_l^\nu
+ \hbar \delta_{kl}^{\mu\nu} a_i^\alpha b_j^\beta
= B_{il}^{\alpha\nu} B_{kj}^{\mu\beta}
- \hbar \delta_{jk}^{\beta\mu} B_{il}^{\alpha\nu}
+ \hbar \delta_{kl}^{\mu\nu} B_{ij}^{\alpha\beta},
\end{equation*}
or
\begin{equation*}
B_1^1 B_2^2 = B_1^1 B_2^2 C_{12} P^{12} - \hbar B_1^1 C_{12} P^{12} + \hbar B_1^1.
\end{equation*}
For $V := T^{-1} B U$, this means
\begin{align*}
(V_1^1-\hbar W_1) V_2^2 &= C_{12} P^{12} R_{21} (V_1^1-\hbar W_1) V_2^2, \\
V_1^1 (V_2^2-\hbar W_2) &= V_1^1 (V_2^2-\hbar W_2) R_{12} C_{12} P^{12}
\end{align*}
For $\tilde S := VP$, this means
\begin{align*}
(P_{12}^{12} \tilde S_1^1 + \hbar R_{21} L_1) \bar R_{21}^{-1} \tilde S_2^2 &= (R_{21} \tilde S_1^1 + \hbar P_{12}^{12} L_1) \bar R_{21}^{-1} \tilde S_2^2, \\
\tilde S_2^2 \bar R_{12}^{-1} (\tilde S_1^1 P_{12}^{12} + \hbar L_1 \underline{R}_{21}) &= \tilde S_2^2 \bar R_{12}^{-1} (\tilde S_1^1 \underline{R}_{21} + \hbar L_1 P_{12}^{12})
\end{align*}
The relation
\begin{align*}
(R_{21} \tilde S_1^1 + \hbar P^{12}_{12} L_1) \bar R_{21}^{-1} \tilde S_2^2
&= \tilde S_2^2 \bar R_{12}^{-1} (\tilde S_1^1 \underline{R}_{21} + \hbar L_1 P^{12}_{12})
\end{align*}
can thus also be written as
\begin{equation*}
(P_{12}^{12} \tilde S_1^1 + \hbar R_{21} L_1) \bar R_{21}^{-1} \tilde S_2^2 = \tilde S_2^2 \bar R_{12}^{-1} (\tilde S_1^1 P_{12}^{12} + \hbar L_1 \underline{R}_{21})
\end{equation*}
Using the relation
\begin{equation*}
R_{21} L_1 \bar R_{21}^{-1} \tilde S_2^2
= \tilde S_2^2 \bar R_{12}^{-1} L_1 \underline{R}_{21},
\end{equation*}
this reduces to
\begin{equation*}
P_{12}^{12} \tilde S_1^1 \bar R_{21}^{-1} \tilde S_2^2 = \tilde S_2^2 \bar R_{12}^{-1} \tilde S_1^1 P_{12}^{12},
\end{equation*}
which is trivial.

\pagebreak

In terms of indices, we have
\begin{align*}
\tilde S_{ji}^1 \tilde S_{lk}^2 - \hbar \frac{1}{q_{li}} \tilde S_{ji}^1 (\tilde S_{lk}^2-\tilde S_{ik}^2)
&= P^{12} \tilde S_{li}^1 \tilde S_{jk}^2 \\
&+ \hbar P^{12} \frac{1}{q_{jl}} (\tilde S_{li}^1-\tilde S_{ji}^1) (\tilde S_{jk}^2-\tilde S_{lk}^2) \\
&-\hbar P^{12} \frac{1}{q_{ji}} \tilde S_{li}^1 (\tilde S_{jk}^2-\tilde S_{ik}^2) \\
&-\hbar^2 P^{12} \frac{1}{q_{jl}} \frac{1}{q_{ji}} (\tilde S_{li}^1-\tilde S_{ji}^1) (\tilde S_{jk}^2-\tilde S_{ik}^2) \\
&+\hbar^2 P^{12} \frac{1}{q_{jl}} \frac{1}{q_{li}} (\tilde S_{li}^1-\tilde S_{ji}^1) (\tilde S_{lk}^2-\tilde S_{ik}^2) \\
&+ \hbar L_{ji} \tilde S_{lk}^2 - \hbar P^{12} L_{li} \tilde S_{jk}^2 - \hbar^2 P^{12} \frac{1}{q_{jl}} (L_{li}-L_{ji}) (\tilde S_{jk}^2-\tilde S_{lk}^2)
\end{align*}
For $l=i$ and $k=j$, this reduces to
\begin{align*}
\tilde S_{ji}^1 \tilde S_{ij}^2
&= P^{12} \tilde S_{ii}^1 \tilde S_{jj}^2 \\
&- \hbar P^{12} \frac{1}{q_{ji}} \tilde S_{ji}^1 (\tilde S_{jj}^2-\tilde S_{ij}^2) \\
&-\hbar^2 P^{12} \frac{1}{q_{ji}^2} (\tilde S_{ii}^1-\tilde S_{ji}^1) (\tilde S_{jj}^2-\tilde S_{ij}^2) \\
&+ \hbar L_{ji} \tilde S_{ij}^2 - \hbar P^{12} L_{ii} \tilde S_{jj}^2 - \hbar^2 P^{12} \frac{1}{q_{ji}} (L_{ii}-L_{ji}) (\tilde S_{jj}^2-\tilde S_{ij}^2)
\end{align*}
while for $j=i,l=i,k=j$, this means
\begin{align*}
\tilde S_{ii}^1 \tilde S_{ij}^2
&= P^{12} \tilde S_{ii}^1 \tilde S_{ij}^2 + \hbar L_{ii} \tilde S_{ij}^2 - \hbar P^{12} L_{ii} \tilde S_{ij}^2
\end{align*}
or for $l=i,k=i$:
\begin{align*}
\tilde S_{ji}^1 \tilde S_{ii}^2
&= P^{12} \tilde S_{ii}^1 \tilde S_{ji}^2 \\
&- \hbar P^{12} \frac{1}{q_{ji}} \tilde S_{ji}^1 (\tilde S_{ji}^2-\tilde S_{ii}^2) \\
&-\hbar^2 P^{12} \frac{1}{q_{ji}^2} (\tilde S_{ii}^1-\tilde S_{ji}^1) (\tilde S_{ji}^2-\tilde S_{ii}^2) \\
&+ \hbar L_{ji} \tilde S_{ii}^2 - \hbar P^{12} L_{ii} \tilde S_{ji}^2 - \hbar^2 P^{12} \frac{1}{q_{ji}} (L_{ii}-L_{ji}) (\tilde S_{ji}^2-\tilde S_{ii}^2)
\end{align*}

\subsection{Computation with appropriate shifts}

Before, we ignored how $P$ commutes with $Q$ by shifts. Let us again take
\begin{align*}
(R_{21} \tilde S_1^1 + \hbar P^{12}_{12} L_1) \bar R_{21}^{-1} \tilde S_2^2
&= \tilde S_2^2 \bar R_{12}^{-1} (\tilde S_1^1 \underline{R}_{21} + \hbar L_1 P^{12}_{12}).
\end{align*}
Then we obtain
\begin{align*}
(1+\frac{\hbar}{q_{ij}})[\tilde S_{ii}^1,\tilde S_{jj}^2]
&= -\hbar(L_{ji} \tilde S_{ij}^2-\tilde S_{ji}^2 L_{ij}) \\
&- \frac{\hbar}{q_{ij}+\hbar} \tilde S_{ji}^1 \tilde S_{ij}^2
+ \frac{\hbar}{q_{ij}} \tilde S_{ji}^2 \tilde S_{ij}^1
+ \frac{\hbar}{q_{ij}}\tilde S_{ii}^1 \tilde S_{jj}^2
+ (1 \oplus \frac{\hbar}{q_{ij}}) \frac{\hbar}{q_{ij}-\hbar}\tilde S_{jj}^2 \tilde S_{ii}^1 \\
&- (1+\frac{\hbar}{q_{ij}}) \frac{\hbar}{q_{ij}-\hbar} \tilde S_{jj}^2 \tilde S_{ji}^1
+ \frac{\hbar}{q_{ij}+\hbar} \tilde S_{ji}^1 \tilde S_{jj}^2
\end{align*}
If the $\oplus$ is the wrong sign, then we would actually have
\begin{align*}
[\tilde S_{ii}^1,\tilde S_{jj}^2]
&= -\hbar(L_{ji} \tilde S_{ij}^2-\tilde S_{ji}^2 L_{ij}) - \frac{\hbar}{q_{ij}+\hbar} \tilde S_{ji}^1 \tilde S_{ij}^2
+ \frac{\hbar}{q_{ij}} \tilde S_{ji}^2 \tilde S_{ij}^1 \\
&- (1+\frac{\hbar}{q_{ij}}) \frac{\hbar}{q_{ij}-\hbar} \tilde S_{jj}^2 \tilde S_{ji}^1
+ \frac{\hbar}{q_{ij}+\hbar} \tilde S_{ji}^1 \tilde S_{jj}^2,
\end{align*}
which is not manifestly skew-symmetric.

\subsection{New variables}

From the point of view of invariance, we have classically
\begin{equation*}
S_i^{\alpha\beta} \to h[T]_i^{-1} (T^{-1} a^\alpha)_i (b^\beta UP)_i h[T]_i,
\end{equation*}
which is unnatural, since $h[T]$ does not commute easily, as it is a function of $T$. Rather, we should take the ordering
\begin{equation*}
S_i^{\alpha\beta} = (b^\beta UP)_i (T^{-1} a^\alpha)_i = \sum_{kj} (UP)_{ki} T_{ij}^{-1} b_k^\beta a_j^\alpha = \operatorname{Tr} UP e_{ii} T^{-1} B^{\alpha\beta}
\end{equation*}
We have determined
\begin{equation*}
h[T]_2 P_1 = P_1 \bar R_{21}^{-1} h[T]_2.
\end{equation*}
The transformation $T \to h T h[T]$ does not leave the quantum algebra relations invariant. Instead, $h$ must be taken to be in the quantized Frobenius group. The isotropy group of the moment map in the classical case is
\begin{equation*}
F := \{ g \in GL_N \mid ge = e, e^t g = e^t \}.
\end{equation*}
This gets quantized to the algebra
\begin{equation*}
F_q: R_{12} W_1 W_2 = W_2 W_1 R_{12}, \quad (\bar W_1 \bar W_2 R_{12} = R_{12} \bar W_2 \bar W_1).
\end{equation*}
with
\begin{align*}
W_1 P_2 \bar R_{12}^{-1} = P_2 \bar R_{12}^{-1} W_1, \quad [Q_1,P_2] = \hbar P_2 \sum_i e_{ii} \otimes e_{ii}.
\end{align*}

\subsection{Factorization homology}

Using the theory of factorization homology of surfaces, we can start with the quantum Frobenius group $\mathcal{F}_q$, defined as a formal quantum group by the $R$-matrix $R = 1 + \hbar r, r = \sum_{i \neq j} \frac{1}{q_{ij}} f_{ij} \otimes f_{ji}, f_{ij} = e_{ii}-e_{ij}$, so
\begin{equation*}
R_{12} W_1 W_2 = W_2 W_1 R_{12}.
\end{equation*}
This has a monoidal category of representations and there is a braiding, at least for tensor products of vector representations. If we take factorization homology with coefficients in $\mathcal{F}_q$, then the disc will give back $\mathcal{F}_q$ and the annulus will give back $\mathcal{F}_q^\vee$, with relations
\begin{equation*}
R_{12} W_1^\vee R_{21} W_2^\vee = W_2^\vee R_{12} W_1^\vee R_{21}.
\end{equation*}

Now, if $\mathcal{A}$ is the algebra of the quantized cotangent bundle (similar to $\mathcal{D}_q(GL_N)$ associated to the punctured torus in \cite{article:benzvi2:2018}), meaning $T^{-1},U,Q,P$, then there should be a simple moment map
\begin{equation*}
\mu: \mathcal{F}_q^\vee \to \mathcal{A}, \quad W^\vee \mapsto T^{-1} U Q U^{-1} T - Q
\end{equation*}
(however this is not a homomorphism) and the usual rational RS model corresponds to an ideal $\mathcal{I}$ in $\mathcal{F}_q^\vee$ generated by $W^\vee - \gamma(e \otimes e - I)$. The quantum Hamiltonian reduction is
\begin{equation*}
(\mathcal{A}/\mathcal{A} \mu(\mathcal{I}))^{\mathcal{F}_q^\vee}
\end{equation*}

\subsection{Quantum Hamiltonian reduction}

Introduce the algebra $\mathcal{A}$ generated by a matrix $T$ and a diagonal matrix $Q$ such that $T_1 T_2 = T_2 T_1 R_{12}$ and everything else commutes. The analog of the braided dual of this algebra is $\mathcal{F}$ generated by a matrix $W$ satisfying $R_{12} W_1 W_2 = W_2 W_1 R_{12}$ as well as $Q$. The analog of the elliptic double is $\mathcal{D}$ generated by $T, U, Q, P$ satisfying
\begin{align*}
T_1 T_2 = T_2 T_1 R_{12}, \quad U_1 U_2 R_{12} = U_2 U_1, \quad T_1 P_2 = P_2 T_1 \bar R_{12}, \quad U_1 P_2 = P_2 U_1 \bar R_{12}, \quad [Q_1,P_2] = \hbar P_2 \sum_i e_{ii} \otimes e_{ii}.
\end{align*}
This is the quantum cotangent bundle.

%Note the relation
%\begin{align*}
%(T_1^{-1} A_1 U_1) (T_2^{-1} A_2 U_2) R_{12}
%&= 
%R_{12} (T_2^{-1} A_2 U_2) (T_1^{-1} A_1 U_1)
%\end{align*}
%for any $A$ with $[A_1,A_2] = [A_1,T_2] = [A_1,U_2] = 0$. We also have
%\begin{equation*}
%(T_1^{-1} A_1 T_1) R_{12} (T_2^{-1} A_2 T_2) R_{21} = R_{12} (T_2^{-1} A_2 T_2) R_{21} (T_1^{-1} A_1 T_1)
%\end{equation*}
%and
%\begin{equation*}
%(U_1^{-1} A_1 U_1) R_{21} (U_2^{-1} A_2 U_2) R_{12} = R_{21} (U_2^{-1} A_2 U_2) R_{12} (U_1^{-1} A_1 U_1)
%\end{equation*}
Importantly, we observe
\begin{align*}
(T_1 Q_1 T_1^{-1}) (T_2 Q_2 T_2^{-1})
&= T_2 T_1 R_{12} Q_1 R_{21} Q_2 R_{12} T_2^{-1} T_1^{-1} \\
&= (T_2 Q_2 T_2^{-1}) (T_1 Q_1 T_1^{-1}) \\
&+ \hbar C_{12} ((T_1 Q_1 T_1^{-1})-(T_2 Q_2 T_2^{-1}))
\end{align*}
and
\begin{align*}
(U_1 Q_1 U_1^{-1}) (U_2 Q_2 U_2^{-1})
&= U_2 U_1 R_{21} Q_1 R_{12} Q_2 R_{21} U_2^{-1} U_1^{-1} \\
&= U_2 U_1 Q_2 R_{21} Q_1 U_2^{-1} U_1^{-1} \\
&- \hbar C_{12} ((U_1 Q_1 U_1^{-1})-(U_2 Q_2 U_2^{-1}))
\end{align*}
where we have used
\begin{align*}
R_{12} Q_1 R_{21} Q_2 - Q_2 R_{12} Q_1 R_{21} &= \hbar C_{12} Q_1 R_{21} - \hbar R_{12} Q_1 C_{12} \\
R_{21} Q_1 R_{12} Q_2 - Q_2 R_{21} Q_1 R_{12} &= \hbar C_{12} Q_1 \hat R_{21} - \hbar \hat R_{12} Q_1 C_{12}
\end{align*}
with $\hat R_{12} = R_{12} - 2$ satisfying $\hat R_{12} R_{12} = -1$. These relations match exactly with the Lie algebra satisfied by $a \otimes b$ due to $[a_i,b_j] = -\hbar \delta_{ij}$ (note the minus sign):
\begin{align*}
[(a \otimes b)_1,(a \otimes b)_2] = -\hbar C_{12} ((a \otimes b)_1 + (a \otimes b)_2)
\end{align*}
Furthermore $TQT^{-1},UQU^{-1},a \otimes b$ all commute among each other, making the quantum moment map
\begin{equation*}
\mu: \mathcal{U}(\mathfrak{gl}_N) \to \mathcal{D}', \quad A \mapsto TQT^{-1} - UQU^{-1} - a \otimes b
\end{equation*}
into a homomorphism, having added oscillators $a_i,b_j$ onto $\mathcal{D}$. The quantum rational RS model is obtained by reduction along $\mu$ at the ideal $\mathcal{I}$ generated by $A-\gamma$. Usually, one just uses the moment map $TQT^{-1}-UQU^{-1}$ and reduces at the ideal $A-\gamma(e \otimes e-1)$, but we have introduced the oscillator matrix $a \otimes b$ replacing $\gamma e \otimes e$.

We can now add more oscillators $a_i^\alpha,b_j^\beta$, giving the moment map
\begin{equation*}
\mu: \mathcal{U}(\mathfrak{gl}_N) \to \bar{\mathcal{D}}, \quad A \mapsto TQT^{-1} - UQU^{-1} - \sum_\alpha a^\alpha \otimes b^\alpha
\end{equation*}
and we can perform the same reduction, giving the quantum rational spin RS model. For this, we need to find the induced action. This is just given by the Lie algebra adjoint action on each of the mutually commuting subalgebras, which gives:
\begin{align*}
\operatorname{ad}_{A_1} T_2
&= [T_1 Q_1 T_1^{-1},T_2] \\
&= T_2 T_1 (R_{12} Q_1 R_{21}-Q_1) T_1^{-1} \\
&= -\hbar C_{12} T_2 + \hbar T_2 (T_1 X_{21} T_1^{-1}) \\
\operatorname{ad}_{A_1} U_2
&= [-U_1 Q_1 U_1^{-1},U_2] \\
&= -U_2 U_1 (R_{21} Q_1 R_{12}-Q_1) U_1^{-1} \\
&= -\hbar C_{12} U_2 + \hbar U_2 (U_1 X_{21} U_1^{-1}) \\
\operatorname{ad}_{A_1} (a^\alpha \otimes b^\beta)_2 &= -\hbar (a^\alpha \otimes b^\beta)_2 C_{12} + \hbar C_{12} (a^\alpha \otimes b^\beta)_2 \\
\operatorname{ad}_{A_1} Q_2 &= 0 \\
\operatorname{ad}_{A_1} P_2 &= [T_1 Q_1 T_1^{-1},P_2] - [U_1 Q_1 U_1^{-1},P_2] \\
&= P_2 T_1 (\bar R_{12} Q_1 \bar R_{12}^{-1}-Q_1) T_1^{-1}
+ \hbar P_2 T_1 \bar R_{12} \sum_i e_{ii} \otimes e_{ii} \bar R_{12}^{-1} T_1^{-1} \\
&- P_2 U_1 (\bar R_{12} Q_1 \bar R_{12}^{-1}-Q_1) U_1^{-1}
- \hbar P_2 U_1 \bar R_{12} \sum_i e_{ii} \otimes e_{ii} \bar R_{12}^{-1} U_1^{-1} \\
&= \hbar P_2 T_1 \bar R_{12} X_{21} T_1^{-1} - \hbar P_2 U_1 \bar R_{12} X_{21} U_1^{-1} \\
&= \hbar (T_1 X_{21} T_1^{-1}) P_2 - \hbar (U_1 X_{21} U_1^{-1}) P_2
\end{align*}
where $X_{12} := \sum_{ij} e_{ii} \otimes e_{ji}$ and we have used
\begin{align*}
R_{12} Q_1 R_{21}-Q_1 &= -\hbar C_{12} (1-X_{12}) R_{21} \\
R_{21} Q_1 R_{12}-Q_1 &= \hbar C_{12} (1-X_{12}) R_{12} \\
R_{12} X_{12} = X_{12}, \quad R_{21} X_{12} &= X_{12}, \quad X_{12} R_{12} = X_{12}, \quad X_{12} R_{21} = X_{12} \\
\bar R_{21} X_{12} = X_{12}, \quad X_{12} \bar R_{12} &= X_{12}, \quad X_{12} \bar R_{21} = X_{12}, \quad C_{12} X_{12} = X_{21} \\
\sum_i e_{ii} \otimes e_{ii} \bar R_{12}^{-1} &= \sum_i e_{ii} \otimes e_{ii} \\
\bar R_{12} Q_1 \bar R_{12}^{-1} - Q_1 &= \hbar \bar R_{12} C_{12} \sum_{i \neq j} e_{ii} \otimes e_{ji}
\end{align*}
Inserting a Lie algebra element $h$ gives:
\begin{equation*}
\operatorname{Tr}_1 h_1 \operatorname{ad}_{A_1} T_2
= \hbar (hT - \frac{1}{N} \sum_{ij} T e_{ii} T^{-1} (he \otimes e) e_{ji}),
\end{equation*}
where the second term corresponds to the quantization of $h[T]$. This indeed preserves the Frobenius condition for $T$. Also, for $h$ in the Frobenius Lie algebra, $\operatorname{Tr}_1 h_1 \operatorname{ad}_{A_1} T_2$ indeed becomes zero.

It furthermore follows that
\begin{align*}
\operatorname{ad}_{A_1} (U_2 P_2)
&= -\hbar C_{12} (U_2 P_2) + \hbar (U_2 P_2) (T_1 X_{21} T_1^{-1}),
\end{align*}
so
\begin{align*}
\operatorname{ad}_{A_1} (T_2^{-1} U_2 P_2)
&= -T_2^{-1} \operatorname{ad}_{A_1}(T_2) T_2^{-1} U_2 P_2 + T_2^{-1} \operatorname{ad}_{A_1} (U_2 P_2) \\
&= \hbar (T_2^{-1} U_2 P_2) (T_1 X_{21} T_1^{-1}) - \hbar (T_1 X_{21} T_1^{-1}) (T_2^{-1} U_2 P_2)
\end{align*}
as well as
\begin{align*}
\operatorname{ad}_{A_1}(T_2^{-1} (a^\alpha \otimes b^\beta)_2 U_2 P_2)
&= \operatorname{ad}_{A_1}(T_2^{-1}) (a^\alpha \otimes b^\beta)_2 U_2 P_2 \\
&+ T_2^{-1} \operatorname{ad}_{A_1}(a^\alpha \otimes b^\beta) U_2 P_2 \\
&+ T_2^{-1} (a^\alpha \otimes b^\beta) \operatorname{ad}_{A_1}(U_2 P_2) \\
&= \hbar (T_2^{-1} (a^\alpha \otimes b^\beta)_2 U_2 P_2) (T_1 X_{21} T_1^{-1}) \\
&- \hbar (T_1 X_{21} T_1^{-1}) (T_2^{-1} (a^\alpha \otimes b^\beta)_2 U_2 P_2)
\end{align*}
We also find for $\tilde t := \sum_\alpha T^{-1} (a^\alpha \otimes e)$
\begin{align*}
\operatorname{ad}_{A_1} \tilde t_2
&= -T_2^{-1} \operatorname{ad}_{A_1}(T_2) \tilde t_2 + T_2^{-1} \operatorname{ad}_{A_1} \bigg( \sum_\alpha (a^\alpha \otimes e)_2 \bigg) \\
&= -\hbar (T_1 X_{21} T_1^{-1}) \tilde t_2.
\end{align*}
Let $t$ be the unique diagonal matrix fulfilling $\tilde t = t (e \otimes e)$. Then it also follows that
\begin{equation*}
\operatorname{ad}_{A_1} t_2 = -\hbar (T_1 X_{21} T_1^{-1}) t_2,
\end{equation*}
since $(T_1 X_{21} T_1^{-1})$ is diagonal in the second space, and thus
\begin{align*}
\operatorname{ad}_{A_1} t_2^{-1} = \hbar t_2^{-1} (T_1 X_{21} T_1^{-1}).
\end{align*}
All in all, we obtain
\begin{align*}
\operatorname{ad}_{A_1}(t_2^{-1} T_2^{-1} (a^\alpha \otimes b^\beta)_2 U_2 P_2 t_2)
&= \operatorname{ad}_{A_1}(t_2^{-1}) T_2^{-1} (a^\alpha \otimes b^\beta)_2 U_2 P_2 t_2 \\
&+ t_2^{-1} \operatorname{ad}_{A_1}(T_2^{-1} (a^\alpha \otimes b^\beta)_2 U_2 P_2) t_2 \\
&+ t_2^{-1} T_2^{-1} (a^\alpha \otimes b^\beta)_2 U_2 P_2 \operatorname{ad}_{A_1}(t_2) \\
&= 0
\end{align*}
Similarly, we can introduce $\tilde u := \sum_\alpha U^{-1} (a^\alpha \otimes e)$ and $u$. We have thus found $2N + 2N(\ell-1) = 2N\ell$ invariants: Firstly $Q$, secondly $\mathbf{P} := u^{-1} P t$, and lastly $\mathbf{a}^\alpha := t^{-1} T^{-1} a^\alpha$ and $\mathbf{b}^\beta := b^\beta U u$, with each of the $N$ components satisfying one constraint. We started with an algebra with $2N^2+2N\ell$ independent generators, fixed the $N^2$ values of the quantum moment map, and formed the quotient by the action of an algebra with $N^2$ independent generators, giving exactly
\begin{equation*}
(2N^2+2N\ell)-N^2-N^2 = 2N\ell,
\end{equation*}
so we have indeed found all invariants.

Now, let us determine their algebraic relations. Firstly, we note
\begin{align*}
\tilde t_1 \tilde t_2 &= T_1^{-1} \sum_\alpha (a^\alpha \otimes e)_1 T_2^{-1} \sum_\alpha (a^\alpha \otimes e)_2 = R_{12} T_2^{-1} \sum_\alpha (a^\alpha \otimes e)_2 T_1^{-1} \sum_\alpha (a^\alpha \otimes e)_1
= R_{12} \tilde t_2 \tilde t_1,
\end{align*}
which becomes
\begin{equation*}
[t_{ii},t_{jj}] = \frac{\hbar}{q_{ij}} (\tilde t_{jj}-\tilde t_{ij})(\tilde t_{ii}-\tilde t_{ji}) = -\frac{\hbar}{q_{ij}} (t_{ii}-t_{jj})^2,
\end{equation*}
just as in the classical case. We collect
\begin{align*}
[t_1,t_2] = -\hbar r_{12} (t_1-t_2)^2, \quad [u_1,u_2] = \hbar r_{12} (u_1-u_2)^2, \quad [t_1,u_2] = 0.
\end{align*}
Similarly $t_1 (e \otimes e)_1 T_2 = T_2 R_{21} t_1 (e \otimes e)_1$, which means
\begin{align*}
[t_1,T_2] = -\hbar T_2 r_{12} (t_1-t_2)
\end{align*}
or
\begin{align*}
t_1 T_2 = T_2 R_{21} t_1 + \hbar T_2 r_{12} t_2
\end{align*}
hence $[t_1,T_2^{-1}] = \hbar r_{12} (t_1-t_2) T_2^{-1}$ and $[t_1^{-1},T_2^{-1}] = -\hbar t_1^{-1} r_{12} (t_1-t_2) T_2^{-1} t_1^{-1}$ as well as $u_1 (e \otimes e)_1 U_2 = U_2 R_{12} u_1 (e \otimes e)_1$, meaning
\begin{align*}
[u_1,U_2] = \hbar U_2 r_{12} (u_1-u_2)
\end{align*}
or
\begin{align*}
u_1 U_2 = U_2 R_{12} u_1 - \hbar U_2 r_{12} u_2.
\end{align*}
Finally, we have $t_1 (e \otimes e)_1 P_2 = P_2 \bar R_{12}^{-1} t_1 (e \otimes e)_1$, meaning
\begin{align*}
[t_1,P_2] = -\hbar P_2 \bar r_{12} (t_1-t_2), \quad [u_1,P_2] = -\hbar P_2 \bar r_{12} (u_1-u_2)
\end{align*}
or
\begin{align*}
t_1 P_2 = P_2 \bar R_{12}^{-1} t_1 + \hbar P_2 \bar r_{12} t_2, \quad u_1 P_2 = P_2 \bar R_{12}^{-1} u_1 + \hbar P_2 \bar r_{12} u_2.
\end{align*}

Let $\mathbf{W} := t^{-1} T^{-1} U u$. For the quantum moment map, we find
\begin{equation*}
-t^{-1} T^{-1} \mu(A) U u = \mathbf{W} Q - Q \mathbf{W} + \sum_\alpha \mathbf{a}^\alpha \otimes \mathbf{b}^\alpha,
\end{equation*}
which we now fix to equal $\hbar \gamma \mathbf{W}$, since the moment map scales with $\hbar$, as it has units of angular momentum. The solution to this is
\begin{align*}
\mathbf{W}_{ij} = \frac{\sum_\alpha \mathbf{a}_i^\alpha \mathbf{b}_j^\alpha}{q_{ij}+\hbar\gamma}.
\end{align*}
Note that for $\gamma \neq 0$, we can always rescale $Q$ and $\mathbf{a}^\alpha$ such that $\gamma$ equals any fixed non-zero value. This will become important later.

We can now construct an invariant that does not involve any $t$ or $u$:
%$J_1^{\alpha\beta} := \mathbf{b}^\beta \mathbf{P} \mathbf{a}^\alpha = b^\beta U P T^{-1} a^\alpha$, which satisfies
%\begin{align*}
%(b_1^\beta U_1 P_1 T_1^{-1} a_1^\alpha) (b_2^\beta U_2 P_2 T_2^{-1} a_2^\alpha)
%&= b_2^\beta U_2 P_2 T_2^{-1} a_2^\alpha b_1^\beta U_1 P_1 T_1^{-1} a_1^\alpha \\
%&+ b_2^\beta U_2 P_2 T_2^{-1} [b_1^\beta,a_2^\alpha] U_1 P_1 T_1^{-1} a_1^\alpha \\
%&+ b_1^\beta U_1 P_1 T_1^{-1} [a_1^\alpha,b_2^\beta] U_2 P_2 T_2^{-1} a_2^\alpha,
%\end{align*}
%but the more interesting invariants are
\begin{equation*}
\mathbf{S}_i^{\alpha\beta} := \mathbf{b}_i^\alpha \mathbf{P}_i \mathbf{a}_i^\beta = \mathbf{b}^\alpha \mathbf{P} e_{ii} \mathbf{a}^\beta = b^\alpha U P e_{ii} T^{-1} a^\beta.
\end{equation*}
These satisfy
\begin{align}
\mathbf{S}_i^{\alpha\beta} \mathbf{S}_j^{\mu\nu}
&= b_1^\alpha (UP)_1 (e_{ii})_1 T_1^{-1} a_1^\beta b_2^\mu (UP)_2 (e_{jj})_2 T_2^{-1} a_2^\nu \\
&= b_1^\alpha (UP)_1 (e_{ii})_1 T_1^{-1} b_2^\mu a_1^\beta (UP)_2 (e_{jj})_2 T_2^{-1} a_2^\nu \\
&+ b_1^\alpha (UP)_1 (e_{ii})_1 T_1^{-1} [a_1^\beta ,b_2^\mu] (UP)_2 (e_{jj})_2 T_2^{-1} a_2^\nu \\
&= b_2^\mu b_1^\alpha (UP)_2 (UP)_1 \underline{R}_{21} (e_{ii})_1 \bar R_{12}^{-1} (e_{jj})_2 R_{12} T_2^{-1} T_1^{-1} a_2^\nu a_1^\beta \\
&- \hbar \delta^{\beta\mu} b_1^\alpha (UP)_1 (e_{ii})_1 T_1^{-1} (UP)_1 (e_{jj})_1 T_1^{-1} a_1^\nu \\
&= \mathbf{S}_j^{\mu\nu} \mathbf{S}_i^{\alpha\beta}  + \hbar \sum_\rho \mathbf{S}_j^{\mu\rho} \frac{\delta^{\alpha\nu}}{q_{ji}+\hbar\gamma} \mathbf{S}_i^{\rho\beta} - \hbar \sum_\rho \mathbf{S}_i^{\alpha\rho} \frac{\delta^{\beta\mu}}{q_{ij}+\hbar\gamma} \mathbf{S}_j^{\rho\nu} \label{equation:SSfirstline} \\
&+ \frac{\hbar}{q_{ij}+\epsilon\delta_{ij}} \bigg( \mathbf{S}_i^{\mu\beta} \mathbf{S}_j^{\alpha\nu} + \hbar \sum_\rho \mathbf{S}_i^{\mu\rho} \frac{\delta^{\alpha\beta}}{q_{ij}+\hbar\gamma} \mathbf{S}_j^{\rho\nu} \bigg) - (i \leftrightarrow j)
\label{equation:SSsecondline}
\end{align}
due to the identity (checked for $N=2,3,4$)
\begin{align*}
\underline{R}_{21} (e_{ii})_1 \bar R_{12}^{-1} (e_{jj})_2 R_{12}
&= (e_{jj})_2 \bar R_{21}^{-1} (e_{ii})_1 + \frac{\hbar}{q_{ij}+\epsilon\delta_{ij}} (e_{ii})_2 \bar R_{21}^{-1} (e_{jj})_1 C_{12} R_{12} - (i \leftrightarrow j)
\end{align*}
for any $\epsilon$, as well as
\begin{align*}
b_2^\mu b_1^\alpha (UP)_2 (UP)_1 (e_{jj})_2 \bar R_{21}^{-1} (e_{ii})_1 T_2^{-1} T_1^{-1} a_2^\nu a_1^\beta &= \mathbf{S}_j^{\mu\nu} \mathbf{S}_i^{\alpha\beta} + \hbar \sum_\rho \mathbf{S}_j^{\mu\rho} \frac{\delta^{\alpha\nu}}{q_{ji}+\hbar\gamma} \mathbf{S}_i^{\rho\beta}.
\end{align*}
Note that we can rewrite the relation above in a nice, compact form:
\begin{equation*}
R_{ab}(q_{ji}) \mathbf{S}_i^a R_{ab}(q_{ij}+\hbar\gamma) \mathbf{S}_j^b = \mathbf{S}_j^b R_{ab}(q_{ji}+\hbar\gamma) \mathbf{S}_i^a R_{ab}(q_{ij}), \quad R_{ab}(\epsilon) \mathbf{S}_i^a R_{ab}(\hbar\gamma) \mathbf{S}_i^b = \mathbf{S}_i^b R_{ab}(\hbar\gamma) \mathbf{S}_i^a R_{ab}(\epsilon),
\end{equation*}
where $R(z) = 1+\frac{\hbar}{z} P$ and $P$ is the permutation in spin space. We can not easily take another normalization, since it would not commute with the $\mathbf{S}$-operators. Instead, we get
\begin{equation*}
R_{ab}'(q_{ji}) \mathbf{S}_i^a R_{ab}'(q_{ij}+\hbar\gamma) \mathbf{S}_j^b = \frac{f(q_{ji}) f(q_{ij}+\hbar+\hbar\gamma)}{f(q_{ji}+\hbar+\hbar\gamma) f(q_{ij})} \mathbf{S}_j^b R_{ab}'(q_{ji}+\hbar\gamma) \mathbf{S}_i^a R_{ab}'(q_{ij})
\end{equation*}
with $R'(z) = f(z) R(z)$. A solution to the equation
\begin{equation*}
\frac{f(z) f(-z+\eta)}{f(z+\eta) f(-z)} = 1, \quad \eta := \hbar(\gamma+1)
\end{equation*}
is easily found to be $f(z) = \frac{z}{z-\eta}$. This becomes especially nice for $\gamma = -2$, giving $\eta = -\hbar$, which results in the unitary $R$-matrix
\begin{equation*}
R'(z) = \frac{z}{z+\hbar}+\frac{\hbar}{z+\hbar} P.
\end{equation*}

Let us examine skew-symmetry in the formula for the commutator $[\mathbf{S}_i^{\alpha\beta},\mathbf{S}_j^{\mu\nu}]$. The line (\ref{equation:SSfirstline}) gives something manifestly skew-symmetric, but the line (\ref{equation:SSsecondline}) does not. However, we have the identity
\begin{align*}
(e_{jj})_2 \bar R_{21}^{-1} (e_{ii})_1 C_{12} R_{12} + (e_{ii})_2 \bar R_{21}^{-1} (e_{jj})_1 C_{12} R_{12}
&= \underline{R}_{21} C_{12} (e_{ii})_2 \bar R_{21}^{-1} (e_{jj})_1 + \underline{R}_{21} C_{12} (e_{jj})_2 \bar R_{21}^{-1} (e_{ii})_1,
\end{align*}
giving the additional relation
\begin{align*}
&\mathbf{S}_j^{\mu\beta} \mathbf{S}_i^{\alpha\nu} + \hbar \sum_\rho \mathbf{S}_j^{\mu\rho} \frac{\delta^{\alpha\beta}}{q_{ji}+\hbar\gamma} \mathbf{S}_i^{\rho\nu} + \mathbf{S}_i^{\mu\beta} \mathbf{S}_j^{\alpha\nu} + \hbar \sum_\rho \mathbf{S}_i^{\mu\rho} \frac{\delta^{\alpha\beta}}{q_{ij}+\hbar\gamma} \mathbf{S}_j^{\rho\nu} \\
&= \mathbf{S}_i^{\alpha\nu} \mathbf{S}_j^{\mu\beta} + \hbar \sum_\rho \mathbf{S}_i^{\alpha\rho} \frac{\delta^{\mu\nu}}{q_{ij}+\hbar\gamma} \mathbf{S}_j^{\rho\beta} + \mathbf{S}_j^{\alpha\nu} \mathbf{S}_i^{\mu\beta} + \hbar \sum_\rho \mathbf{S}_j^{\alpha\rho} \frac{\delta^{\mu\nu}}{q_{ji}+\hbar\gamma} \mathbf{S}_i^{\rho\beta},
\end{align*}
making skew-symmetry manifest. Compactly, this reads
\begin{align*}
\mathbf{S}_j^b R_{ab}(q_{ji}+\hbar\gamma) \mathbf{S}_i^a + \mathbf{S}_i^b R_{ab}(q_{ij}+\hbar\gamma) \mathbf{S}_j^a = \mathbf{S}_i^a R_{ab}(q_{ij}+\hbar\gamma) \mathbf{S}_j^b + \mathbf{S}_j^a R_{ab}(q_{ji}+\hbar\gamma) \mathbf{S}_i^b.
\end{align*}

%Let us combine this all nicely by defining
%\begin{equation*}
%S(z) := \prod_i (z-q_i) + \sum_i \prod_{i \neq j} \frac{z-q_j}{q_i-q_j} \mathbf{S}_i.
%\end{equation*}
%%\begin{equation*}
%%S(z) = \sum_i \frac{1}{z-q_i} \mathbf{S}_i.
%%\end{equation*}
%Then $\mathbf{S}_i := S(q_i)$ and the relation
%\begin{equation*}
%R_{ab}(w-z) S_a(z) R_{ab}(z-w+\hbar\gamma) S_b(w) = S_b(w) R_{ab}(w-z+\hbar\gamma) S_a(z) R_{ab}(z-w)
%\end{equation*}
%is satisfied at $N^2-N$ points $z=q_i,w=q_j$. When $w \to z = q_i$, we create a pole, but the residues coincide, only leaving the wanted relation for $i=j$ in the zeroth order, giving another $N$ points where the relation is satisfied.

%Before reduction, there are two scales, two different $\hbar$, resulting from rescaling $Q$ and $\mathbf{a}^\alpha$ separately. After reduction, this is reduced to one scale which scales with the moment.

%This allows us to rewrite the relation as
%\begin{equation*}
%R_{ab}(\tfrac{1}{\mu} q_{ji}) \mathbf{S}_i^a R_{ab}(\tfrac{1}{\mu\lambda} q_{ij}+\tfrac{\hbar\gamma}{\lambda}) \mathbf{S}_j^b = \mathbf{S}_j^b R_{ab}(\tfrac{1}{\mu\lambda} q_{ji}+\tfrac{\hbar\gamma}{\lambda}) \mathbf{S}_i^a R_{ab}(\tfrac{1}{\mu} q_{ij}), \quad \mathbf{S}_i^a R_{ab}(\tfrac{\hbar\gamma}{\lambda}) \mathbf{S}_i^b = \mathbf{S}_i^b R_{ab}(\tfrac{\hbar\gamma}{\lambda}) \mathbf{S}_i^a
%\end{equation*}
%For the representation to exist, we have to fix $\frac{\hbar\gamma}{\lambda} = \pm 2\hbar$, so $\lambda = \pm \frac{\gamma}{2}$.

%Indeed, letting
%\begin{equation*}
%E(z) := \prod_i (z-q_i) + \sum_i \prod_{i \neq j} \frac{z-q_j}{q_i-q_j} e_{ii},
%\end{equation*}
%we obtain
%\begin{align*}
%\underline{R}_{21} E_1(z) \bar R_{12}^{-1} E_2(w) R_{12}
%&= E_2(w) \bar R_{21}^{-1} E_1(z) + \frac{\hbar}{z-w}(E_2(w) \bar R_{21}^{-1} E_1(z) C_{12} R_{12} + E_2(z) \bar R_{21}^{-1} E_1(w) C_{12} R_{12}),
%\end{align*}
%which implies this relation.

\subsection{Representation of the reduced algebra}

%The diagram for this relation when acting on $(\C^\ell)^{\otimes N} \otimes \C[y_1,...,y_N]$ should look like two rays coming in from left infinity, reflecting off the $i$th and $j$th strand, meanwhile mutually intersecting twice. The local reflection operator induces a shift of the spectral parameter. This reminds us of the form of the chiral Hamiltonians of long range spin-chains seen in Jules' papers.
%If this is a good picture, then we should take
%\begin{align*}
%\mathbf{S}_i^1 = R_1^1(q_{1i}-\gamma\Sigma_1) \cdots R_{i-1}^1(q_{i-1,i}-\gamma\Sigma_{i-1}) F_i R_{i-1}^1(q_{i,i-1}+\gamma\Sigma_{i-1}) \cdots R_1^1(q_{i1}+\gamma\Sigma_1).
%\end{align*}
%with
%\begin{equation*}
%F_i := e^{-\gamma\partial_i} \pi, \quad \pi = \sum_\alpha e^{\alpha,\alpha+1}, \quad \Sigma_j := \sum_{i=1}^j \sigma_i, \quad \sigma = \frac{1}{2} \sum_\alpha (\ell+1-2\alpha) e^{\alpha\alpha}.
%\end{equation*}
%Note that $\pi \sigma = (\sigma-1)\pi$ and $\sigma$ generalizes $\frac{1}{2} \sigma^z$. In particular, we get
%\begin{align*}
%R_1^2(q_{21}+\gamma\Sigma_1) R^{12}(q_{21}+\gamma) F_1 &= F_1 R_1^2(q_{21}+\gamma\Sigma_1) R^{12}(q_{21}), \\
%R^{12}(q_{12}+\gamma) R_1^2(q_{12}-\gamma\Sigma_1) F_1 &= F_1 R^{12}(q_{12}+\gamma) R_1^2(q_{12}-\gamma\Sigma_1)
%\end{align*}
%For $N=2$, we can then do the following calculation:
%\begin{align*}
%R^{12}(q_{12}) \mathbf{S}_2^2 R^{12}(q_{21}+\gamma) \mathbf{S}_1^1
%&= R^{12}(q_{12}) R_1^2(q_{12}-\gamma\Sigma_1) F_2 R_1^2(q_{21}+\gamma\Sigma_1) R^{12}(q_{21}+\gamma) F_1 \\
%&= R^{12}(q_{12}) R_1^2(q_{12}-\gamma\Sigma_1) F_2 F_1 R_1^2(q_{21}+\gamma\Sigma_1) R^{12}(q_{21}) \\
%&= F_1 R^{12}(q_{12}+\gamma) R_1^2(q_{12}-\gamma\Sigma_1) F_2 R_1^2(q_{21}+\gamma\Sigma_1) R^{12}(q_{21}) \\
%&= \mathbf{S}_1^1 R^{12}(q_{12}+\gamma) \mathbf{S}_2^2 R^{12}(q_{21}).
%\end{align*}
%
%It is a little suspect that we are not doing any Yang-Baxter moves in this calculation, even though the diagrams suggest this. We should look at
%\begin{align*}
%&R^{12}(q_{12}) R_1^2(q_{12}+\gamma+\gamma\sigma_1) \check R_2^2(\gamma+\gamma\sigma_2) R_1^2(q_{21}+\gamma\sigma_1) R^{12}(q_{21}+\gamma) \check R_1^1(\gamma+\gamma\sigma_1) \\
%&= R^{12}(q_{12}) R_1^2(q_{12}+\gamma+\gamma\sigma_1) \check R_2^2(\gamma+\gamma\sigma_2) \check R_1^1(\gamma) R_1^2(q_{21}+\gamma+\gamma\sigma_1) R^{12}(q_{21}) \\
%&= R^{12}(q_{12}) R_1^2(q_{12}+\gamma+\gamma\sigma_1) \check R_1^1(\gamma) \check R_2^2(\gamma+\gamma\sigma_2) R_1^2(q_{21}+\gamma+\gamma\sigma_1) R^{12}(q_{21}) \\
%&= \check R_1^1(\gamma+\gamma\sigma_1) R^{12}(q_{12}+\gamma) R_1^2(q_{12}+\gamma\sigma_1) \check R_2^2(\gamma+\gamma\sigma_2) R_1^2(q_{21}+\gamma+\gamma\sigma_1) R^{12}(q_{21})
%\end{align*}
%Let us define $P_{i,i+1}(x) := \check R_{i,i+1}(x+\gamma\Sigma_{i-1})$ as a rational limit of the deformed Inozemtsev chain. Then this is unitary and fulfills the braid-like QYBE
%\begin{equation*}
%P_{i-1,i}(x-y) P_{i,i+1}(x) P_{i-1,i}(y) = P_{i,i+1}(y) P_{i-1,i}(x) P_{i,i+1}(y).
%\end{equation*}
%Suppose $\mathbf{S}_1^a = P_1^a(\gamma) e^{\gamma \partial_1}$ and $\mathbf{S}_2^a = P_{12}(q_{21}) P_1^a(\gamma) P_{12}(q_{12}+\gamma) e^{\gamma\partial_2} = P_1^a(q_{12}+\gamma) P_{12}(\gamma) P_1^a(q_{21}) e^{\gamma\partial_2}$. Then
%\begin{align*}
%R^{ab}(q_{12}) \mathbf{S}_2^b R^{ab}(q_{21}+\gamma) \mathbf{S}_1^a
%&= R^{ab}(q_{12}) P_1^a(q_{12}+\gamma) P_{12}(\gamma) P_1^a(q_{21}) e^{\gamma\partial_2} R^{ab}(q_{21}+\gamma) R_1^a(\gamma) e^{\gamma \partial_1}
%\end{align*}

%To get closer to their formalism, let's think of the RSRS relation in terms of the total permutation
%\begin{equation*}
%P_{i,i+1}^\text{tot} := s_{i,i+1} \check R_{i,i+1}(q_{i,i+1})
%\end{equation*}
%where $s_{i,i+1}$ permutes polynomial variables. This gives the ansatz
%%\begin{equation*}
%%S_1^a := \check R_{a1}(\eta) e^{-\eta \partial_1}, \quad S_2^a := P_{12}^\text{tot} S_1^a P_{12}^\text{tot} = \check R_{12}(q_{21}) \check R_{a1}(\eta) e^{-\eta \partial_2} \check R_{12}(q_{12}) = \check R_{a1}(q_{12}+\eta) \check R_{12}(\eta) \check R_{a1}(q_{21}) e^{-\eta \partial_2}
%%\end{equation*}
%\begin{equation*}
%S_1^a := \check R_{a1}(\eta) e^{-\eta \partial_1}, \quad S_2^a := P_{12}^\text{tot} S_1^a P_{12}^\text{tot} = \check R_{12}(q_{21}) \check R_{a1}(\eta) e^{-\eta \partial_2} \check R_{12}(q_{12})
%= \check R_{12}(q_{12}+\eta) \check R_{a1}(\eta) \check R_{12}(q_{21}) e^{-\eta \partial_2}
%\end{equation*}
%which indeed fulfills
%\begin{align*}
%R_{ab}(q_{12}) S_2^b R_{ab}(q_{21}+2\eta) S_1^a
%&= \check R_{ab}(q_{12}) S_2^a \check R_{ab}(q_{21}+2\eta) S_1^a \\
%&= \check R_{ab}(q_{12}) \check R_{a1}(q_{12}+\eta) \check R_{12}(\eta) \check R_{a1}(q_{21}) e^{-\eta \partial_2} \check R_{ab}(q_{21}+2\eta) \check R_{a1}(\eta) e^{-\eta\partial_1} \\
%&= \check R_{ab}(q_{12}) \check R_{a1}(q_{12}+\eta) \check R_{12}(\eta) \check R_{a1}(q_{21}) \check R_{ab}(q_{21}+\eta) \check R_{a1}(\eta) e^{-\eta\partial_1} e^{-\eta \partial_2} \\
%&= \check R_{ab}(q_{12}) \check R_{a1}(q_{12}+\eta) \check R_{12}(\eta) \check R_{ab}(\eta) \check R_{a1}(q_{21}+\eta) \check R_{ab}(q_{21}) e^{-\eta\partial_1} e^{-\eta \partial_2} \\
%&= \check R_{ab}(q_{12}) \check R_{a1}(q_{12}+\eta) \check R_{ab}(\eta) \check R_{12}(\eta) \check R_{a1}(q_{21}+\eta) \check R_{ab}(q_{21}) e^{-\eta\partial_1} e^{-\eta \partial_2} \\
%&= \check R_{a1}(\eta) \check R_{ab}(q_{12}+\eta) \check R_{a1}(q_{12}) \check R_{12}(\eta) \check R_{a1}(q_{21}+\eta) \check R_{ab}(q_{21}) e^{-\eta\partial_1} e^{-\eta \partial_2} \\
%&= \check R_{a1}(\eta) e^{-\eta\partial_1} \check R_{ab}(q_{12}+2\eta) \check R_{a1}(q_{12}+\eta) \check R_{12}(\eta) \check R_{a1}(q_{21}) e^{-\eta \partial_2} \check R_{ab}(q_{21}) \\
%&= S_1^a \check R_{ab}(q_{12}+2\eta) S_2^a \check R_{ab}(q_{21}) \\
%&= S_1^a R_{ab}(q_{12}+2\eta) S_2^b R_{ab}(q_{21})
%\end{align*}
%Also recall that the relation for $i=j=1$ reduces to
%\begin{equation*}
%\mathbf{S}_1^a R_{ab}(\hbar\gamma) \mathbf{S}_1^b = \mathbf{S}_1^b R_{ab}(\hbar\gamma) \mathbf{S}_1^a,
%\end{equation*}
%This is satisfied by $\mathbf{S}_1^a := R_{a1}(\tfrac{\hbar\gamma}{2})$. We note that $\check R_{a1}(\frac{\hbar\gamma}{2}) = \pm R_{a1}(\frac{\hbar\gamma}{2})$ when $\frac{\gamma}{2} = \pm 1$. So these values give special solutions that extend to the multiparticle case.

%We have found the relation
%\begin{equation*}
%R_{ab}(w-z) S_a(z) R_{ab}(z-w+\hbar\gamma) S_b(w) = S_b(w) R_{ab}(w-z+\hbar\gamma) S_a(z) R_{ab}(z-w)
%\end{equation*}
%where $R(z) = 1 + \frac{\hbar}{z} P$. Let $S(z) := T^\vee(z) e^{-\gamma\hbar\partial_z} T(z+\hbar\gamma)$, where $T(z),T^\vee(z)$
%\begin{align*}
%T(z) = R_{01}(z-q_1) \cdots R_{0N}(z-q_N), \quad T^\vee(z) = R_{0N}(q_N-z) \cdots R_{01}(q_1-z).
%\end{align*}
%are the monodromy matrices generating the double Yangian, meaning
%\begin{align*}
%R_{ab}(z-w) T_a(z) T_b(w) &= T_b(w) T_a(z) R_{ab}(z-w) \\
%T_a^\vee(z) T_b^\vee(w) R_{ab}(z-w) &= R_{ab}(z-w) T_b^\vee(w) T_a^\vee(z) \\
%T_a(z) R_{ab}(z-w) T_b^\vee(w) &= T_b^\vee(w) R_{ab}(z-w) T_a(z) \\
%e^{\gamma\hbar \sum_{i \neq j} \partial_i} T(z) &= e^{-\gamma\hbar\partial_j}T(z-\hbar\gamma) \\
%e^{\gamma\hbar \sum_i \partial_i} T^\vee(z) &= T^\vee(z-\hbar\gamma)
%\end{align*}
%We can also take the partial monodromy matrix $M_{0[i,j]}(z) = R_{0i}(z-q_i) \cdots R_{0j}(z-q_j)$ and we are allowed to take different normalizations of the $R$-matrix.
%
%Specializing to $N=1$ and $w \to z=q_1$ using $S(z) = T^\vee(z) T(z) e^{-\gamma\hbar\partial_z}$, we obtain $\mathbf{S}_1 = e^{-\gamma\hbar\partial_1}$, which satisfies the relation $\mathbf{S}_1^a R_{ab}(\hbar\gamma) \mathbf{S}_1^b = \mathbf{S}_1^b R_{ab}(\hbar\gamma) \mathbf{S}_1^a$. This is the basic solution and differs from the other ordering $S(z) = T^\vee(z) e^{-\gamma\hbar\partial_z} T(z+\hbar\gamma)$, where we obtain $\mathbf{S}_1 = \check R(\hbar\gamma) e^{-\gamma\hbar\partial_1}$, which does not satisfy this relation! Somehow, during the specialization, something goes wrong.

%Let $\mathbf{S}_1 = F_{01} e^{-\gamma\hbar\partial_1}$. Then we need
%\begin{equation*}
%F_{a1} R_{ab}(\hbar\gamma) F_{b1} = F_{b1} R_{ab}(\hbar\gamma) F_{a1}
%\end{equation*}
%and
%\begin{align*}
%R_{ab}(q_{21}) \mathbf{S}_1^a R_{ab}(q_{12}+\hbar\gamma) \mathbf{S}_2^b
%&= R_{ab}(q_{21}) F_{a1} R_{ab}(q_{12}) R_{12}(q_{21}+\gamma\hbar) F_{b2} R_{12}(q_{12}) e^{-\gamma\hbar\partial_1} e^{-\gamma\hbar\partial_2} \\
%\mathbf{S}_2^b R_{ab}(q_{21}+\hbar\gamma) \mathbf{S}_1^a R_{ab}(q_{12})
%&= R_{12}(q_{21}) F_{02} R_{12}(q_{12}+\hbar\gamma) R_{ab}(q_{21}) F_{01} R_{ab}(q_{12}) e^{-\gamma\hbar\partial_2} e^{-\gamma\hbar\partial_1}
%\end{align*}
%With general $\gamma$, this means $F_{a1} F_{b1} = F_{b1} F_{a1}$ and $F_{a1}^2 = F_{b1}^2$. This means that $F_{a1}^2 = D_1$, so $D_1^2 = (F_{a1}F_{b1})^2 = (F_{b1}F_{a1})^2$. According to Mathematica for $\ell=2$, the only solutions are $F_{a1} = C_1$, where $C$ is a diagonal matrix with entries $\pm c$. This seems like we have to restrict the argument of the $R$-matrix somehow. But this fixes $\gamma$! There should be an additional free parameter that makes this work, because then we get
%\begin{align*}
%\mathbf{S}_1 = \check R(\eta) e^{-\eta\partial_1}, \quad \mathbf{S}_2 = P_{12}^\text{tot} \mathbf{S}_1 P_{12}^\text{tot}
%\end{align*}
%which satisfies the RSRS relation and the $i=j$ relation whenever $\check R(\eta) = \pm R(\eta)$. This gives the (anti-)symmetrizer.

We let $\mathbf{S}_1^a = \check R_{a1}(-\hbar) e^{\hbar\partial_1}$, so
\begin{align*}
\mathbf{S}_2^a = P_{12}^\text{tot} \mathbf{S}_1 P_{12}^\text{tot} = \check R_{12}(q_{21}) \check R_{a1}(-\hbar) e^{\hbar\partial_2} \check R_{12}(q_{12}) = \check R_{a1}(q_{12}-\hbar) \check R_{12}(-\hbar) \check R_{a1}(q_{21}) e^{\hbar\partial_2}
\end{align*}
Note that $\check R(-\hbar) = -R(-\hbar) = -2\Pi^-$. We derive
\begin{equation*}
\mathbf{S}_1^a R_{ab}(-2\hbar) \mathbf{S}_1^b = \mathbf{S}_1^b R_{ab}(-2\hbar) \mathbf{S}_1^a
\end{equation*}
and
\begin{align*}
\mathbf{S}_1^a \check R_{ab}(q_{12}-2\hbar) \mathbf{S}_2^a \check R_{ab}(q_{21})
&= \check R_{a1}(-\hbar) e^{\hbar\partial_1} \check R_{ab}(q_{12}-2\hbar) \check R_{a1}(q_{12}-\hbar) \check R_{12}(-\hbar) \check R_{a1}(q_{21}) e^{\hbar\partial_2} \check R_{ab}(q_{21}) \\
&= \check R_{a1}(-\hbar) \check R_{ab}(q_{12}-\hbar) \check R_{a1}(q_{12}) \check R_{12}(-\hbar) \check R_{a1}(q_{21}-\hbar) \check R_{ab}(q_{21}) e^{\hbar\partial_1} e^{\hbar\partial_2} \\
&= \check R_{ab}(q_{12}) \check R_{a1}(q_{12}-\hbar) \check R_{ab}(-\hbar) \check R_{12}(-\hbar) \check R_{a1}(q_{21}-\hbar) \check R_{ab}(q_{21}) e^{\hbar\partial_1} e^{\hbar\partial_2} \\
&= \check R_{ab}(q_{12}) \check R_{a1}(q_{12}-\hbar) \check R_{12}(-\hbar) \check R_{ab}(-\hbar) \check R_{a1}(q_{21}-\hbar) \check R_{ab}(q_{21}) e^{\hbar\partial_1} e^{\hbar\partial_2} \\
&= \check R_{ab}(q_{12}) \check R_{a1}(q_{12}-\hbar) \check R_{12}(-\hbar) \check R_{a1}(q_{21}) \check R_{ab}(q_{21}-\hbar) \check R_{a1}(-\hbar) e^{\hbar\partial_1} e^{\hbar\partial_2} \\
&= \check R_{ab}(q_{12}) \check R_{a1}(q_{12}-\hbar) \check R_{12}(-\hbar)  e^{\hbar\partial_2} \check R_{a1}(q_{21}-\hbar) \check R_{ab}(q_{21}-2\hbar) \check R_{a1}(-\hbar) e^{\hbar\partial_1} \\
&= \check R_{ab}(q_{12}) \mathbf{S}_2^a \check R_{ab}(q_{21}-2\hbar) \mathbf{S}_1^a.
\end{align*}
This would again fix $\gamma = -2$.

Let's just take a look at this representation. So we take $\mathbf{S}_1^a := \Pi_{a1}^\pm e^{\mp\hbar\partial_1}$, meaning
\begin{align*}
\mathbf{S}_i^a
&= P_{1i}^\text{tot} \mathbf{S}_1^a P_{1i}^\text{tot} \\
&= P_{i,i-1}^\text{tot} \cdots P_{12}^\text{tot} \mathbf{S}_1^a P_{12}^\text{tot} \cdots P_{i,i-1}^\text{tot} \\
&= \check R_{i-1,i}(q_{i,i-1}) \cdots \check R_{12}(q_{i1}) \Pi_{a1}^\pm \check R_{12}(q_{1i}\pm\hbar) \cdots \check R_{i-1,i}(q_{i-1,i}\pm\hbar) e^{\mp\hbar\partial_i} \\
&= \check R_{a1}(q_{1i}\pm\hbar) \check R_{12}(q_{2i}\pm\hbar) \cdots \check R_{i-2,i-1}(q_{i-1,i}\pm\hbar) \Pi_{i-1,i}^\pm \check R_{i-2,i-1}(q_{i,i-1}) \cdots \check R_{12}(q_{i2}) \check R_{a1}(q_{i1}) e^{\mp\hbar\partial_i}
\end{align*}
Then we have
\begin{align*}
\check R_{ab}(q_{ji}) \mathbf{S}_i^a \check R_{ab}(q_{ij}\pm 2\hbar) \mathbf{S}_j^a = \mathbf{S}_j^a \check R_{ab}(q_{ji}\pm 2\hbar) \mathbf{S}_i^a \check R_{ab}(q_{ij})
\end{align*}
since for $j<i$
\begin{align*}
&\check R_{ab}(q_{ji}) \mathbf{S}_i^a \check R_{ab}(q_{ij}\pm 2\hbar) \mathbf{S}_j^a \\
&= \check R_{ab}(q_{ji}) \check R_{a1}(q_{1i}\pm\hbar) \cdots \check R_{i-2,i-1}(q_{i-1,i}\pm\hbar) \Pi_{i-1,i}^\pm \\
&\times \check R_{i-2,i-1}(q_{i,i-1}) \cdots \check R_{a1}(q_{i1}) \check R_{ab}(q_{ij}\pm \hbar) \check R_{a1}(q_{1j}\pm\hbar) \cdots \check R_{j-2,j-1}(q_{j-1,j}\pm\hbar) \\
&\times \Pi_{j-1,j}^\pm \check R_{j-2,j-1}(q_{j,j-1}) \cdots \check R_{a1}(q_{j1}) e^{\mp\hbar\partial_i} e^{\mp\hbar\partial_j} \\
&= \check R_{ab}(q_{ji}) \check R_{a1}(q_{1i}\pm\hbar) \cdots \check R_{i-2,i-1}(q_{i-1,i}\pm\hbar) \Pi_{i-1,i}^\pm \check R_{i-2,i-1}(q_{i,i-1}) \cdots \check R_{j-1,j}(q_{j-1,j}) \\
&\times \check R_{ab}(q_{1j}\pm\hbar) \check R_{a1}(q_{2j}\pm\hbar) \cdots \check R_{j-2,j-1}(q_{ij}\pm \hbar) \cdots \check R_{a1}(q_{i2}) \check R_{ab}(q_{i1}) \\
&\times \Pi_{j-1,j}^\pm \check R_{j-2,j-1}(q_{j,j-1}) \cdots \check R_{a1}(q_{j1}) e^{\mp\hbar\partial_i} e^{\mp\hbar\partial_j}
\end{align*}

%However, we can introduce some parameters into the RSRS relation by rescaling the last term of the moment map by $\lambda$, giving
%\begin{align*}
%[\mathbf{S}_i^{\alpha\beta},\mathbf{S}_j^{\mu\nu}]
%&= \hbar \sum_\rho \mathbf{S}_j^{\mu\rho} \frac{\lambda\delta^{\alpha\nu}}{q_{ji}+\hbar\gamma} \mathbf{S}_i^{\rho\beta} - \hbar \sum_\rho \mathbf{S}_i^{\alpha\rho} \frac{\lambda\delta^{\beta\mu}}{q_{ij}+\hbar\gamma} \mathbf{S}_j^{\rho\nu} \\
%&+ (1-\delta_{ij}) \frac{\hbar}{q_{ij}} \bigg( \mathbf{S}_j^{\mu\beta} \mathbf{S}_i^{\alpha\nu} + \hbar \sum_\rho \mathbf{S}_j^{\mu\rho} \frac{\lambda\delta^{\alpha\beta}}{q_{ji}+\hbar\gamma} \mathbf{S}_i^{\rho\nu} + \mathbf{S}_i^{\mu\beta} \mathbf{S}_j^{\alpha\nu} + \hbar \sum_\rho \mathbf{S}_i^{\mu\rho} \frac{\lambda\delta^{\alpha\beta}}{q_{ij}+\hbar\gamma} \mathbf{S}_j^{\rho\nu} \bigg)
%\end{align*}
%Setting $\lambda = \gamma/2$ results in
%\begin{equation*}
%R_{ab}(q_{ji}) \mathbf{S}_i^a R_{ab}(2q_{ij}/\gamma+2\hbar) \mathbf{S}_j^b = \mathbf{S}_j^b R_{ab}(2q_{ji}/\gamma+2\hbar) \mathbf{S}_i^a R_{ab}(q_{ij}), \quad \mathbf{S}_i^a R_{ab}(2\hbar) \mathbf{S}_i^b = \mathbf{S}_i^b R_{ab}(2\hbar) \mathbf{S}_i^a
%\end{equation*}
%This looks unnatural, but it gives us a good representation.

If we only want to solve the $i \neq j$ relation, then we can take $\mathbf{S}_1^a := \check R_{a1}(\eta\hbar) e^{-\eta\hbar\partial_1}$ and
\begin{equation*}
\mathbf{S}_2^a = P_{12}^\text{tot} \mathbf{S}_1^a P_{12}^\text{tot} = \check R_{12}(q_{21}) \check R_{a1}(\eta\hbar) e^{-\eta\hbar\partial_2} \check R_{12}(q_{12}) = \check R_{a1}(q_{12}+\eta\hbar) \check R_{12}(\eta\hbar) \check R_{a1}(q_{21}) e^{-\eta\hbar\partial_2}.
\end{equation*}
Then
\begin{align*}
&\mathbf{S}_1^a \check R_{ab}(q_{12}+2\eta\hbar) \mathbf{S}_2^a \check R_{ab}(q_{21}) \\
&= \check R_{a1}(\eta\hbar) e^{-\eta\hbar\partial_1} \check R_{ab}(q_{12}+2\eta\hbar) \check R_{a1}(q_{12}+\eta\hbar) \check R_{12}(\eta\hbar) \check R_{a1}(q_{21}) e^{-\eta\hbar\partial_2} \check R_{ab}(q_{21}) \\
&= \check R_{a1}(\eta\hbar) \check R_{ab}(q_{12}+\eta\hbar) \check R_{a1}(q_{12}) \check R_{12}(\eta\hbar) \check R_{a1}(q_{21}+\eta\hbar) \check R_{ab}(q_{21}) e^{-\eta\hbar\partial_1} e^{-\eta\hbar\partial_2} \\
&= \check R_{ab}(q_{12}) \check R_{a1}(q_{12}+\eta\hbar) \check R_{ab}(\eta\hbar) \check R_{12}(\eta\hbar) \check R_{a1}(q_{21}+\eta\hbar) \check R_{ab}(q_{21}) e^{-\eta\hbar\partial_1} e^{-\eta\hbar\partial_2} \\
&= \check R_{ab}(q_{12}) \check R_{a1}(q_{12}+\eta\hbar) \check R_{12}(\eta\hbar) \check R_{ab}(\eta\hbar) \check R_{a1}(q_{21}+\eta\hbar) \check R_{ab}(q_{21}) e^{-\eta\hbar\partial_1} e^{-\eta\hbar\partial_2} \\
&= \check R_{ab}(q_{12}) \check R_{a1}(q_{12}+\eta\hbar) \check R_{12}(\eta\hbar) \check R_{a1}(q_{21}) e^{-\eta\hbar\partial_2} \check R_{ab}(q_{21}+2\eta\hbar) \check R_{a1}(\eta\hbar) e^{-\eta\hbar\partial_1} \\
&= \check R_{ab}(q_{12}) \mathbf{S}_2^a \check R_{ab}(q_{21}+2\eta\hbar) \mathbf{S}_1^a
\end{align*}

Recall that for the loop Yangian with $N=1$ we have $J_1^{\alpha\beta} = e^{\alpha\beta} e^{\hbar\partial_1}$, meaning $\mathbf{S}_1^a = P_{a1} e^{\hbar\partial_1}$. This should satisfy the $i=j$ relation that we want, but it rather satisfies $\mathbf{S}_1^a P_{ab} \mathbf{S}_1^b = \mathbf{S}_1^b P_{ab} \mathbf{S}_1^a$. However, representations of the loop algebra can always be twisted by subtracting the central element $\sum_\alpha J_n^{\alpha\alpha}$. This gives us
\begin{equation*}
J_1^a - \sum_\alpha J_1^{\alpha\alpha} = (P_{a1}-1) e^{\hbar\partial_1} = \check R(-\hbar) e^{\hbar\partial_1}.
\end{equation*}
But this is not an automorphism of the whole loop Yangian, unless a variation of the $\beta$-parameter deforms the loop algebra in this way.

The representation where $\mathbf{S}_i^a = V_i P_{ia}^t V_i^\#$ is highly degenerate, making the LHS and RHS be equal to zero.

%\subsection{New representation ansatz}
%
%If we take the equation
%\begin{equation*}
%\mathbf{S}_1^a R_{ab}(\hbar\gamma) \mathbf{S}_1^b = \mathbf{S}_1^b R_{ab}(\hbar\gamma) \mathbf{S}_1^a
%\end{equation*}
%and make the ansatz
%\begin{equation*}
%\mathbf{S}_1^a = V_1 P_{1a}^t W_1, \quad (\mathbf{S}_1)_{ij}^{\alpha\beta} = v_i^\alpha w_j^\beta,
%\end{equation*}
%then we find satisfaction when $\gamma = -\ell$ and $W = V^\#$, the adjugate. Indeed, we have
%\begin{equation*}
%P_{1a}^t R_{ab}(-\ell\hbar) P_{1b}^t = P_{1b}^t R_{ab}(-\ell\hbar) P_{1a}^t.
%\end{equation*}
%Furthermore, the relation
%\begin{equation*}
%\mathbf{S}_1^a R_{ab}(-\ell\hbar) \mathbf{S}_1^b = \mathbf{S}_1^b R_{ab}(-\ell\hbar) \mathbf{S}_1^a
%\end{equation*}
%holds also for $\mathbf{S}_2^a := R_{12}(q_{12}) P_{1a}^t R_{12}(q_{21})$ and $\mathbf{S}_2^a := \check R_{12}(q_{12}) P_{1a}^t \check R_{12}(q_{21})$ In the second case, we have
%\begin{equation*}
%(\mathbf{S}_2^a)^{t_a} = \check R(q_{12}) P_{a1} \check R_{12}(q_{21}) = \check R_{a1}(q_{21}) P_{12} \check R_{a1}(q_{12}).
%\end{equation*}
%
%Now additionally, we want the relation
%\begin{equation*}
%R_{ab}(q_{21}) \mathbf{S}_1^a R_{ab}(q_{12}+\hbar\gamma) \mathbf{S}_2^b = \mathbf{S}_2^b R_{ab}(q_{21}+\hbar\gamma) \mathbf{S}_1^a R_{ab}(q_{12})
%\end{equation*}
%We can apply the transposition $t_a \circ t_b$ here and obtain
%\begin{equation*}
%(\mathbf{S}_1^a)^{t_a} R_{ab}(q_{21}) (\mathbf{S}_2^b)^{t_b} R_{ab}(q_{12}+\hbar\gamma) = R_{ab}(q_{21}+\hbar\gamma) (\mathbf{S}_2^b)^{t_b} R_{ab}(q_{12}) (\mathbf{S}_1^a)^{t_a},
%\end{equation*}
%which should be taken as a relation in the opposite algebra. We rewrite this as
%\begin{equation*}
%(\mathbf{S}_1^a)^{t_a} \check R_{ab}(q_{21}) (\mathbf{S}_2^a)^{t_a} \check R_{ab}(q_{12}+\hbar\gamma) = \check R_{ab}(q_{21}+\hbar\gamma) (\mathbf{S}_2^a)^{t_a} \check R_{ab}(q_{12}) (\mathbf{S}_1^a)^{t_a},
%\end{equation*}
%which after transposing and setting $\gamma = -\ell$ gives
%\begin{equation*}
%\check R_{ab}^t(q_{12}-\ell\hbar) \mathbf{S}_2^a \check R_{ab}^t(q_{21}) \mathbf{S}_1^a = \mathbf{S}_1^a \check R_{ab}^t(q_{12}) \mathbf{S}_2^a \check R_{ab}^t(q_{21}-\ell\hbar).
%\end{equation*}
%The shift by $-\ell\hbar$ can be eliminated using the inverse $R^t(z)^{-1} = R^t(-z-\ell\hbar)$:
%\begin{equation*}
%\mathbf{S}_2^a \check R_{ab}^t(q_{21}) \mathbf{S}_1^a \check R_{ab}^t(q_{12}) = \check R_{ab}^t(q_{21}) \mathbf{S}_1^a \check R_{ab}^t(q_{12}) \mathbf{S}_2^a.
%\end{equation*}
%We transpose back and arrive at the final relation
%\begin{equation*}
%\check R_{ab}(q_{12}) (\mathbf{S}_1^a)^{t_a} \check R_{ab}(q_{21}) (\mathbf{S}_2^a)^{t_a} = (\mathbf{S}_2^a)^{t_a} \check R_{ab}(q_{12}) (\mathbf{S}_1^a)^{t_a} \check R_{ab}(q_{21}).
%\end{equation*}
%Indeed, we find
%\begin{align*}
%(\mathbf{S}_1^a)^{t_a} \check R_{ab}(q_{21}) (\mathbf{S}_2^a)^{t_a} \check R_{ab}(q_{12})
%&= P_{a1} \check R_{ab}(q_{21}) \check R_{a1}(q_{21}) P_{12} \check R_{a1}(q_{12}) \check R_{ab}(q_{12}) \\
%&= \check R_{ab}(q_{21}) \check R_{a1}(q_{21}) P_{12} \check R_{a1}(q_{12}) \check R_{ab}(q_{12}) P_{a1} \\
%&= \check R_{ab}(q_{21}) (\mathbf{S}_2^a)^{t_a} \check R_{ab}(q_{12}) (\mathbf{S}_1^a)^{t_a}.
%\end{align*}
%This suggests that in the diagrams we are supposed to take the lines of index $a$ and $b$ to be dual defining representations.
%
%The local reflection operator should look like the auxiliary space looping around the quantum space with a twist. We have the pseudo-braidings $c_{V_q^\pm,V_q^\pm} = P$ and $c_{V_{q+\ell\hbar}^-,V_q^+} = P^t$, and the loop is given by
%\begin{equation*}
%\kappa_{V_q^+,V_q^+} = (\id_{V_q^+} \otimes \operatorname{ev}_{V_{q+2\ell\hbar}^+,V_{q+\ell\hbar}^-} \otimes \id_{V_q^+}) (c_{V_{q+2\ell\hbar}^+,V_q^+} \otimes c_{V_{q+\ell\hbar}^-,V_q^+}^2) (\id_{V_{q+2\ell\hbar}^+} \otimes \operatorname{coev}_{V_q^+,V_{q+\ell\hbar}^-} \otimes \id_{V_q^+})
%\end{equation*}
%This is nothing but $\check R(2\ell\hbar)$ or $\check R(\ell\hbar)$, depending on convention. The double braiding is given by $c_{V_{q+\ell\hbar}^-,V_q^+}^2 = \ell P^t$. But this does not provide any shift. Only the twist provides a shift.
%
%There is a solution $\mathbf{S}_1^a = V_1 P_{a1} V_1^\#, \mathbf{S}_2^a = V_2 P_{a2} V_2^\#$ for $\ell=2$ for arbitrary $z,\hbar,\eta$ with $V_1,V_2$ satisfying $v_{22} = v_{12} v_{21}/v_{11}$, so that the determinant vanishes. The same seems to hold for $\ell=3$. But this makes both LHS and RHS vanish, which is too degenerate to be useful. We would want to focus on the $P^t,\gamma=-\ell$ case, but this also makes LHS and RHS vanish.
%
%The twist of one strand labeled by $z$ around a strand labeled by $q$ is given by
%\begin{equation*}
%K(z,q) := \ell+1/\ell - \frac{\hbar^2 \ell}{(z-q)^2} (1 + \frac{1}{\ell} P)
%\end{equation*}
%while the simple twist on one strand is given by the scalar $\ell+1$.
%
%In the unitary convention, it becomes
%\begin{align*}
%K(z,q) = \frac{(\ell^2+1) (z-q)^2-\ell^2 \hbar^2}{(\ell+1)(z-q+\hbar)(z-q-\hbar)} + \frac{\ell \hbar^2}{(\ell+1)(z-q+\hbar)(z-q-\hbar)} P.
%\end{align*}
%the simple one strand twist is the scalar $\frac{\ell^2+1}{\ell+1}$.

%\subsection{Another attempt at a representation}
%
%Looking at the diagrams, we expect that the local reflection operator $K$ satisfies the following equation:
%\begin{equation*}
%K_{12} \check R_{a1}(q_{12}) \check R_{12}(q_{12}) = \check R_{a1}(q_{12}-\hbar\gamma) \check R_{12}(q_{12}-\hbar\gamma) K_{a1}
%\end{equation*}
%A possible solution is $K = \check R(0) e^{-\hbar\gamma\partial_1}$. In that case we can write
%\begin{align*}
%&\check R_{ab}(q_{21}) \check R_{a1}(q_{21}) K_{12} \check R_{a1}(q_{12}-\hbar\gamma) \check R_{ab}(q_{12}-\hbar\gamma) K_{a1} \\
%&= \check R_{ab}(q_{21}) \check R_{a1}(q_{21}) K_{12} K_{ab} \check R_{a1}(q_{12}) \check R_{ab}(q_{12}) \\
%&= \check R_{ab}(q_{21}) \check R_{a1}(q_{21}) K_{ab} K_{12} \check R_{a1}(q_{12}) \check R_{ab}(q_{12}) \\
%&= K_{a1} \check R_{ab}(q_{21}-\hbar\gamma) \check R_{a1}(q_{21}-\hbar\gamma) K_{12} \check R_{a1}(q_{12}) \check R_{ab}(q_{12})
%\end{align*}
%But here we encounter the problem that there is an unwanted shift by $\hbar\gamma$ left over. This occurs because $\mathbf{S}_i^{\alpha\beta}$ does not commute with $\mathbf{P}_j$.

\subsection{Commutation relation with momentum}

Let us compute the commutation relation between $\mathbf{P}_i$ and $\mathbf{S}_j^{\alpha\beta}$.
\begin{align*}
\mathbf{P}_i \mathbf{S}_j^{\alpha\beta}
&= u_1^{-1} (e_{ii})_1 P_1 t_1 b_2^\alpha U_2 P_2 (e_{jj})_2 T_2^{-1} a_2^\beta \\
&= \mathbf{S}_j^{\alpha\beta} \mathbf{P}_i - \hbar b_2^\alpha u_1^{-1} U_2 r_{12} (u_1-u_2) u_1^{-1} (e_{jj})_2 T_2^{-1} a_2^\beta (e_{ii})_1 P_1 t_1 \\
&+ \hbar b_2^\alpha u_1^{-1} P_2 \bar r_{12}(u_1-u_2) u_1^{-1} (e_{jj})_2 T_2^{-1} a_2^\beta (e_{ii})_1 P_1 t_1 \\
&+ \hbar b_2^\alpha u_1^{-1} U_2 P_2 (e_{ii})_1 P_1 \bar R_{21}^{-1} (e_{jj})_2 r_{12} (t_1-t_2) T_2^{-1} a_2^\beta \\
&- \hbar b_2^\alpha u_1^{-1} U_2 (e_{ii})_1 P_1 \bar R_{21}^{-1} P_2 \bar r_{12} (t_1-t_2) (e_{jj})_2 T_2^{-1} a_2^\beta
\end{align*}
where we have assumed $(e_{ii})_1 P_1 \bar R_{21}^{-1} P_2 (e_{jj})_2 = P_2 (e_{jj})_2 \bar R_{21}^{-1} (e_{ii})_1 P_1$, but there are correction terms of order $\hbar$. This becomes horribly complicated and I don't see easy simplifications. However, let us introduce the operator
\begin{equation*}
\Pi(z) := e^t \mathbf{P} (z-Q)^{-1} e = e^t u^{-1} P (z-Q)^{-1} t e.
\end{equation*}
Recall that $te = \sum_\alpha T^{-1} a^\alpha$ and $ue = \sum_\alpha U^{-1} a^\alpha$. Then
\begin{align*}
\Pi(z) \mathbf{S}^{\alpha\beta}(w)
&= \sum_\rho e_1^t u_1^{-1} P_1 (z-Q_1)^{-1} T_1^{-1} a_1^\rho b_2^\alpha U_2 P_2 (w-Q_2)^{-1} T_2^{-1} a_2^\beta \\
&= \sum_\rho b_2^\alpha e_1^t U_2 u_1^{-1} R_{21} \bar R_{21}^{-1} P_2 P_1 (z-Q_1+\hbar X)^{-1} \bar R_{12}^{-1} (w-Q_2)^{-1} R_{12} T_2^{-1} T_1^{-1} a_2^\beta a_1^\rho \\
&+ \hbar \sum_\rho b_2^\alpha e_1^t u_1^{-1} U_2 r_{12} u_2 u_1^{-1} R_{21} \bar R_{21}^{-1} P_2 P_1 (z-Q_1+\hbar X)^{-1} \bar R_{12}^{-1} (w-Q_2)^{-1} R_{12} T_2^{-1} T_1^{-1} a_2^\beta a_1^\rho \\
&- \hbar e^t \mathbf{P} (z-Q)^{-1} \mathbf{W} \mathbf{P} (w-Q)^{-1} \mathbf{a}^\beta
\end{align*}
where we used for example
\begin{align*}
u_1^{-1} P_1 U_2 P_2
&= u_1^{-1} U_2 P_1 \bar R_{21}^{-1} P_2 \\
&= U_2 u_1^{-1} R_{21} P_1 \bar R_{21}^{-1} P_2 \\
&+\hbar u_1^{-1} U_2 r_{12} u_2 u_1^{-1} R_{21} P_1 \bar R_{21}^{-1} P_2 \\
&= U_2 u_1^{-1} P_2 \bar R_{12}^{-1} P_1 \bar R_{21}^{-1} R_{21} \bar R_{12} \\
&+\hbar u_1^{-1} U_2 r_{12} u_2 u_1^{-1} P_2 \bar R_{12}^{-1} P_1 \bar R_{21}^{-1} R_{21} \bar R_{12}
\end{align*}
\begin{align*}
u_1^{-1} U_2
&= U_2 u_1^{-1} R_{21} + \hbar u_1^{-1} U_2 r_{12} u_2 u_1^{-1} R_{21} \\
&= (U_2 u_1^{-1} + \hbar (U_2 u_1^{-1} + \hbar u_1^{-1} U_2 r_{12} u_2 u_1^{-1}) r_{12} u_2 u_1^{-1}) R_{21} \\
&= \sum_{i=1}^n U_2 u_1^{-1} (\hbar r_{12} u_2 u_1^{-1})^i R_{21} + u_1^{-1} U_2 (\hbar r_{12} u_2 u_1^{-1})^{n+1} R_{21} \\
&= U_2 u_1^{-1} R_{21} + \hbar U_2 u_1^{-1} r_{12} u_2 u_1^{-1} R_{21}
\end{align*}

When it comes to the loop Yangian, we know exactly what generates the Yangian and the loop algebra. The missing generators are the $X_0^\pm(z)$, which respectively should represent loop level $\pm 1$ and matrix entry $e_{1\ell}$ for $-$ and $e_{\ell 1}$ for $+$. Hence, a natural guess for $X^+(z)$ would be $\mathbf{S}^{\ell 1}(z)$.

\subsection{Commutation relations with spectral parameter}

Define $\mathbf{S}^{\alpha\beta}(z) := \mathbf{b}^\alpha \mathbf{P} (z-Q)^{-1} \mathbf{a}^\beta = b^\alpha U P (z-Q)^{-1} T^{-1} a^\beta$. Then $\mathbf{S}_i^{\alpha\beta} = \operatorname{Res}_{z=q_i}^L \mathbf{S}^{\alpha\beta}(z)$.
%\begin{align*}
%\mathbf{S}^{\alpha\beta}(z) \mathbf{S}^{\mu\nu}(w)
%&= b_1^\alpha U_1 P_1 (z-Q_1)^{-1} T_1^{-1} a_1^\beta b_2^\mu U_2 P_2 (w-Q_2)^{-1} T_2^{-1} a_2^\nu \\
%&= b_2^\mu (UP)_2 b_1^\alpha (UP)_1 \underline{R}_{21} (z-Q_1+\hbar X)^{-1} \bar R_{12}^{-1} (w-Q_2)^{-1} R_{12} T_2^{-1} a_2^\nu T_1^{-1} a_1^\beta \\
%&-\hbar \delta^{\mu\beta} \sum_{ij} \sum_\rho \mathbf{b}_i^\alpha \mathbf{P}_i \mathbf{a}_i^\rho (z-q_i)^{-1} (q_i-q_j+\hbar\gamma)^{-1} (w-q_j-\hbar)^{-1} \mathbf{b}_j^\rho \mathbf{P}_j \mathbf{a}_j^\nu \\
%&= \mathbf{S}^{\mu\nu}(w) \mathbf{S}^{\alpha\beta}(z) \\
%&+ \hbar \delta^{\alpha\nu} \sum_{ij} \sum_\rho \mathbf{b}_i^\mu \mathbf{P}_i \mathbf{a}_i^\rho (w-q_i)^{-1} (q_i-q_j+\hbar\gamma)^{-1} (z-q_j-\hbar)^{-1} \mathbf{b}_j^\rho \mathbf{P}_j \mathbf{a}_j^\beta \\% b^\mu UP (w-Q)^{-1} T^{-1} UP (z-Q)^{-1} T^{-1} a^\beta \\
%&- \hbar (z-w) \sum_{ij} \mathbf{b}_i^\mu \mathbf{P}_i \mathbf{a}_i^\nu (z-q_i)^{-1} (w-q_i)^{-1} (z-q_j-\hbar)^{-1} (w-q_j-\hbar)^{-1}  \mathbf{b}_j^\alpha \mathbf{P}_j \mathbf{a}_j^\beta \\
%&+ 2\hbar (z-w) \sum_i \mathbf{b}_i^\mu \mathbf{P}_i \mathbf{a}_i^\nu (z-q_i)^{-1} (w-q_i)^{-1} (z-q_i-\hbar)^{-1} (w-q_i-\hbar)^{-1} \mathbf{b}_i^\alpha \mathbf{P}_i \mathbf{a}_i^\beta \\
%&+ 2\hbar \delta^{\alpha\nu} (z-w) \sum_i \sum_\rho \mathbf{b}_i^\mu \mathbf{P}_i \mathbf{a}_i^\rho (z-q_i)^{-1} (w-q_i)^{-1} (\hbar\gamma)^{-1} (z-q_i-\hbar)^{-1} (w-q_i-\hbar)^{-1} \mathbf{b}_i^\rho \mathbf{P}_i \mathbf{a}_i^\beta \\
%&-\hbar \delta^{\mu\beta} \sum_{ij} \sum_\rho \mathbf{b}_i^\alpha \mathbf{P}_i \mathbf{a}_i^\rho (z-q_i)^{-1} (q_i-q_j+\hbar\gamma)^{-1} (w-q_j-\hbar)^{-1} \mathbf{b}_j^\rho \mathbf{P}_j \mathbf{a}_j^\nu,
%\end{align*}
%\begin{align*}
%P_2 Q_1 = (Q_1 - \hbar \sum_i e_{ii} \otimes e_{ii}) P_2.
%\end{align*}
%\begin{align*}
%&[\mathbf{S}^{\alpha\beta}(z),\mathbf{S}^{\mu\nu}(w)] \\
%&=\hbar \delta^{\alpha\nu} \sum_{ij} \sum_\rho \mathbf{b}_i^\mu \mathbf{P}_i \mathbf{a}_i^\rho (z-q_i)^{-1} (q_i-q_j+\hbar\gamma)^{-1} \mathbf{b}_j^\rho \mathbf{P}_j (w-q_j)^{-1} \mathbf{a}_j^\beta \\
%&-\hbar (z-w) \sum_{ij} (1-2\delta_{ij}) \mathbf{b}_i^\alpha \mathbf{P}_i (z-q_i)^{-1} (w-q_i)^{-1} \mathbf{a}_i^\nu \mathbf{b}_j^\mu \mathbf{P}_j (z-q_j)^{-1} (w-q_j)^{-1} \mathbf{a}_j^\beta \\
%&-\hbar^2 \delta^{\mu\nu} (z-w) \sum_{ij} (1-2\delta_{ij}) \sum_\rho \mathbf{b}_i^\alpha \mathbf{P}_i (z-q_i)^{-1} (w-q_i)^{-1} \mathbf{a}_i^\rho (q_i-q_j+\hbar\gamma)^{-1} \mathbf{b}_j^\rho \mathbf{P}_j (z-q_j)^{-1} (w-q_j)^{-1} \mathbf{a}_j^\beta \\
%%&+\hbar (z-w) \sum_i \mathbf{b}_i^\alpha \mathbf{P}_i (z-q_i)^{-1} (w-q_i)^{-1} \mathbf{a}_i^\nu \mathbf{b}_i^\mu \mathbf{P}_i (z-q_i)^{-1} (w-q_i)^{-1} \mathbf{a}_i^\beta \\
%%&+\hbar^2 \delta^{\mu\nu} (z-w) \sum_i \sum_\rho \mathbf{b}_i^\alpha \mathbf{P}_i (z-q_i)^{-1} (w-q_i)^{-1} \mathbf{a}_i^\rho (\hbar\gamma)^{-1} \mathbf{b}_i^\rho \mathbf{P}_i (z-q_i)^{-1} (w-q_i)^{-1} \mathbf{a}_i^\beta \\
%&-\hbar \delta^{\mu\beta} \sum_{ij} \sum_\rho \mathbf{b}_i^\alpha \mathbf{P}_i \mathbf{a}_i^\rho (z-q_i)^{-1} (q_i-q_j+\hbar\gamma)^{-1} \mathbf{b}_j^\rho \mathbf{P}_j (w-q_j)^{-1} \mathbf{a}_j^\nu
%\end{align*}
%where $X = \sum_i e_{ii} \otimes e_{ii}$.
We have the identity (checked for $N=2,3,4$)
\begin{align*}
&\underline{R}_{21} (z-Q_1+\hbar X)^{-1} \bar R_{12}^{-1} (w-Q_2)^{-1} R_{12} \\
&= (w-Q_2+\hbar X)^{-1} \bar R_{21}^{-1} (z-Q_1)^{-1} \\
&+ \hbar (w-Q_2+\hbar X)^{-1} (Q_1-Q_2-\hbar X)^{-1} \bar R_{21}^{-1} (z-Q_1)^{-1} C_{12} R_{12} - (z \leftrightarrow w)
\end{align*}
We thus obtain
\begin{align*}
\mathbf{S}^{\alpha\beta}(z) \mathbf{S}^{\mu\nu}(w)
&= b_1^\alpha U_1 P_1 (z-Q_1)^{-1} T_1^{-1} a_1^\beta b_2^\mu U_2 P_2 (w-Q_2)^{-1} T_2^{-1} a_2^\nu \\
&= b_2^\mu b_1^\alpha (UP)_2 (UP)_1 \underline{R}_{21} (z-Q_1+\hbar X)^{-1} \bar R_{12}^{-1} (w-Q_2)^{-1} R_{12} T_2^{-1} T_1^{-1} a_2^\nu a_1^\beta \\
&- \hbar \delta^{\mu\beta} \mathbf{b}^\alpha \mathbf{P} (z-Q)^{-1} \mathbf{W} \mathbf{P} (w-Q)^{-1} \mathbf{a}^\nu \\
&= \mathbf{S}^{\mu\nu}(w) \mathbf{S}^{\alpha\beta}(z) \\
&+ \hbar \delta^{\alpha\nu} \sum_{ij} \sum_\rho \mathbf{b}_i^\mu \mathbf{P}_i (w-q_i)^{-1} \mathbf{a}_i^\rho (q_{ij}+\gamma\hbar)^{-1} \mathbf{b}_j^\rho \mathbf{P}_j (z-q_j)^{-1} \mathbf{a}_j^\beta \\
&+ \hbar \sum_{ij} (q_{ij}-\hbar \delta_{ij})^{-1} \mathbf{b}_i^\mu \mathbf{P}_i (z-q_i)^{-1} \mathbf{a}_i^\beta \mathbf{b}_j^\alpha \mathbf{P}_j (w-q_j)^{-1} \mathbf{a}_j^\nu - (z \leftrightarrow w) \\
&+ \hbar^2 \delta^{\alpha\beta} \sum_{ij} \sum_\rho (q_{ij}-\hbar \delta_{ij})^{-1} \mathbf{b}_i^\mu \mathbf{P}_i (z-q_i)^{-1} \mathbf{a}_i^\rho (q_{ij}+\gamma\hbar)^{-1} \mathbf{b}_j^\rho \mathbf{P}_j (w-q_j)^{-1} \mathbf{a}_j^\nu - (z \leftrightarrow w) \\
&- \hbar \delta^{\mu\beta} \sum_{ij} \sum_\rho \mathbf{b}_i^\alpha \mathbf{P}_i (z-q_i)^{-1} \mathbf{a}_i^\rho (q_{ij}+\gamma\hbar)^{-1} \mathbf{b}_j^\rho \mathbf{P}_j (w-q_j)^{-1} \mathbf{a}_j^\nu
\end{align*}
%This seems like the diagonal relation is deformed to
%\begin{align*}
%\mathbf{S}_i^{\alpha\beta} \mathbf{S}_i^{\mu\nu}
%&= \mathbf{S}_i^{\mu\nu} \mathbf{S}_i^{\alpha\beta} \\
%&+ \hbar \delta^{\alpha\nu} \sum_\rho \mathbf{S}_i^{\mu\rho} (q_{ij}+\gamma\hbar)^{-1} \mathbf{S}_i^{\rho\beta} \\
%&- \mathbf{S}_i^{\mu\beta} \mathbf{S}_i^{\alpha\nu} 
%\end{align*}

We are allowed to write
\begin{align*}
\frac{1}{q_{ij}+\gamma\hbar} = \frac{1}{2\pi\I}\oint_C \frac{1}{\tau-q_i} \frac{1}{\tau-q_j+\gamma\hbar} d\tau = \frac{1}{2\pi\I} \oint_{C'} \frac{1}{\tau-q_i} \frac{1}{\tau-q_j+\gamma\hbar} d\tau.
\end{align*}
where we let $C$ be the contour that moves counterclockwise around the $q_i$ only and $C'$ be the contour that moves clockwise around the $q_i-\gamma\hbar$ only. This separates the sums over $i$ and $j$. Additionally we have the partial fraction decomposition
\begin{align*}
&\frac{1}{w-q_i} \frac{1}{\tau-q_i} \frac{1}{\tau-q_j+\gamma\hbar} \frac{1}{z-q_j-\hbar} \\
&= \bigg( \frac{1}{w-q_i} - \frac{1}{\tau-q_i} \bigg) \frac{1}{\tau-w} \frac{1}{\tau-z+(\gamma+1)\hbar} \bigg( \frac{1}{z-q_j-\hbar} - \frac{1}{\tau-q_j+\gamma\hbar} \bigg).
\end{align*}
This allows us to write
\begin{align*}
&\sum_\rho \sum_{ij} \mathbf{b}_i^\mu \mathbf{P}_i (w-q_i)^{-1} \mathbf{a}_i^\rho (q_{ij}+\gamma\hbar)^{-1} \mathbf{b}_j^\rho \mathbf{P}_j (z-q_j)^{-1} \mathbf{a}_j^\beta \\
&= \sum_\rho \sum_{ij} \mathbf{b}_i^\mu \mathbf{P}_i (w-q_i)^{-1} \mathbf{a}_i^\rho \frac{1}{2\pi\I} \oint_C \frac{1}{\tau-q_i} \frac{1}{\tau-q_j+\gamma\hbar} \mathbf{b}_j^\rho \mathbf{P}_j (z-q_j)^{-1} \mathbf{a}_j^\beta d\tau \\
&= \frac{1}{2\pi\I} \oint_C \sum_\rho \sum_{ij} \mathbf{b}_i^\mu \mathbf{P}_i \mathbf{a}_i^\rho \frac{1}{w-q_i} \frac{1}{\tau-q_i} \frac{1}{\tau-q_j+\gamma\hbar} \frac{1}{z-q_j-\hbar} \mathbf{b}_j^\rho \mathbf{P}_j \mathbf{a}_j^\beta d\tau \\
&= \frac{1}{2\pi\I} \oint_C \sum_\rho (\mathbf{S}^{\mu\rho}(w)-\mathbf{S}^{\mu\rho}(\tau)) \frac{1}{\tau-w} \frac{1}{\tau-z+(\gamma+1)\hbar} (\mathbf{S}^{\rho\beta}(z) - \mathbf{S}^{\rho\beta}(\tau+(\gamma+1)\hbar) d\tau
\end{align*}
But we could have also used the contour $C'$.

We can write in matrix form
\begin{align*}
\mathbf{W} = \frac{1}{2\pi\I} \oint_C \sum_\rho (\tau-Q)^{-1} \mathbf{a}^\rho \mathbf{b}^\rho (\tau-Q+\gamma\hbar)^{-1},
\end{align*}
where in the middle we have a column vector multiplying a row vector, giving an outer product. Similarly, we can write
\begin{align*}
(Q_1-Q_2-\hbar X)^{-1} = \lim_{\epsilon\to 0} \frac{1}{2\pi\I} \oint_C (\tau - Q_2 - \hbar X + \epsilon)^{-1} (\tau-Q_1)^{-1} d\tau
\end{align*}

Let us repeat the computation with these tricks:
{\small
\begin{align*}
&\mathbf{S}^{\alpha\beta}(z) \mathbf{S}^{\mu\nu}(w) \\
&= b_1^\alpha U_1 P_1 (z-Q_1)^{-1} T_1^{-1} a_1^\beta b_2^\mu U_2 P_2 (w-Q_2)^{-1} T_2^{-1} a_2^\nu \\
&= b_2^\mu b_1^\alpha U_2 P_2 U_1 P_1 \underline{R}_{21} (z-Q_1+\hbar X)^{-1} \bar R_{12}^{-1} (w-Q_2)^{-1} R_{12} T_2^{-1} T_1^{-1} a_2^\nu a_1^\beta \\
&- \hbar \delta^{\mu\beta} \mathbf{b}^\alpha \mathbf{P} (z-Q)^{-1} \mathbf{W} \mathbf{P} (w-Q)^{-1} \mathbf{a}^\nu \\
&= b_2^\mu b_1^\alpha U_2 P_2 U_1 P_1 (w-Q_2+\hbar X)^{-1} \bar R_{21}^{-1} (z-Q_1)^{-1} T_2^{-1} T_1^{-1} a_2^\nu a_1^\beta \\
&+ \frac{\hbar}{2\pi\I} \oint_C b_2^\mu b_1^\alpha U_2 P_2 U_1 P_1 (w-Q_2+\hbar X)^{-1} (\tau-Q_2-\hbar X+\epsilon)^{-1} (\tau-Q_1)^{-1} \bar R_{21}^{-1} (z-Q_1)^{-1} T_2^{-1} T_1^{-1} a_2^\beta a_1^\nu d\tau - (z \leftrightarrow w) \\
&- \frac{\hbar}{2\pi\I} \oint_C \delta^{\mu\beta} \sum_\rho \mathbf{b}^\alpha \mathbf{P} (z-Q)^{-1} (\tau-Q)^{-1} \mathbf{a}^\rho \mathbf{b}^\rho (\tau-Q+\gamma\hbar)^{-1} \mathbf{P} (w-Q)^{-1} \mathbf{a}^\nu d\tau
\end{align*}
}
where the limit $\epsilon \to 0$ is implied. This becomes:
{
\small
\begin{align*}
&\mathbf{S}^{\alpha\beta}(z) \mathbf{S}^{\mu\nu}(w) \\
&= \mathbf{S}^{\mu\nu}(w) \mathbf{S}^{\alpha\beta}(z) \\
&+ \frac{\hbar}{2\pi\I} \oint_C \delta^{\alpha\nu} \sum_\rho \mathbf{b}^\mu \mathbf{P} (w-Q)^{-1} (\tau-Q)^{-1} \mathbf{a}^\rho \mathbf{b}^\rho \mathbf{P} (\tau-Q+(\gamma+1)\hbar)^{-1} (z-Q)^{-1} \mathbf{a}^\beta d\tau - (z \leftrightarrow w, \alpha \leftrightarrow \mu, \beta \leftrightarrow \nu) \\
&+ \frac{\hbar}{2\pi\I} \oint_C \mathbf{b}^\mu \mathbf{P} (w-Q)^{-1} (\tau-Q+\epsilon)^{-1} \mathbf{a}^\beta \mathbf{b}^\alpha \mathbf{P} (\tau-Q+\hbar)^{-1} (z-Q)^{-1} \mathbf{a}^\nu d\tau - (z \leftrightarrow w) \\
&+ \frac{\hbar}{2\pi\I} \oint_C \frac{\hbar}{2\pi\I} \oint_C \delta^{\alpha\beta} \sum_\rho \mathbf{b}^\mu \mathbf{P} (w-Q)^{-1} (\tau-Q+\epsilon)^{-1} (\sigma-Q)^{-1} \mathbf{a}^\rho \mathbf{b}^\rho \mathbf{P} (\sigma-Q+(\gamma+1)\hbar)^{-1} (\tau-Q+\hbar)^{-1} (z-Q)^{-1} \mathbf{a}^\nu d\tau d\sigma - (z \leftrightarrow w)
\end{align*}
}
Let us define the divided differences
\begin{equation*}
\mathbf{S}^{\mu\rho}(\tau;w) := \frac{\mathbf{S}^{\mu\rho}(\tau)-\mathbf{S}^{\mu\rho}(w)}{\tau-w} = \mathbf{b}^\mu \mathbf{P} (\tau-Q)^{-1} (w-Q)^{-1} \mathbf{a}^\rho
\end{equation*}
and
\begin{equation*}
\mathbf{S}^{\mu\rho}(\tau;\sigma;w) := \frac{\mathbf{S}^{\mu\rho}(\tau;\sigma)-\mathbf{S}^{\mu\rho}(\sigma;w)}{\tau-w} = \mathbf{b}^\mu \mathbf{P} (\tau-Q)^{-1} (\sigma-Q)^{-1} (w-Q)^{-1} \mathbf{a}^\rho.
\end{equation*}
Notice
\begin{align*}
\frac{1}{2\pi\I} \oint_\infty \mathbf{S}^{\mu\rho}(\tau;w) d\tau = \mathbf{S}^{\mu\rho}(w).
\end{align*}
So if we take 
The contour integrals have to be taken carefully using \emph{left} residues, meaning that all $\mathbf{P}$ have to be on the left of all $Q$. In the case where we have terms where all poles are $q_i$s, we can exchange the contour $C$ to be the contour around infinity. We can also write
\begin{align*}
\mathbf{S}^{\mu\rho}(\tau;w) \mathbf{S}^{\mu\rho}(\tau+(\gamma+1)\hbar;z) = (z-w)^{-1}(\mathbf{S}^{\mu\rho}(\tau;w) - \mathbf{S}^{\mu\rho}(\tau+(\gamma+1)\hbar;z))
\end{align*}

%\pagebreak
%
%If $\mathbf{P}$ shifts by $\hbar\gamma$, then we would have nice behaviour and cancellations on the diagonal, giving
%\begin{align*}
%&[\mathbf{S}^{\alpha\beta}(z),\mathbf{S}^{\mu\nu}(w)] \\
%&=\hbar \delta^{\alpha\nu} \sum_{ij} \sum_\rho \mathbf{b}_i^\mu \mathbf{P}_i (z-q_i)^{-1} \mathbf{a}_i^\rho (q_i-q_j+\hbar\gamma)^{-1} \mathbf{b}_j^\rho \mathbf{P}_j (w-q_j)^{-1} \mathbf{a}_j^\beta \\
%&-\hbar \sum_{ij} \mathbf{b}_i^\alpha \mathbf{P}_i (z-q_i)^{-1} \mathbf{a}_i^\nu (q_i-q_j+\hbar\gamma)^{-1} \mathbf{b}_j^\mu \mathbf{P}_j (w-q_j)^{-1} \mathbf{a}_j^\beta \\
%&+\hbar \sum_{ij} \mathbf{b}_i^\alpha \mathbf{P}_i (w-q_i)^{-1} \mathbf{a}_i^\nu (q_i-q_j+\hbar\gamma)^{-1} \mathbf{b}_j^\mu \mathbf{P}_j (z-q_j)^{-1} \mathbf{a}_j^\beta \\
%&-\hbar^2 \delta^{\mu\nu} \sum_{ij} (q_i-q_j)^{-1} \sum_\rho \mathbf{b}_i^\alpha \mathbf{P}_i (z-q_i)^{-1} \mathbf{a}_i^\rho (q_i-q_j+\hbar\gamma)^{-1} \mathbf{b}_j^\rho \mathbf{P}_j (w-q_j)^{-1} \mathbf{a}_j^\beta \\
%&+\hbar^2 \delta^{\mu\nu} \sum_{ij} (q_i-q_j)^{-1} \sum_\rho \mathbf{b}_i^\alpha \mathbf{P}_i (w-q_i)^{-1} \mathbf{a}_i^\rho (q_i-q_j+\hbar\gamma)^{-1} \mathbf{b}_j^\rho \mathbf{P}_j (z-q_j)^{-1} \mathbf{a}_j^\beta \\
%&-\hbar \delta^{\mu\beta} \sum_{ij} \sum_\rho \mathbf{b}_i^\alpha \mathbf{P}_i (z-q_i)^{-1} \mathbf{a}_i^\rho (q_i-q_j+\hbar\gamma)^{-1} \mathbf{b}_j^\rho \mathbf{P}_j (w-q_j)^{-1} \mathbf{a}_j^\nu,
%\end{align*}
%which is supposed to reduce to
%\begin{equation*}
%R_{ba}(w-z) \mathbf{S}_a(z) R_{ab}(z-w+\hbar\gamma) \mathbf{S}_b(w) = \mathbf{S}_b(w) R_{ba}(w-z+\hbar\gamma) \mathbf{S}_a(z) R_{ab}(z-w).
%\end{equation*}
%In this case, we should have the solution where $\mathbf{S}(z)$ is a given by an auxiliary strand wrapping around the quantum strands and with a twist that is the shift operator
%\begin{equation*}
%\pi(z) := \sum_i e^{\gamma\hbar\partial_i} (z-q_i)^{-1}
%\end{equation*}

%\subsection{Commutation relations for invariants}
%
%We introduce $\mathbf{a}^\alpha := t^{-1} T^{-1} a^\alpha$ and $\mathbf{b}^\alpha := b^\alpha T t$. Then
%\begin{align*}
%[\mathbf{b}_i^\alpha,\mathbf{b}_j^\beta]
%&= -\hbar (1-\delta_{ij}) \frac{1}{q_{ij}} \operatorname{Tr}_{12} (b^\beta T)_j (b^\alpha T)_i e_{ii} \otimes e_{jj} t_j t_i \\
%&+\hbar \delta_{ij} \sum_{i \neq l} \frac{1}{q_{il}} \operatorname{Tr}_{12} b_2^\beta T_2 b_1^\alpha T_1 e_{ii} \otimes e_{lj} t_j t_i \\
%&+ \hbar \delta_{ij} \sum_{k \neq i} \frac{1}{q_{ki}} \operatorname{Tr}_{12} b_2^\beta T_2 b_1^\alpha T_1 e_{ki} \otimes e_{ij} t_j t_i \\
%&- \hbar (1-\delta_{ij}) \frac{1}{q_{ji}} \operatorname{Tr}_{12} b_2^\beta T_2 b_1^\alpha T_1 e_{ji} \otimes e_{ij} t_j t_i
%\end{align*}
%and
%\begin{align*}
%\mathbf{P}_i \mathbf{S}_j^{\alpha\beta}
%&= b_2^\alpha u_1^{-1} U_2 P_2 (e_{ii})_1 \bar R_{21}^{-1} (e_{jj})_2 P_1 t_1 T_2^{-1} a_2^\beta \\
%&+ b_2^\alpha u_1^{-1} U_2 (e_{ii})_1 P_1 \bar R_{21}^{-1} [t_1,P_2] (e_{jj})_2 T_2^{-1} a_2^\beta
%\end{align*}
%
%Note that $\operatorname{tr} \mathbf{S}_i$ satisfies
%\begin{align*}
%[\operatorname{tr} \mathbf{S}_i, \operatorname{tr} \mathbf{S}_j]
%&= \hbar \sum_\rho \sum_{\alpha} \mathbf{S}_j^{\alpha\rho} \frac{1}{q_{ji}+\hbar\gamma} \mathbf{S}_i^{\rho\alpha} - \hbar \sum_\rho \sum_{\alpha} \mathbf{S}_i^{\alpha\rho} \frac{1}{q_{ij}+\hbar\gamma} \mathbf{S}_j^{\rho\alpha} \\
%&+ (1-\delta_{ij}) \frac{\hbar}{q_{ij}} \bigg( \cdots \bigg)
%\end{align*}
%and you can see that everything cancels when you sum over $i$ and $j$.

\subsection{Representation with spectral parameter}

We want to represent the relation
\begin{equation*}
\mathbf{S}_a(z) R_{ab}(z-w+\hbar\gamma) \mathbf{S}_b(w) R_{ba}(w-z) = R_{ab}(z-w) \mathbf{S}_b(w) R_{ba}(w-z+\hbar\gamma) \mathbf{S}_a(z)
\end{equation*}
or
\begin{equation*}
\mathbf{S}_a(z) \check R_{ab}(z-w+\hbar\gamma) \mathbf{S}_a(w) \check R_{ba}(w-z) = \check R_{ab}(z-w) \mathbf{S}_a(w) \check R_{ba}(w-z+\hbar\gamma) \mathbf{S}_a(z).
\end{equation*}
Let $\Pi(z) := \sum_i e^{\hbar\partial_i} (z-q_i)^{-1}$, which represents $e \mathbf{P} (z-Q)^{-1} e$.

The operator $\mathbf{S}(z)$ should look like an auxiliary space wrapping around the quantum spaces with a twist, and the shift by $\gamma\hbar$ is due to the framing anomaly, bending by an angle proportional to $\gamma$, kind of like a reflection equation with spectral parameters $u-v=0, u+v=\gamma$, meaning $u=v=\frac{\gamma}{2}$. Let us set
\begin{equation*}
\mathbf{S}_a(z) = \check R_{a1}(z-q_1) \cdots \check R_{N-1,N}(z-q_N) \Pi(z) \check R_{N-1,N}(q_N-z) \cdots \check R_{a1}(q_1-z).
\end{equation*}
Indeed we have for $N=1$ without any shifts
\begin{align*}
&[\check R_{a1}(z-q_1) \check R_{1a}(q_1-z)] \check R_{ba}(w-z) [\check R_{a1}(w-q_1) \check R_{1a}(q_1-w)] \check R_{ab}(z-w) \\
&= \check R_{a1}(z-q_1) \check R_{ba}(w-q_1) \check R_{1a}(w-z) \check R_{ba}(q_1-z) \check R_{a1}(q_1-w) \check R_{ab}(z-w) \\
&= \check R_{ba}(w-z) \check R_{a1}(w-q_1) \check R_{ab}(z-q_1) \check R_{a1}(z-w) \check R_{ab}(q_1-w) \check R_{1a}(q_1-z) \\
&= \check R_{ba}(w-z) [\check R_{a1}(w-q_1) \check R_{1a}(q_1-w)] \check R_{ab}(z-w) [\check R_{a1}(z-q_1) \check R_{1a}(q_1-z)]
\end{align*}
With $\gamma = \ell$, we can again let the auxiliary space loop around with a twist, giving a shift by $\ell\hbar$.

We can solve the RSRS equation with spectral parameter easily by
\begin{equation*}
\mathbf{S}(z) = T^\vee(z) T(z) e^{-\gamma\hbar \partial_z}
\end{equation*}
since
\begin{align*}
&R_{ba}(w-z) T_a^\vee(z) T_a(z) e^{-\gamma\hbar\partial_z} R_{ab}(z-w+\gamma\hbar) T_b^\vee(w) T_b(w) e^{-\gamma\hbar\partial_w} \\
&= R_{ba}(w-z) T_a^\vee(z) T_a(z) R_{ab}(z-w) T_b^\vee(w) T_b(w) e^{-\gamma\hbar\partial_z} e^{-\gamma\hbar\partial_w} \\
&= R_{ba}(w-z) T_a^\vee(z) T_b^\vee(w) R_{ab}(z-w) T_a(z) T_b(w) e^{-\gamma\hbar\partial_z} e^{-\gamma\hbar\partial_w} \\
&= T_b^\vee(w) T_a^\vee(z) R_{ba}(w-z) T_b(w) T_a(z) R_{ab}(z-w) e^{-\gamma\hbar\partial_z} e^{-\gamma\hbar\partial_w} \\
&= T_b^\vee(w) T_b(w) R_{ba}(w-z) T_a^\vee(z) T_a(z) R_{ab}(z-w) e^{-\gamma\hbar\partial_z} e^{-\gamma\hbar\partial_w} \\
&= T_b^\vee(w) T_b(w) e^{-\gamma\hbar\partial_w} R_{ba}(w-z+\gamma\hbar) T_a^\vee(z) T_a(z) e^{-\gamma\hbar\partial_z} R_{ab}(z-w)
\end{align*}
Recall that $T(z) e^{-\gamma\hbar \partial_z}$ is a Manin matrix. But this does not give the right commutation relations with $Q$ and $T^\vee(z) T(z)$ can become a very trivial object.

\subsection{Units}

Note that $Q$ has dimensions of $\hbar$, which is indeed correct. To give it units of position let us rescale $Q \to \frac{1}{\mu} Q$, where $\mu$ has units of momentum, so that the RSRS relation becomes
\begin{equation*}
R_{ab}(q_{ji}) \mathbf{S}_i^a R_{ab}(q_{ij}+\tfrac{\hbar\gamma}{\mu}) \mathbf{S}_j^b = \mathbf{S}_j^b R_{ab}(q_{ji}+\tfrac{\hbar\gamma} {\mu}) \mathbf{S}_i^a R_{ab}(q_{ij}),
\end{equation*}
with $R(z) = 1+\frac{\hbar}{\mu z} P$. Let $\eta = \frac{\hbar\gamma}{2\mu}$ with $e^{-\eta\partial_i}$. For this to be nice, let $\mu = -\I \frac{mc}{2}$, giving $e^{\frac{\gamma}{mc} \hat p_i}$.

%We know that there is a factorization $\mathbf{S}_i = \mathbf{b}_i \mathbf{P}_i \mathbf{a}_i$.
%
%Introduce $S(z)$ and consider the relation
%\begin{equation*}
%R^{12}(w-z) S^1(z) R^{12}(z-w+\gamma) S^2(w) = S^2(w) R^{12}(w-z+\gamma) S^1(z) R^{12}(z-w).
%\end{equation*}
%Let $B(z) := T(z) T^{-1}(z-\gamma)$, where $T(z)$ generates the Yangian. Then:
%\begin{align*}
%R^{12}(z-w) B^1(z) R^{12}(w-z+\gamma) B^2(w) = B^2(w) R^{12}(w-z+\gamma) B^1(z) R^{12}(z-w)
%\end{align*}
%Note that the order of $z$ and $w$ is wrong on the LHS. This can only be corrected if $S(z)$ contains some difference operator that permutes $z$ and $w$. Maybe something like $e^{\hbar(y_i \partial_j - y_j \partial_i)}$?
%
%When $\ell=1$, the relation reduces to
%\begin{align*}
%\frac{q_{ji}+\hbar}{q_{ji}} \mathbf{S}_i \frac{q_{ij}+\hbar+\gamma}{q_{ij}+\gamma} \mathbf{S}_j = \mathbf{S}_j \frac{q_{ji}+\hbar+\gamma}{q_{ji}+\gamma} \mathbf{S}_i \frac{q_{ij}+\hbar}{q_{ij}}
%\end{align*}
%This is solved by $\mathbf{S}_i = e^{-\gamma \partial_i},\mathbf{S}_j = e^{-\gamma \partial_j}$.
%
%Let's take
%\begin{equation*}
%\mathbf{S}_i^1 := R_i^1(z-q_i), \quad \mathbf{S}_j^2 := R_j^2(w-q_j),
%\end{equation*}
%so
%\begin{equation*}
%R^{12}(q_{ji}) \mathbf{S}_i^1 R^{12}(q_{ij}+\gamma) \mathbf{S}_j^2
%= R^{12}(q_{ji}) R_i^1(z-q_i) R^{12}(q_{ij}+\gamma) R_j^2(w-q_j)
%\end{equation*}
%
%In Gorsky's duality paper, there are operators of the form
%\begin{equation*}
%K_i := R_{i,i-1}(q_{i,i-1}+\gamma) \cdots R_{i1}(q_{i1}+\gamma) e_i^{\alpha\beta} R_{iN}(q_{iN}) \cdots R_{i,i+1}(q_{i,i+1}).
%\end{equation*}
%These are commuting up to a shift and fulfill
%\begin{equation*}
%\check R_{i+1,i}(q_{i+1,i}+\gamma) K_i = K_{i+1} \check R_{i,i+1}(q_{i,i+1}).
%\end{equation*}
%Here, we also see that we need a difference operator that permutes $q_i$ and $q_{i+1}$ to match the RSRS relation.
%
%Let us determine the commutation relations of $\mathbf{S}_i^{\alpha\beta}$ with $\mathbf{P}_j$:
%\begin{align*}
%\mathbf{S}_i^{\alpha\beta} \mathbf{P}_j
%&= b_1^\alpha (UP)_1 (e_{ii})_1 T_1^{-1} a_1^\beta u_2^{-1} P_2 (e_{jj})_2 t_2 \\
%&= b_1^\alpha u_2^{-1} P_2 (UP)_1 \underline{R}_{21} (e_{ii})_1 \bar R_{12}^{-1} (e_{jj})_2 R_{12} t_2 T_1^{-1} a_1^\beta \\
%&= \mathbf{P}_j \mathbf{S}_i^{\alpha\beta} \\
%&+ (1-\delta_{ij}) \frac{\hbar}{q_{ij}} \bigg( \mathbf{P}_j \mathbf{a}_j^\beta \mathbf{b}_i^\alpha - \mathbf{P}_i \mathbf{a}_i^\beta \mathbf{b}_j^\alpha + \delta^{\alpha\beta} \sum_\rho \mathbf{P}_j \mathbf{a}_j^\rho \mathbf{b}_i^\rho \frac{1}{q_{ji}+\gamma} - \delta^{\alpha\beta} \sum_\rho \mathbf{P}_i \mathbf{a}_i^\rho \mathbf{b}_j^\rho \frac{1}{q_{ij}+\gamma} \bigg)
%\end{align*}
%(The commutators of $b_1^\alpha$ with $u_2^{-1}$ and $t_2$ are missing here, but they might cancel) In particular, we have $\mathbf{S}_i^{\alpha\beta} \mathbf{P}_i = \mathbf{P}_i \mathbf{S}_i^{\alpha\beta}$. Choosing a representation in which
%\begin{equation*}
%R^{12}(q_{ij}+\gamma) = \mathbf{P}_i^{-1} R^{12}(q_{ij}) \mathbf{P}_j^{-1},
%\end{equation*}
%and introducing $\mathbf{T}_i := \mathbf{S}_i \mathbf{P}_i^{-1} = \mathbf{P}_i^{-1} \mathbf{S}_i$, we can then rewrite the RSRS relation as
%\begin{equation*}
%R^{12}(q_{ji}) \mathbf{T}_i^1 R^{12}(q_{ij}) \mathbf{T}_j^2 = \mathbf{T}_j^2 R^{12}(q_{ji}) \mathbf{T}_i^1 R^{12}(q_{ij}).
%\end{equation*}
%This just says that the operators $R^{12}(q_{ji}) \mathbf{T}_i^1 R^{12}(q_{ij})$ and $\mathbf{T}_j^2$ commute for $i \neq j$. We can also write it with the braid-like $R$-matrix $\check R(z) := P R(z)$:
%\begin{equation*}
%\check R^{12}(q_{ji}) \mathbf{T}_i^1 \check R^{12}(q_{ij}) \mathbf{T}_j^1 = \mathbf{T}_j^1 \check R^{12}(q_{ji}) \mathbf{T}_i^1 \check R^{12}(q_{ij}).
%\end{equation*}
%One way to find a representation of this algebra on $(\C^\ell)^{\otimes N} \otimes \C[y_1,...,y_N]$ is to look for a commutative family that is closed under conjugation by $\check R^{12}(q_{ij})$. Such a commutative family indeed exists and is given by the Laurent generators $X_1,...,X_N$ in the degenerate double affine Hecke algebra $d\ddot H_N$, another one are the the transfer matrix residues in the Bethe subalgebra.
%
%Let's consider the case $N=2$ and take
%$\mathbf{T}_1^1 \mapsto P_1^1 R_{12}(q_{12}) = P_{12} P_2^1 \check R_{12}(q_{12}), \mathbf{T}_2^1 \mapsto R_{12}(q_{21}) P_2^1 = \check R_{12}(q_{21}) P_1^1 P_{12}$. Then
%\begin{align*}
%\check R_{12}(q_{21}) \mathbf{T}_1^1 \check R_{12}(q_{12}) \mathbf{T}_2^1 = \check R_{12}(q_{21}) P_{12} P_2^1 \check R_{12}(q_{21}) P_1^2 P_{12}
%\end{align*}

%\subsection{Higher currents}
%
%We have $J_1^{\alpha\beta} = \sum_i S_i^{\alpha\beta}$ and for $N = 2$:
%\begin{align*}
%J_1^{\alpha\beta}
%&= \sum_i e_i^{\alpha\beta} \otimes X_i \\
%&= (e_1^{\alpha\beta} \otimes e^{\hbar \partial_1}) R_{12} + R_{21} (e_2^{\alpha\beta} \otimes e^{\hbar \partial_2}) \\
%&= \frac{y_1-y_2+\hbar}{y_1-y_2+\hbar-\eta} e_1^{\alpha\beta} \otimes e^{\hbar \partial_1}
%+ \frac{y_2-y_1}{y_2-y_1-\eta} e_2^{\alpha\beta} \otimes e^{\hbar \partial_2} \\
%&- \frac{\eta}{y_1-y_2+\hbar-\eta} \sum_\gamma e_1^{\alpha\gamma} e_2^{\gamma\beta} \otimes e^{\hbar \partial_1}
%- \frac{\eta}{y_2-y_1-\eta} \sum_\gamma e_1^{\alpha\gamma} e_2^{\gamma\beta} \otimes e^{\hbar \partial_2}
%\end{align*}
%We also have
%\begin{align*}
%J_2^{\alpha\beta}
%&= \sum_i e_i^{\alpha\beta} \otimes X_i^2 \\
%&= e_1^{\alpha\beta} \otimes e^{\hbar \partial_1} x_{12} e^{\hbar \partial_1} x_{12} + e_2^{\alpha\beta} \otimes x_{21} e^{\hbar \partial_2} x_{21} e^{\hbar \partial_2} \\
%&= (\frac{y_1-y_2+\hbar}{y_1-y_2+\hbar-\eta} e_1^{\alpha\beta} \otimes e^{2 \hbar \partial_1} - \frac{\eta}{y_1-y_2+\hbar-\eta} \sum_\gamma e_1^{\alpha\gamma} e_2^{\gamma\beta} \otimes e^{2 \hbar \partial_1}) R_{12} \\
%&+ R_{21} (\frac{y_2-y_1 + \hbar}{y_2-y_1+\hbar-\eta} e_2^{\alpha\beta} \otimes e^{2\hbar \partial_2} - \frac{\eta}{y_2-y_1+\hbar-\eta} \sum_\gamma e_1^{\alpha\gamma} e_2^{\gamma\beta} \otimes e^{2\hbar \partial_2}) \\
%&= \frac{y_1-y_2+2\hbar}{y_1-y_2+2\hbar-\eta} \frac{y_1-y_2+\hbar}{y_1-y_2+\hbar-\eta} e_1^{\alpha\beta} \otimes e^{2 \hbar \partial_1} - \frac{y_1-y_2+2\hbar}{y_1-y_2+2\hbar-\eta} \frac{\eta}{y_1-y_2+\hbar-\eta} \sum_\gamma e_1^{\alpha\gamma} e_2^{\gamma\beta} \otimes e^{2 \hbar \partial_1} \\
%&- \frac{\eta}{y_1-y_2+2\hbar-\eta} \frac{y_1-y_2+\hbar}{y_1-y_2+\hbar-\eta} \sum_\gamma e_1^{\alpha\gamma} e_2^{\gamma\beta} \otimes e^{2 \hbar \partial_1} - \frac{\eta}{y_1-y_2+2\hbar-\eta} \frac{\eta}{y_1-y_2+\hbar-\eta} e_1^{\alpha\beta} \otimes e^{2 \hbar \partial_1} \\
%&+ \frac{y_2-y_1}{y_2-y_1-\eta} \frac{y_2-y_1 + \hbar}{y_2-y_1+\hbar-\eta} e_2^{\alpha\beta} \otimes e^{2\hbar \partial_2} - \frac{y_2-y_1}{y_2-y_1-\eta} \frac{\eta}{y_2-y_1+\hbar-\eta} \sum_\gamma e_1^{\alpha\gamma} e_2^{\gamma\beta} \otimes e^{2\hbar \partial_2} \\
%&- \frac{\eta}{y_2-y_1-\eta} \frac{y_2-y_1 + \hbar}{y_2-y_1+\hbar-\eta} \sum_\gamma e_1^{\alpha\gamma} e_2^{\gamma\beta} \otimes e^{2\hbar \partial_2} - \frac{\eta}{y_2-y_1-\eta} \frac{\eta}{y_2-y_1+\hbar-\eta} e_2^{\alpha\beta} \otimes e^{2\hbar \partial_2}
%\end{align*}

\subsection{Loop Yangian}

Let $\mathbf{L} := \mathbf{WP}$ and $\mathbf{J}^{\alpha\beta}[n] := \mathbf{b}^\alpha \mathbf{W}^{-1} \mathbf{L}^n \mathbf{a}^\beta = \mathbf{b}^\alpha \mathbf{P} \mathbf{L}^{n-1} \mathbf{a}^\beta = b^\alpha UP (T^{-1}UP)^{n-1} T^{-1} a^\beta$. It is clear that $\mathbf{J}^{\alpha\beta}[0] = b^\alpha a^\beta$ satisfy the Lie algebra $\mathfrak{gl}_\ell$. Furthermore,
\begin{align*}
&\mathbf{J}^{\alpha\beta}[n] \mathbf{J}^{\mu\nu}[1] \\
&= b_1^\alpha (UP)_1 (T^{-1}UP)_1^{n-1} T_1^{-1} a_1^\beta b_2^\mu (UP)_2 T_2^{-1} a_2^\nu \\
&= b_2^\mu b_1^\alpha (UP)_2 (UP)_1 \underline{R}_{21} (\bar R_{12}^{-1} T_1^{-1} U_1 P_1 \underline{R}_{21})^{n-1} \bar R_{12}^{-1} R_{12} T_2^{-1} T_1^{-1} a_2^\nu a_1^\beta \\
&+ b_1^\alpha (UP)_1 (T_1^{-1} U_1 P_1)^{n-1} T_1^{-1} [a_1^\beta,b_2^\mu] (UP)_2 T_2^{-1} a_2^\nu \\
&= \mathbf{J}^{\mu\nu}[1] \mathbf{J}^{\alpha\beta}[n] + \hbar \delta^{\alpha\nu} \mathbf{J}^{\mu\beta}[n+1] - \hbar \delta^{\mu\beta} \mathbf{J}^{\alpha\nu}[n+1],
\end{align*}
showing that these satisfy the relations of the loop algebra $L(\mathfrak{gl}_\ell)$.

%Let $\mathbf{J}^{\alpha\beta} := \mathbf{b}^\alpha \mathbf{P} \mathbf{a}^\beta = b^\alpha U P T^{-1} a^\beta$. Then we find
%\begin{align*}
%\mathbf{J}_1^{\alpha\beta} \mathbf{J}_2^{\mu\nu}
%&= \mathbf{J}_2^{\mu\nu} \mathbf{J}_1^{\alpha\beta} + \hbar\delta^{\alpha\nu} \mathbf{b}^\mu \mathbf{P} \mathbf{L} \mathbf{a}^\beta - \hbar \delta^{\beta\mu} \mathbf{b}^\alpha \mathbf{P} \mathbf{L} \mathbf{a}^\nu
%\end{align*}
%where $\mathbf{L} := \mathbf{W} \mathbf{P}$. More generally, let $\mathbf{J}[n]^{\alpha\beta} := \mathbf{b}^\alpha \mathbf{P} \mathbf{L}^{n-1} \mathbf{a}^\beta$. Then we can rewrite this as
%\begin{equation*}
%[\mathbf{J}[1]^{\alpha\beta},\mathbf{J}[1]^{\mu\nu}] = \hbar \delta^{\alpha\nu} \mathbf{J}[2]^{\mu\beta} - \hbar \delta^{\beta\mu} \mathbf{J}[2]^{\alpha\nu}.
%\end{equation*}
%We also have $\mathbf{J}[0]^{\alpha\beta} = \mathbf{b}^\alpha \mathbf{W}^{-1} \mathbf{a}^\beta = b^\alpha a^\beta$. These clearly fulfill the relations of $\mathfrak{gl}_\ell$. Furthermore
%\begin{align*}
%\mathbf{J}[0]^{\alpha\beta} \mathbf{J}[1]^{\mu\nu}
%&= \mathbf{J}[1]^{\mu\nu} \mathbf{J}[0]^{\alpha\beta} + \hbar \delta^{\alpha\nu} \mathbf{J}[1]^{\mu\beta} - \hbar \delta^{\beta\mu} \mathbf{J}[1]^{\alpha\nu}
%\end{align*}

Also consider
\begin{equation*}
\mathbf{T}^{\alpha\beta}(z) = \delta^{\alpha\beta} + \mathbf{b}^\alpha \mathbf{W}^{-1} (z-Q)^{-1} \mathbf{a}^\beta = \delta^{\alpha\beta} + b^\alpha T (z-Q)^{-1} T^{-1}a^\beta
\end{equation*}
This gives
\begin{align*}
\mathbf{T}^{\alpha\beta}(z) \mathbf{T}^{\mu\nu}(w)
&= \delta^{\alpha\beta} \delta^{\mu\nu} + \delta^{\alpha\beta} b^\mu T (w-Q)^{-1} T^{-1} a^\nu + b^\alpha T (z-Q)^{-1} T^{-1} a^\beta \\
&+ b_2^\mu b_1^\alpha T_2 T_1 R_{12} (z-Q_1)^{-1} R_{21} (w-Q_2)^{-1} R_{12} T_2^{-1} T_1^{-1} a_2^\nu a_1^\beta \\
&- \hbar \delta^{\mu\beta} b^\alpha T (z-Q)^{-1} (w-Q)^{-1} T^{-1} a^\nu \\
&= \mathbf{T}^{\mu\nu}(w) \mathbf{T}^{\alpha\beta}(z) - \frac{\hbar}{z-w} (\mathbf{T}^{\mu\beta}(z) \mathbf{T}^{\alpha\nu}(w) - \mathbf{T}^{\mu\beta}(w) \mathbf{T}^{\alpha\nu}(z))
\end{align*}
due to $(z-Q)^{-1} (w-Q)^{-1} = (z-w)^{-1} (w-Q)^{-1} - (z-w)^{-1} (z-Q)^{-1}$ as well as (checked for $\ell=2,3,4$)
\begin{align*}
&R_{12} (z-Q_1)^{-1} R_{21} (w-Q_2)^{-1} R_{12}
= (w-Q_2)^{-1} R_{12} (z-Q_1)^{-1} \\
&- \frac{\hbar}{z-w} ((z-Q_2)^{-1} R_{12} (w-Q_1)^{-1} C_{12} R_{12} - (w-Q_2)^{-1} R_{12} (z-Q_1)^{-1} C_{12} R_{12}),
\end{align*}
showing that these satisfy the relations of the Yangian $Y(\mathfrak{gl}_\ell)$. All in all, we obtain the \emph{level $N$ loop Yangian} $LY_N(\mathfrak{gl}_\ell)$ as the ($S_N$-invariant) quantum Hamiltonian reduction, where we probably additionally have to fix the moment, which introduces the second deformation parameter. We conjecture that there are maps $LY(\mathfrak{gl}_\ell) \to LY_N(\mathfrak{gl}_\ell)$ that make the loop Yangian into their projective limit. The Hilbert space that we are considering then becomes a Fock space and the corresponding field theory should be related to Toda field theories, since an $N$-fold tensor product of Fock space reps of $LY$ gives $\widehat{\mathfrak{sl}}_N$ Toda field theory. Does $\gamma = -2$ correspond to the free fermion point?

%To match the loop algebra and Yangian relations to the conventions in my master's thesis, we need $\hbar = -\eta$. According to Kodera, we match $\hbar_{LY} = \eta$ and $\beta_{LY} = \hbar_\text{RS}/2-\ell \eta/4+\eta/2$. The correct matching to my master's thesis would result from $\mathbf{a}^\beta \to \frac{1}{\hbar}\mathbf{a}^\beta$ and $Q \to -\frac{\hbar}{\eta} Q$. This gives the relation
%\begin{align*}
%\mathbf{S}_i^{\alpha\beta} \mathbf{S}_j^{\mu\nu}
%&= \mathbf{S}_j^{\mu\nu} \mathbf{S}_i^{\alpha\beta}  - \sum_\rho \mathbf{S}_j^{\mu\rho} \frac{\eta\delta^{\alpha\nu}}{q_{ji}-\eta\gamma} \mathbf{S}_i^{\rho\beta} + \sum_\rho \mathbf{S}_i^{\alpha\rho} \frac{\eta\delta^{\beta\mu}}{q_{ij}-\eta\gamma} \mathbf{S}_j^{\rho\nu} \\
%&- (1-\delta_{ij}) \frac{\eta}{q_{ij}} \bigg( \mathbf{S}_j^{\mu\beta} \mathbf{S}_i^{\alpha\nu} - \sum_\rho \mathbf{S}_j^{\mu\rho} \frac{\eta\delta^{\alpha\beta}}{q_{ji}-\eta\gamma} \mathbf{S}_i^{\rho\nu} + \mathbf{S}_i^{\mu\beta} \mathbf{S}_j^{\alpha\nu} - \sum_\rho \mathbf{S}_i^{\mu\rho} \frac{\eta\delta^{\alpha\beta}}{q_{ij}-\eta\gamma} \mathbf{S}_j^{\rho\nu} \bigg)
%\end{align*}
%or
%\begin{equation*}
%R_{ab}(q_{ji}) \mathbf{S}_i^a R_{ab}(q_{ij}-\eta\gamma) \mathbf{S}_j^b = \mathbf{S}_j^b R_{ab}(q_{ji}-\eta\gamma) \mathbf{S}_i^a R_{ab}(q_{ij}), \quad \mathbf{S}_i^a R_{ab}(-\eta\gamma) \mathbf{S}_i^b = \mathbf{S}_i^b R_{ab}(-\eta\gamma) \mathbf{S}_i^a
%\end{equation*}
%where $R(z) = 1-\frac{\eta}{z} P$.

Let us consider the relation between $\mathbf{T}(z)$ and $\mathbf{J}[1]$:
\begin{align*}
\mathbf{T}^{\alpha\beta}(z) \mathbf{J}^{\mu\nu}[1]
&= \delta^{\alpha\beta} \mathbf{J}^{\mu\nu}[1] + b_1^\alpha T_1 (z-Q_1)^{-1} T_1^{-1} a_1^\beta b_2^\mu U_2 P_2 T_2^{-1} a_2^\nu \\
&= \delta^{\alpha\beta} \mathbf{J}^{\mu\nu}[1] + b_2^\mu b_1^\alpha U_2 P_2 T_1 \bar R_{12} (z-Q_1-\hbar\sum_i e_{ii} \otimes e_{ii})^{-1} \bar R_{12}^{-1} R_{12} T_2^{-1} T_1^{-1} a_2^\nu a_1^\beta \\
&- \hbar \delta^{\mu\beta} \mathbf{b}^\alpha \mathbf{W}^{-1} (z-Q)^{-1} \mathbf{L} \mathbf{a}^\nu \\
&= \cdots = \mathbf{J}^{\mu\nu}[1] \mathbf{T}^{\alpha\beta}(z-\hbar) + \hbar \delta^{\alpha\nu} \mathbf{b}^\mu \mathbf{W}^{-1} \mathbf{L} (z-\hbar-Q)^{-1} \mathbf{a}^\beta - \hbar \delta^{\mu\beta} \mathbf{b}^\alpha \mathbf{W}^{-1} (z-Q)^{-1} \mathbf{L} \mathbf{a}^\nu
\end{align*}
It looks like we should define the elements $\widehat{\mathbf{T}}^{\alpha\beta}[n](z) := \delta_{0n} \delta^{\alpha\beta} + \mathbf{b}^\alpha \mathbf{W}^{-1} \mathbf{L}^n (z-Q)^{-1} \mathbf{a}^\beta$.

It is unclear which loop Yangian this is, or rather which $\beta$ it corresponds to. However, $LY_{\eta,\beta} \cong LY_{\lambda\eta,\lambda\beta}$ for $\lambda \in \C^\times$, so we can assume $\beta = 1$ or $\beta = 0$. Note that we have $\hbar_\text{RS} = (\ell-2) \eta/4-\beta$. We also have a special point at $\beta = \frac{\eta}{2}$, which fits the representation fixing $\hbar = \pm \frac{\hbar\gamma}{2}$.

\section{Central extension of quantum current groups}

The \emph{Schwinger relation}
\begin{equation*}
[X^a(z),X^b(w)] = f_c^{ab} X^c(z) \delta(z-w) + c \delta^{ab} \delta'(z-w)
\end{equation*}
can be quantized to become
\begin{equation*}
L_1(z) R_{21}(w-z+\hbar c) L_2(w) R_{21}(w-z)^{-1} = R_{12}(z-w)^{-1} L_2(w) R_{12}(z-w+\hbar c) L_1(z).
\end{equation*}
This is usually solved by factorization
\begin{equation*}
L(z) = L^+(z+\tfrac{\hbar c}{2}) L^-(z)^t
\end{equation*}
where we have the relations
\begin{align*}
R_{12}^{++}(z-w) L_1^+(z) L_2^+(w) &= L_2^+(w) L_1^+(z) R_{12}^{++}(z-w) \\
R_{12}^{+-}(z-w+\tfrac{\hbar c}{2}) L_1^+(z) L_2^-(w) &= L_2^-(w) L_1^+(z) R_{12}^{+-}(z-w-\tfrac{\hbar c}{2}) \\
R_{12}^{--}(z-w) L_1^-(z) L_2^-(w) &= L_2^-(w) L_1^-(z) R_{12}^{--}(z-w)
\end{align*}
with
\begin{align*}
R^{++}(z) := R(z), \quad R^{+-}(z) := (R(z)^{t_2})^{-1}, \quad R^{--}(z) := R(z)^{t_1 t_2}.
\end{align*}
We have to be able to rewrite these as
\begin{align*}
R_{12}(z-w) L_1^+(z) L_2^+(w) &= L_2^+(w) L_1^+(z) R_{12}(z-w) \\
L_1^-(z)^t R_{21}(w-z+\tfrac{\hbar c}{2}) L_2^+(w) &= L_2^+(w) R_{21}(w-z-\tfrac{\hbar c}{2}) L_1^-(z)^t \\
L_1^-(z)^t L_2^-(w)^t R_{12}(z-w) &= R_{12}(z-w) L_2^-(w)^t L_1^-(z)^t.
\end{align*}
Then indeed for $L(z) := L^+(z+\tfrac{\hbar c}{2}) L^-(z)^t$, we get
\begin{align*}
&L_1(z) R_{21}(w-z+\hbar c) L_2(w) R_{12}(z-w) \\
&= L_1^+(z+\tfrac{\hbar c}{2}) L_1^-(z)^t R_{21}(w-z+\hbar c) L_2^+(w+\tfrac{\hbar c}{2}) L_2^-(w)^t R_{12}(z-w) \\
&= L_1^+(z+\tfrac{\hbar c}{2}) L_2^+(w+\tfrac{\hbar c}{2}) R_{21}(w-z) L_1^-(z)^t L_2^-(w)^t R_{12}(z-w) \\
&= R_{21}(w-z) L_2^+(w+\tfrac{\hbar c}{2}) L_1^+(z+\tfrac{\hbar c}{2}) R_{12}(z-w) L_2^-(w)^t L_1^-(z)^t \\
&= R_{21}(w-z) L_2^+(w+\tfrac{\hbar c}{2}) L_2^-(w)^t R_{12}(z-w+\hbar c) L_1^+(z+\tfrac{\hbar c}{2}) L_1^-(z)^t \\
&= R_{21}(w-z) L_2(w) R_{12}(z-w+\hbar c) L_1(z).
\end{align*}
These relations look like a central extension of the double Yangian. How do solutions of the middle equation look like? An easy approach is to use the shift operator $e^{\tfrac{\hbar c}{2} \partial_z}$.

Later they wrote a paper giving a presentation of the quantum affine algebra via the trigonometric $R$-matrix and generator matrices $L^\pm(z)$ (with conditions on $L^\pm(0)$) in the following way:
\begin{align*}
R(z/w) L_1^\pm(z) L_2^\pm(w) &= L_2^\pm(w) L_1^\pm(z) R(z/w) \\
R(q^cz/w) L_1^+(z) L_2^-(w) &= L_2^-(w) L_1^+(z) R(q^{-c}z/w).
\end{align*}
In terms of $L^-(z)^{-1}$, these get rewritten as
\begin{align*}
L_1^-(z)^{-1} L_2^-(w)^{-1} R(z/w) &= R(z/w) L_2^-(w)^{-1} L_1^-(z)^{-1} \\
L_2^-(w)^{-1} R(q^cz/w) L_1^+(z) &= L_1^+(z) R(q^{-c}z/w) L_2^-(w)^{-1}.
\end{align*}
Then the total $q$-deformed current is
\begin{equation*}
L(z) := L^+(q^{-\frac{c}{2}}z) L^-(zq^{\frac{c}{2}})^{-1} = 1 - (q-1) \sum_{\alpha\beta} e^{\alpha\beta} \otimes J^{\alpha\beta}(z) + \mathcal{O}((q-1)^2)
\end{equation*}
and satisfies the $q$-deformed Schwinger relation
\begin{equation*}
R_{12}(z/w) L_1(z) R_{12}(q^{2c}z/w)^{-1} L_2(w) = L_2(w) R_{12}(q^{-2c}z/w) L_2(w) R_{12}(z/w)^{-1}
\end{equation*}
or
\begin{equation*}
R_{12}(z/w) L_1(z) R_{21}(q^{-2c}w/z) L_2(w) = L_2(w) R_{12}(q^{-2c}z/w) L_2(w) R_{21}(w/z)
\end{equation*}
in the unitary case. We may obtain representations of $L^\pm(z)$ via $L^+(z) := R_{21}(q^{\frac{c}{2}}z), L^-(z) := R(q^{-\frac{c}{2}}z^{-1})$, giving
\begin{equation*}
L(z) = R_{21}(z) R(q^{-c}z^{-1})^{-1} = R_{21}(z) R_{21}(q^c z),
\end{equation*}
where the last step uses unitarity.

\subsection{Representations of the reduced algebra from CE of QCGs}

We want to solve the equation
\begin{equation*}
R_{ab}(q_{ji}) \mathbf{S}_i^a R_{ab}(q_{ij}+\gamma\hbar) \mathbf{S}_j^b = \mathbf{S}_j^b R_{ab}(q_{ji}+\gamma\hbar) \mathbf{S}_i^a R_{ab}(q_{ij}).
\end{equation*}
which can be derived from
\begin{equation*}
R_{ba}(w-z) \mathbf{S}_a(z) R_{ab}(z-w+\gamma\hbar) \mathbf{S}_b(w) = \mathbf{S}_b(w) R_{ba}(w-z+\gamma\hbar) \mathbf{S}_a(z).
\end{equation*}
when $\mathbf{S}_i := \mathbf{S}(q_i)$. In line with the above, we can make the following factorization ansatz
\begin{equation*}
\mathbf{S}(z) = \mathbf{S}^-(z-\tfrac{\gamma\hbar}{2}) \mathbf{S}^+(z),
\end{equation*}
with $\mathbf{S}^\pm(z)$ satisfying the relations
\begin{align*}
R_{ab}(z-w) \mathbf{S}_a^+(z) \mathbf{S}_b^+(w) &= \mathbf{S}_b^+(w) \mathbf{S}_a^+(z) R_{ab}(z-w) \\
\mathbf{S}_a^+(z) R_{ab}(z-w+\tfrac{\gamma\hbar}{2}) \mathbf{S}_b^-(w) &= \mathbf{S}_b^-(w) R_{ab}(z-w-\tfrac{\gamma\hbar}{2}) \mathbf{S}_a^+(z) \\
\mathbf{S}_a^-(z) \mathbf{S}_b^-(w) R_{ab}(z-w) &= R_{ab}(z-w) \mathbf{S}_b^-(w) \mathbf{S}_a^-(z).
\end{align*}
This looks like a central extension of the double Yangian, which is examined in a paper by Koroshkin. Its representations away from level zero are Fock spaces. Then indeed
\begin{align*}
&R_{ba}(w-z) \mathbf{S}_a(z) R_{ab}(z-w+\gamma\hbar) \mathbf{S}_b(w) \\
&= R_{ba}(w-z) \mathbf{S}_a^-(z-\tfrac{\gamma\hbar}{2}) \mathbf{S}_a^+(z) R_{ab}(z-w+\gamma\hbar) \mathbf{S}_b^-(w-\tfrac{\gamma\hbar}{2}) \mathbf{S}_b^+(w) \\
&= R_{ba}(w-z) \mathbf{S}_a^-(z-\tfrac{\gamma\hbar}{2}) \mathbf{S}_b^-(w-\tfrac{\gamma\hbar}{2}) R_{ab}(z-w) \mathbf{S}_a^+(z) \mathbf{S}_b^+(w) \\
&= \mathbf{S}_b^-(w-\tfrac{\gamma\hbar}{2}) \mathbf{S}_a^-(z-\tfrac{\gamma\hbar}{2}) R_{ba}(w-z) \mathbf{S}_b^+(w) \mathbf{S}_a^+(z) R_{ab}(z-w) \\
&= \mathbf{S}_b^-(w-\tfrac{\gamma\hbar}{2}) \mathbf{S}_b^+(w) R_{ba}(w-z+\gamma\hbar) \mathbf{S}_a^-(z-\tfrac{\gamma\hbar}{2}) \mathbf{S}_a^+(z) R_{ab}(z-w) \\
&= \mathbf{S}_b(w) R_{ba}(w-z+\gamma\hbar) \mathbf{S}_a(z) R_{ab}(z-w).
\end{align*}
Critical level is attained at $\gamma = -\ell$, which we have also observed to be a special case.

It is an interesting question whether we can insert a $K$-matrix in the middle that commutes with the $\mathbf{S}^\pm(z)$, viz.
\begin{equation*}
\mathbf{S}(z) = \mathbf{S}^-(z-\tfrac{\gamma\hbar}{2}) \mathbf{K}(z) \mathbf{S}^+(z),
\end{equation*}
giving
\begin{align*}
&R_{ba}(w-z) \mathbf{S}_a(z) R_{ab}(z-w+\gamma\hbar) \mathbf{S}_b(w) \\
&= R_{ba}(w-z) \mathbf{S}_a^-(z-\tfrac{\gamma\hbar}{2}) \mathbf{K}_a(z) \mathbf{S}_a^+(z) R_{ab}(z-w+\gamma\hbar) \mathbf{S}_b^-(w-\tfrac{\gamma\hbar}{2}) \mathbf{K}_b(w) \mathbf{S}_b^+(w) \\
&= R_{ba}(w-z) \mathbf{S}_a^-(z-\tfrac{\gamma\hbar}{2}) \mathbf{K}_a(z) \mathbf{S}_b^-(w-\tfrac{\gamma\hbar}{2}) R_{ab}(z-w) \mathbf{S}_a^+(z) \mathbf{K}_b(w) \mathbf{S}_b^+(w) \\
&= R_{ba}(w-z) \mathbf{S}_a^-(z-\tfrac{\gamma\hbar}{2}) \mathbf{S}_b^-(w-\tfrac{\gamma\hbar}{2}) \mathbf{K}_a(z) R_{ab}(z-w) \mathbf{K}_b(w) \mathbf{S}_a^+(z) \mathbf{S}_b^+(w) \\
&= \mathbf{S}_b^-(w-\tfrac{\gamma\hbar}{2}) \mathbf{S}_a^-(z-\tfrac{\gamma\hbar}{2}) R_{ba}(w-z) \mathbf{K}_a(z) R_{ab}(z-w) \mathbf{K}_b(w) \mathbf{S}_a^+(z) \mathbf{S}_b^+(w) \\
&= \mathbf{S}_b^-(w-\tfrac{\gamma\hbar}{2}) \mathbf{S}_a^-(z-\tfrac{\gamma\hbar}{2}) \mathbf{K}_b(w) R_{ba}(w-z) \mathbf{K}_a(z) R_{ab}(z-w) \mathbf{S}_a^+(z) \mathbf{S}_b^+(w) \\
&= \mathbf{S}_b^-(w-\tfrac{\gamma\hbar}{2}) \mathbf{S}_a^-(z-\tfrac{\gamma\hbar}{2}) \mathbf{K}_b(w) R_{ba}(w-z) \mathbf{K}_a(z) \mathbf{S}_b^+(w) \mathbf{S}_a^+(z) R_{ab}(z-w) \\
&= \mathbf{S}_b^-(w-\tfrac{\gamma\hbar}{2}) \mathbf{K}_b(w) \mathbf{S}_a^-(z-\tfrac{\gamma\hbar}{2}) R_{ba}(w-z) \mathbf{S}_b^+(w) \mathbf{K}_a(z) \mathbf{S}_a^+(z) R_{ab}(z-w) \\
&= \mathbf{S}_b(w) R_{ba}(w-z+\gamma\hbar) \mathbf{S}_a(z) R_{ab}(z-w).
\end{align*}

\pagebreak

\chapter*{Notation}\label{chapter:notation}
\addcontentsline{toc}{chapter}{Notation}

\pagebreak

\printindex

\bibliographystyle{alphaurl}
\bibliography{bibliography}
\addcontentsline{toc}{chapter}{Bibliography}

\pagebreak

\section*{Erklärung}

\end{document}






















%
